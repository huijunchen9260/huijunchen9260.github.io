\documentclass[12pt]{article}

% \usepackage[style=authoryear,maxbibnames=9,maxcitenames=2,uniquelist=false,backend=biber,doi=false,url=false]{biblatex}
\usepackage{natbib}
% \addbibresource{$BIB} % bibtex location
% \renewcommand*{\nameyeardelim}{\addcomma\space} % have comma in parencite

\usepackage{xcolor}
 \usepackage{amsmath}
\newcommand{\tuple}[1]{ \langle #1 \rangle }
%\usepackage{automata}
\usepackage{times}
\usepackage{ltablex}
\usepackage{natbib}
\usepackage{fontspec}
\usepackage{luatexja}
\usepackage[mathscr]{euscript}

%%%%%% Template
\usepackage{hyperref}
\hypersetup{colorlinks=true,allcolors=blue}

\usepackage{vmargin}
\setpapersize{USletter}
\setmarginsrb{1.0in}{1.0in}{1.0in}{0.6in}{0pt}{0pt}{0pt}{0.4in}

% HOW TO USE THE ABOVE:
%\setmarginsrb{leftmargin}{topmargin}{rightmargin}{bottommargin}{headheight}{headsep}{footheight}{footskip}
%\raggedbottom
% paragraphs indent & skip:
\parindent  0.3cm
\parskip    -0.01cm

\usepackage{tikz}
\usetikzlibrary{backgrounds}

% hyphenation:
\sloppy

% notes-style paragraph spacing and indentation:
\usepackage{parskip}
\setlength{\parindent}{0cm}

% let derivations break across pages
\allowdisplaybreaks

\def\blue{\color{blue}}
\def\orange{\color{orange}}

\def\qqquad{\quad\qquad}
\def\qqqquad{\qquad\qquad}

\setmainfont{Crimson Pro Light}[
  ItalicFont={* Italic},
  BoldFont={Crimson Pro Medium},
  BoldItalicFont={Crimson Pro Medium Italic}]
\setsansfont{Crimson Pro Light}[
  ItalicFont={* Italic},
  BoldFont={Crimson Pro Medium},
  BoldItalicFont={Crimson Pro Medium Italic}]

%%%%%%%%%%%%%%%%%%%%%%%%%%%%%%%%%%%%%%%%%%%%%%%%%%%%%%%%%%%%%%%%%%%%%%%%%%%%%%%%
%%%%%%%%%%%%%%%%%%%%%%%%%%%%%%%%%%%%%%%%%%%%%%%%%%%%%%%%%%%%%%%%%%%%%%%%%%%%%%%%
\begin{document}

\centerline{\huge\bf Syllabus: Seminar on Macroeconomics}
\medskip
\centerline{\LARGE \bf Spring 2026}
\medskip
\centerline{\Large Instructor: Hui-Jun Chen}
\centerline{Last Update: \today}
\centerline{Lastest Version: \href{https://huijunchen9260.github.io/pdf/PhDMacroSpring2026/syllabus/syllabus.pdf}{Click Here}}

\medskip

% \tableofcontents

% \newpage

\section*{Course Overview}
\addcontentsline{toc}{section}{Course Overview}
\begin{itemize}

    \item Course website:
    \begin{itemize}

       \item Materials: TBA
    \end{itemize}
    \item Meeting Time: Wednesday 10:10 - 13:00 (W34n)
    \item Location: TSMC Building 732
    \item Office: TSMC Building 729-B
    \item Email address: \href{huijunchen@mx.nthu.edu.tw}{huijunchen@mx.nthu.edu.tw}
    \item Please do not hesitate to email me and set an appointment outside of regular office hour.
To get quicker email reply, I would prefer you to:
    \begin{enumerate}
        \item Use \texttt{[Macro]} at the beginning of your subject title.
\begin{itemize}
            \item example title: \texttt{[Macro] Question regarding Extra credit}
        \end{itemize}
    \end{enumerate}
    \item I will reply your email within \textit{2 business day}.
\item Office hour: By appointment
    \item Teaching Assistant: TBA
    % \begin{enumerate}
    %     \item 陳建廷(email: \href{ericchen904230331@gmail.com}{ericchen904230331@gmail.com})
    %     \item 曾子齊(email: \href{freeboard2001@gmail.com}{freeboard2001@gmail.com})
    % \end{enumerate}
    \item TA Office hour: TBA
\end{itemize}

\newpage

\section*{Grades}
\addcontentsline{toc}{section}{Grades}

\newlength\q
\setlength\q{\dimexpr .5\textwidth -2\tabcolsep}
\begin{tabular}{|p{\q}|p{\q}|}
    \hline
    Categories  & Points \\
    \hline
    \hline
    % Problem sets on course material   & 20 points \\
    % \hline
    Midterm Exam I & 30 points \\
    \hline
    Midterm Exam II & 30 points \\
    \hline
    Final Exam & 30 points \\
    \hline
    Attendance & 10 points \\
    \hline
    Total & 100 points \\
    \hline
\end{tabular}
\textit{See course schedule, below, for due dates} \\


\section*{Grading Policy}
\addcontentsline{toc}{section}{Grading Policy}

% \subsection*{Quizzes / Exams}
% \addcontentsline{toc}{subsection}{Quizzes / Exams}

% Weekly quizzes in this class: \underline{Calculus materials}.
% You will have \textbf{unlimited} attempts and \textbf{unlimited time} per attempt for quizzes of Calculus materials.
% When calculating the final grade, I will \textbf{drop two quiz with the lowest grade} in each category (except Quiz on Calculus, Ch. 10 \& 11).
% The exact due date and time for quizzes and exams should refer to the schedule below and the setting on Carmen.
% In principle, all quizzes are due on \textbf{Sunday 11:59pm}, and the answer is available on \textbf{next Monday}.
% Late quizzes are not accepted, unless you have formal excuse (require formal documentation, and the \textbf{instrutor still has rights to decide whether to extend the quiz / exams for this excuse}).
% Final exam are cumulative, so the content from the midterm is also included in final exam.
% \section*{Quizzes and Examinations Integrity Policies}
% \addcontentsline{toc}{section}{Quizzes and Examinations Integrity Policies}
\section*{Examinations Integrity Policies}
\addcontentsline{toc}{section}{Examinations Integrity Policies}

\textbf{Examinations}: Discussions are \textbf{forbidden}, either face to face or via online discussion board / Social media.
% \subsection*{Extra credits}

% Extra credits: At the end of the semester, I will have you do the Student Evaluation of Instruction (SEI).
If I get $80\%$ response rate on SEI by the end of the semester, everyone will get 2 points of extra credits.

\subsection*{Curving}
\addcontentsline{toc}{subsection}{Curving}

If less than $40\%$ of the students get A- or above, I will add some points to everybody until $40\%$ of the students get A- or above, but I don't expect this to occur.

\subsection*{Attendance}
\label{sub:Attendance}
\addcontentsline{toc}{subsection}{Attendance}

If there are more than $ 50\% $ of students attend the class, then I will not take attendance.
If I have not taken attendance until the end of semester, then every students will be granted the entirety of the attendance points.
If I start to take attendance, then I will spread $ 5 $ to $ 10 $ attendance check across the rest of the semester, and the attendance points will be recorded accordingly.


\section*{Tentative Course Schedule}
\addcontentsline{toc}{section}{Tentative Course Schedule}

\newlength\bb
\setlength\bb{\dimexpr .08\textwidth -2\tabcolsep}
\newlength\qq
\setlength\qq{\dimexpr .14\textwidth -2\tabcolsep}
\newlength\rr
\setlength\rr{\dimexpr .3\textwidth -2\tabcolsep}
\newlength\pp
\setlength\pp{\dimexpr .84\textwidth -2\tabcolsep}
\begin{tabular}{|p{\bb}|p{\bb}|p{\pp}|}
    \hline
        Week & Day & Topics and Readings \\
    \hline
    \hline
        1
        &
        2/25
        &
        Topic: Introduction and Dynamic Programming
    \\
    \hline
        2
        &
        3/4
        &
        Topic: One-side Labor Search Model
    \\
    \hline

        3
        &
        3/11
        &
        Topic: Two-side Labor Search Model
    \\
    \hline
        4
        &
        3/18
        &
        Topic: Lucas Asset Pricing: Endowment Economy
    \\
    \hline
        5
        &
        3/25
        &
        Topic: Lucas Asset Pricing: Production Economy
    \\
    \hline
        6
        &
        4/1
        &
        Midterm I
    \\
    \hline

        7
        &
        4/8
        &
        Midterm Recap
        \newline
        Topic: Real Business Cycle Model I
    \\
    \hline
        8
        &
        4/15
        &
        Topic: Real Business Cycle Model II
    \\
    \hline
        9
        &
        4/22
        &
        Topic: Endogenous Growth Model I
    \\
    \hline
        10
        &
        4/29
        &
        Topic: Endogenous Growth Model II
    \\
    \hline
        11
        &
        5/6
        &
        Midterm II
    \\
    \hline
        12
        &
        5/13
        &
        Midterm Recap
        \newline
        Topic: Heterogeneous Agent Model I
    \\
    \hline
        13
        &
        5/20
        &
        Topic: Heterogeneous Agent Model II
    \\
    \hline
        14
        &
        5/27
        &
        Topic: Heterogeneous Agent Model III
    \\
    \hline
        15
        &
        6/3
        &
        Topic: Heterogeneous Agent Model IV
        \newline
        Final Review
    \\
    \hline
        16
        &
        6/10
        &
        Final Exam
    \\
    \hline
\end{tabular}

\section*{Grading scale}

See \url{https://registra.site.nthu.edu.tw/var/file/211/1211/img/609/grade-plan.pdf} for NTHU definition.
\end{document}
