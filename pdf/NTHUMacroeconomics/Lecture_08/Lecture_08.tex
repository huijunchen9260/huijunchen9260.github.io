\documentclass[11pt,aspectratio=169,usenames,dvipsnames]{beamer}
\usetheme{SimplePlus}

\usepackage{threeparttable}
\usepackage{booktabs}
\usepackage{xcolor} % For custom colors
\usepackage{tikz} % For styling enumerate numbers
\usepackage{tcolorbox} % For colored box styling
\usepackage{amsmath, amsfonts, amssymb, amsthm} % Math related
\usepackage{natbib}
\usepackage{fontspec}
\usepackage{luatexja}
\usepackage[mathscr]{euscript}

% ---------------- %
% color definition %
% ---------------- %
\definecolor{main}{HTML}{23373B}
\definecolor{pink}{RGB}{180, 50, 110}
\definecolor{orange}{HTML}{FF8000}
\definecolor{red}{HTML}{990000}
\definecolor{blue}{HTML}{004C99}
\definecolor{lightgray}{HTML}{E7E7E7}
\definecolor{gray}{RGB}{90, 90, 90}

\newcommand{\pink}[1]{\textcolor{pink}{#1}}
\newcommand{\orange}[1]{\textcolor{orange}{#1}}
\newcommand{\red}[1]{\textcolor{red}{#1}}
\newcommand{\blue}[1]{\textcolor{blue}{#1}}
\newcommand{\green}[1]{\textcolor{OliveGreen}{#1}}
\newcommand{\magenta}[1]{\textcolor{magenta}{#1}}
\newcommand{\gray}[1]{\textcolor{gray}{#1}}
\newcommand{\purple}[1]{\textcolor{purple}{#1}}
\definecolor{yellow}{HTML}{EDB120}

% \setbeamercolor{alerted text}{fg=blue}

%%% automatically add spaces into enumerate and itemize environment
\let\tempone\itemize
\let\temptwo\enditemize
\renewenvironment{itemize}{\tempone\addtolength{\itemsep}{\fill}}{\temptwo}
\let\tempa\enumerate
\let\tempb\endenumerate
\renewenvironment{enumerate}{\tempa\addtolength{\itemsep}{\fill}}{\tempb}

\usepackage{fontawesome5}
\setbeamertemplate{itemize item}{\faAngleRight}
\setbeamertemplate{itemize subitem}{\faAngleDoubleRight}

\setmainfont{Crimson Pro Light}[
  ItalicFont={* Italic},
  BoldFont={Crimson Pro Medium},
  BoldItalicFont={Crimson Pro Medium Italic}]
\setsansfont{Crimson Pro Light}[
  ItalicFont={* Italic},
  BoldFont={Crimson Pro Medium},
  BoldItalicFont={Crimson Pro Medium Italic}]

\usepackage[mode=tex]{standalone}
\usepackage{tikz}
\usetikzlibrary{decorations}
\usetikzlibrary{decorations.pathreplacing, intersections}
\usepackage{pgfplots}
\usetikzlibrary{calc,positioning}
\usepgfplotslibrary{fillbetween}
\pgfplotsset{compat=newest, scale only axis, width = 10cm}

% --------------------------- %
% Section title page with toc %
% --------------------------- %
\setbeamertemplate{subsection page}{%
    \usebeamertemplate*{section page}
}
\setbeamertemplate{section in toc}[square]
\setbeamertemplate{subsection in toc}[square]
\AtBeginSection[]{
% \sepframe
\begin{frame}[noframenumbering]{Outline}
    % \tableofcontents[currentsection]
    \tableofcontents[currentsection, currentsubsection]
\end{frame}
}
\AtBeginSubsection[]{
  \begin{frame}[noframenumbering]{Outline}
    \tableofcontents[currentsection, currentsubsection]
  \end{frame}
}

% ------------ %
% beamerbutton %
% ------------ %
\newcommand{\goto}[2]{\hyperlink{#2}{\beamergotobutton{#1}}}
\newcommand{\return}[2]{\hyperlink{#2}{\beamerreturnbutton{#1}}}
\newcommand{\extgoto}[2]{\href{#2}{\beamergotobutton{#1}}}

\hypersetup{
    pdfpagemode=UseNone,
    pdftitle = {Macroeconomics I, Lecture 8},
    pdfauthor = {Hui-Jun Chen},
    pdfsubject = {},
    pdfkeywords = {},
}

\title[Lecture 8]{Lecture 8 \\ Competitive Equilibrium \\ One-Period Model}
\author[Hui-Jun Chen]{Hui-Jun Chen}
\institute[NTHU]{National Tsing Hua University}
\date{\today}

\begin{document}

% Title Page
\begin{frame}[plain]
    \titlepage
\end{frame}

\begin{frame}{Overview}
\label{slide:Overview}
    After constructing both \alert{consumers'} and \alert{firms'} problem, we start to bring them together in \alert{one-period model}:
    \begin{itemize}
        \item Lecture 8: \alert{competitive equilibrium} (CE)
        \begin{itemize}
            \item each agent solve their problems individually
            \item aggregate decision determines ``prices'' (wage, rent, etc.)
        \end{itemize}
        \item Lecture 9: \alert{social planer's problem} (SPP)
        \begin{itemize}
            \item imaginary and benevolent social planner determines the allocation
            \item should be the most efficient outcome
        \end{itemize}
        \item Lecture 10: CE and SPP examples
    \end{itemize}
\end{frame}

\section{Structure}
\label{sec:Structure}

\begin{frame}{Review: Structure of Macro Model: $ 4 $ elements}
\label{slide:Structure_of_Macro_Model____4___elements}
\begin{enumerate}
    \item \textbf{agent}: who is involved?
    \begin{itemize}
        \item e.g. consumers, firms, \alert{government}
    \end{itemize}
    \item \textbf{preferences}: how and what is consumed/valued/invested?
    \begin{itemize}
        \item \alert{consumers}: monotone, convex, consumption $+$ leisure normal
        \item \alert{firms}: profit maximization
        \item \alert{government}: passive (for now)
    \end{itemize}
    \item \textbf{resources}: availability and distribution
    \begin{itemize}
        \item \alert{consumer}: $ h $ unit of time endowment
        \item \alert{firm}: production technology $ z F( K, N^{d} ) $
    \end{itemize}
    \item \textbf{technology}: objective limitation at given period of time
    \begin{itemize}
        \item CRS production function, government tax decision
    \end{itemize}
\end{enumerate}
\end{frame}

\begin{frame}{Government and Budget Balance}
\label{slide:Government_and_Budget_Balance}
\begin{itemize}
    \item Government provide $ G $ unit of gov. spending by imposing \alert{lump-sum tax} $ T $ on consumer.
    \item Assumptions:
    \begin{enumerate}
        \item Government spending requires resources but with \alert{no benefit} $ \Rightarrow $ not \alert{public goods}
        \item No transfers between consumers $ \Rightarrow  $ no heritage, no social security, etc
        \item \textbf{Government budget balance}: $ G = T $, must run balanced budget
        \begin{itemize}
            \item special case: $ G=0 $ means no government!
        \end{itemize}
    \end{enumerate}
\end{itemize}
\end{frame}

\begin{frame}{Using a Macro Model}
\label{slide:Using_a_Macro_Model}
        \textit{\scriptsize ``Making use of the model is a process of running experiments to determine how changes in the exogenous variables change the endogenous variables.'' – Williamson, p.144}
    \begin{figure}
        \includegraphics[width=\textwidth]{./figures/MacroModelStructure.jpg}
    \end{figure}
    \begin{columns}
        \begin{column}{0.5\textwidth}
            \textbf{Exogenous variables}: determined \alert{outside} the model
            \begin{enumerate}
                \item $ G $: gov. spending
                \item $ K $: firms' capital stock
                \item $ z, h $: TFP, consumer's time endowment
            \end{enumerate}
        \end{column}
        \begin{column}{0.5\textwidth}
            \textbf{Endogenous variables}: determined \alert{inside} the model
            \begin{itemize}
                \item $ C, Y$: consumption, output
                \item $ N^{s}, N^{d} $: labor supply \& demand
                \item \red{$ T, w, \pi $: tax level, wage rate, dividends}
            \end{itemize}
        \end{column}
    \end{columns}
\end{frame}

\section{Competitive Equilibrium}
\label{sec:Competitive_Equilibrium}

\begin{frame}{Concept: Competitive Equilibrium}
\label{slide:Concept__Competitive_Equilibrium}
\begin{itemize}
    \item Agents in the economy behave for a \alert{given} set of \alert{exogenous variables} and \alert{parameters}
    \item Both consumer and firm \alert{took the wage rate as given}.
    \item But this wage is \alert{endogenous}! How is this wage determined?
    \item Solution: in competitive equilibrium,
    \begin{itemize}
        \item prices are \alert{exogenous to agent} (``taken as given''), but
        \item \alert{endogenous to the model} (NOT parameter and need to be solved)
    \end{itemize}
    \item \textbf{Market clear}: wage rate is determined by $ N^{s} = N^{d} $ (``endogenous'')
    \item other examples: dividend income, taxes
\end{itemize}

\end{frame}

\begin{frame}{Analysis on Competitive Equilibrium}
\label{slide:Analysis_on_Competitive_Equilibrium}
\begin{itemize}
    \item How many markets exist in this economy?
    \begin{itemize}
        \item There are $ 2 $ goods: consumption goods and leisure
        \item While there is only $ 1 $ market: leisure is traded for consumption with wage rate $ w $
    \end{itemize}
    \item \textbf{Walras' Law}: with $ N $ goods, can only have $ N-1 $ prices
    \begin{itemize}
        \item All prices are \alert{relative prices}: \alert{normalize} price of consumption as $ 1 $ $ \Rightarrow  $ relative price of leisure is $ w $
        \item Trade consumption goods for consumption goods?
    \end{itemize}
\end{itemize}
\end{frame}

\begin{frame}{Competitive Equilibrium in Words}
\label{slide:Competitive_Equilibrium_in_Words}
    A competitive equilibrium given \textit{exogenous levels of government spending, TFP, and capital} is a set of \alert{endogenous quantities of output, consumption, labor demand, labor supply, dividends, and taxes and an endogenous wage rate} such that the following properties are satisfied:
    \begin{enumerate}
        \item the representative consumer chooses \alert{consumption and labor supply} to make herself as well off as possible subject to her budget constraint, taking as \alert{given the wage, taxes, and dividend income}
        \item the representative firm chooses \alert{labor demand} to maximize profits taking \alert{capital, TFP, and the wage as given}.
        \item output (profits) are total (net) revenues, determined ``residually''
        \item the government imposes the \alert{taxes} required by its budget constraint
        \item the \alert{labor market clears}, i.e., the quantity of labor supplied by the consumer is equal to the quantity of labor demanded by the firm.
    \end{enumerate}
\end{frame}

\begin{frame}{Competitive Equilibrium in Math}
\label{slide:Competitive_Equilibrium_in_Math}
    A \alert{competitive equilibrium} given \alert{$ \{ G, z, K \} $} is a set of allocations \red{$ \{ Y^{*}, C^{*}, l^{*}, N^{s*}, N^{d*}, \pi^{*}, T^{*} \}$} and prices $ \orange{\{ w^{*} \}} $ such that
    \begin{enumerate}
        \item Taken prices \orange{$w$} and \orange{$\pi, T$} as given, representative consumer solves
        %
        \begin{equation}
        \label{eq:consumer_problem}
            \max_{\red{C}, \red{l} \in [ 0, \alert{h} ]} U( \red{C, l} ) \quad \text{ subject to } \quad \red{C} \le \orange{w} ( \alert{h} - \red{l} ) + \orange{\pi} - \orange{T}
        \end{equation}
        %
        \item Taken \orange{$w$} as given, the representative firm solves
        %
        \begin{equation}
        \label{eq:firm_problem}
            \max_{\red{N^{d}} \ge 0} \alert{z} F( \alert{K}, \red{N^{d}} ) - \orange{w} \red{N^{d}}
        \end{equation}
        %
        \item Government set taxes to balance budget: $ \red{T^{*}} = \alert{G} $
        \item Labor market clears: $ \red{w^{*}} $ such that $ \red{N^{s*} = N^{d*}} $
    \end{enumerate}
\end{frame}

\begin{frame}{Does it All Add Up? Revisiting the Income-Expenditure Identity}
\label{slide:Does_it_All_Add_Up_}
\begin{itemize}
    \item \textbf{Expenditure approach}: $ Y = C + I + G + NX $
    \begin{itemize}
        \item one period $ \Rightarrow I = 0 $; closed economy $ \Rightarrow NX = 0 \Rightarrow  \alert{Y = C + G}$
    \end{itemize}
    \item \textbf{Income approach}:
    \begin{itemize}
        \item \alert{consumer budget constraint}: $ C = w N^{s} + \pi - T $
        \item \alert{government budget balance}: $ G = T \Rightarrow C = w N^{s} + \pi - G $
        \item \alert{profit}: $ \pi = z F( K, N^{d} ) - w N^{d} = Y - w N^{d} \Rightarrow C = w N^{s} + Y - wN^{d} - G $
        \item \alert{labor market clear}: $ N^{s} = N^{d} \Rightarrow C = Y - G$
    \end{itemize}
    \item \textbf{Income-Expenditure Identity holds!}
\end{itemize}
\end{frame}

\begin{frame}{Example}
\label{slide:Example}
    Assume
    \begin{enumerate}
        \item no government: $ G = T = 0 $
        \item utility function: $ U( C, l ) = \ln C + \ln l $
        \item production function: $ F( K, N ) = K^{\alpha}N^{1-\alpha} $, where $ \alpha = \frac{1}{2} $
        \item $ z = K = 1 $; $ h = 1 $
    \end{enumerate}
    Consumer: $ \max_{C, l} \ln C + \ln l \quad \text{subject to} \quad C \le w( h-l ) + \pi $
    %
    \begin{align}
        \text{FOC} \quad
            & \frac{C}{l} = w
            \label{eq:consumerFOC}
        \\
        \text{Binding budget constraint} \quad
            & C = w ( 1-l ) + \pi
            \label{eq:binding_budget}
        \\
        \text{Time constraint} \quad
            & N^{s} = 1 - l
            \label{eq:time_budget}
    \end{align}
    %
\end{frame}

\begin{frame}{Example (Cont.)}
\label{slide:Example__Cont__}
    Firm: $ \max_{N^{d}} ( N^{d} )^{\frac{1}{2}} - w N^{d} $
    %
    \begin{align}
        \text{FOC} \quad
            & \frac{1}{2} ( N^{d} )^{- \frac{1}{2}} = w
            \label{eq:firmFOC}
        \\
        \text{Output definition} \quad
            & Y = ( N^{d} )^{\frac{1}{2}}
            \label{eq:outputDef}
        \\
        \text{Profit definition} \quad
            & \pi = Y - w N^{d}
            \label{eq:profitDef}
    \end{align}
    %
    Market clear:
    %
    \begin{align}
        N^{s} & = N^{d}
        \label{eq:laborClear}
    \end{align}
    %
$ 7 $ equations (\eqref{eq:consumerFOC}-\eqref{eq:laborClear}), $ 7  $ unknowns ($C, l, N^{s}, N^{d}, Y, \pi, w$), can solve entirely!
\end{frame}


\end{document}

