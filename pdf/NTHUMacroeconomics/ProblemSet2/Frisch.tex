\documentclass[14pt]{extarticle}
\usepackage{natbib}

\usepackage{xcolor}
\usepackage{amsmath}
\newcommand{\tuple}[1]{ \langle #1 \rangle }
%\usepackage{automata}
\usepackage{times}
\usepackage{ltablex}
\usepackage{tasks}
\usepackage{threeparttable}
\usepackage{booktabs}
\usepackage{xcolor} % For custom colors
\usepackage{tikz} % For styling enumerate numbers
\usepackage{tcolorbox} % For colored box styling
\usepackage{amsmath, amsfonts, amssymb, amsthm} % Math related
\usepackage{natbib}
\usepackage{fontspec}
\usepackage{luatexja}
\usepackage[mathscr]{euscript}

%%%%%% Template
\usepackage{hyperref}
\hypersetup{colorlinks=true,allcolors=blue}

\usepackage{vmargin}
\setpapersize{USletter}
\setmarginsrb{1.0in}{1.0in}{1.0in}{0.6in}{0pt}{0pt}{0pt}{0.4in}

% HOW TO USE THE ABOVE:
%\setmarginsrb{leftmargin}{topmargin}{rightmargin}{bottommargin}{headheight}{headsep}{footheight}{footskip}
%\raggedbottom
% paragraphs indent & skip:
\parindent  0.3cm
\parskip    -0.01cm

\usepackage{tikz}
\usetikzlibrary{backgrounds}

% hyphenation:
% \hyphenpenalty=10000 % no hyphen
% \exhyphenpenalty=10000 % no hyphen
\sloppy

% notes-style paragraph spacing and indentation:
\usepackage{parskip}
\setlength{\parindent}{0cm}

% let derivations break across pages
\allowdisplaybreaks

\newcommand{\orange}[1]{\textcolor{orange}{#1}}
\newcommand{\blue}[1]{\textcolor{blue}{#1}}
\newcommand{\red}[1]{\textcolor{red}{#1}}
\newcommand{\freq}[1]{{\bf \sf F}(#1)}
\newcommand{\datafreq}[2]{{{\bf \sf F}_{#1}(#2)}}

\def\qqquad{\quad\qquad}
\def\qqqquad{\qquad\qquad}

%%%%%%%%%%%%%%%%%%%%%%%%%%%%%%%%%%%%%%%%%%%%%%%%%%%%%%%%%%%%%%%%%%%%%%%%%%%%%%%%
%%%%%%%%%%%%%%%%%%%%%%%%%%%%%%%%%%%%%%%%%%%%%%%%%%%%%%%%%%%%%%%%%%%%%%%%%%%%%%%%

% fill-in-blank question style, found in https://tex.stackexchange.com/a/505089

\usepackage{ifthen}
\usepackage{tocloft}
\usepackage{exercise}
% \usepackage{xcolor}

% Set the Show Answers Boolean
\newboolean{showAns}
\setboolean{showAns}{false} %<-- DEFAULT: hides answers
\newcommand{\showAns}{\setboolean{showAns}{true}} %<-- COMMAND: call \showAns to show answers

% The length of the Answer line
\newlength{\answerlength}
\newcommand{\anslen}[1]{\settowidth{\answerlength}{#1}}

% ans command that indicates space for an answer or shows the answer in red
\newcommand{\ans}[1]{\settowidth{\answerlength}{\hspace{2ex}#1\hspace{2ex}}%
    \ifthenelse{\boolean{showAns}}%
        {\textcolor{red}{\underline{\hspace{2ex}#1\hspace{2ex}}}}%
        {\underline{\hspace{\answerlength}}}}%

\newcommand{\details}[1]{\settowidth{\answerlength}{#1}%
    \ifthenelse{\boolean{showAns}}%
        {\begin{tcolorbox}[colback=blue!5!white,colframe=blue!75!black,title=Solution] #1 \end{tcolorbox}}%
        {}}%

% Formatting how multiple choices Questions are formated.
\settasks{label=(\Alph*), label-width=30pt}


% Some commands for the Exercise Question package
\renewcommand{\QuestionNB}{\Large\protect\textcircled{\small\bfseries\arabic{Question}}\ }
\renewcommand{\ExerciseHeader}{} %no header
\renewcommand{\QuestionBefore}{3ex} %Space above each Q
\setlength{\QuestionIndent}{8pt} % Indent after Q number


% To create the list of answers with tocloft...
\newcommand{\listanswername}{Answers}
\newlistof[Question]{answer}{Answers}{\listanswername}

% Creates a TOC for Answers
\newcounter{prevQ}
\newcommand{\answer}[1]{\refstepcounter{answer}%
\ans{#1}%
\ifnum\theQuestion=\theprevQ%
        \addcontentsline{Answers}{answer}{\protect\numberline{}#1}% don't include the Q number
        \else%
        \addcontentsline{Answers}{answer}{\protect\numberline{\theQuestion}#1}%
        \setcounter{prevQ}{\value{Question}}%
        \fi%
        }%

% \hyphenpenalty=10000 % no hyphen
% \exhyphenpenalty=10000 % no hyphen
\sloppy         % hyphen

\newcommand{\HRule}{\rule{\linewidth}{0.5mm}}
\newcommand{\Hrule}{\rule{\linewidth}{0.3mm}}

%tocloft formatting listofanswers
\renewcommand{\cftAnswerstitlefont}{\bfseries\large}
\renewcommand{\cftanswerdotsep}{\cftnodots}
\cftpagenumbersoff{answer}
\addtolength{\cftanswernumwidth}{10pt}

\makeatletter% since there's an at-sign (@) in the command name
\renewcommand{\@maketitle}{%
  \parindent=0pt% don't indent paragraphs in the title block
  \centering
  {\Large \bfseries\textsc{\@title}} \\
  \vspace{5pt}
  {\large \textit{\@author}} \\
  \HRule \\
  \vspace{1em}
}
\makeatother% resets the meaning of the at-sign (@)

\setmainfont{Crimson Pro Light}[
  ItalicFont={* Italic},
  BoldFont={Crimson Pro Medium},
  BoldItalicFont={Crimson Pro Medium Italic}]
\setsansfont{Crimson Pro Light}[
  ItalicFont={* Italic},
  BoldFont={Crimson Pro Medium},
  BoldItalicFont={Crimson Pro Medium Italic}]

\title{Problem Set 2}
\author{Hui-Jun Chen}

\begin{document}

\maketitle

% UNCOMMENT THE LINE BELOW TO SHOW ANSWERS AND SOLUTIONS
\showAns
% \listofanswer

% \section*{Instruction}
% Please answer this problem set on Carmen quizzes ``Problem Set 2''.
% In the following problems, the part that is in \textbf{\red{red and bold}} are the order of questions that should be answered on Carmen quizzes.

\begin{Exercise}

\section*{Problem 3: Macroeconomic Analysis of the 2025 Taiwan Cash Handout}

\subsection*{Applied Context: The Policy}
In late 2025, the Taiwanese government finalized a plan to distribute a universal cash handout of NTD 10,000 to all eligible citizens and residents. This policy, officially part of the "Special Act for Strengthening Economic and Social Resilience," was designed to share the benefits of strong economic performance and provide relief from global inflationary pressures.

As a junior economist at a local think tank in Hsinchu, you are tasked with analyzing the potential impact of this "helicopter money" on household labor supply. You decide to apply a standard micro-founded macroeconomic model.

\textbf{Important Note}: This is a \textbf{partial equilibrium} analysis focusing only on the consumer's decision. It assumes the wage is fixed and does not solve for a general equilibrium where the labor market clears.

\subsection*{The Model}
\begin{itemize}
    \item \textbf{Utility Function}: $U(C, l) = \frac{C^{1-\sigma}}{1-\sigma} + b \cdot l$
    \item \textbf{Budget Constraint}: $C \le w(h - l) + \pi - T$
    \item \textbf{Parameters}: $\sigma = 2$, $b = 5$, $w = 10$, $h = 1$
\end{itemize}

\HRule

\subsection*{Part I: Theoretical Foundations}

\Question The \textbf{Frisch elasticity of labor supply ($\epsilon_{N,w}^{Frisch}$)} measures the percentage change in hours worked ($N$) in response to a one percent change in the real wage ($w$), holding the marginal utility of wealth ($\lambda$) constant. Mathematically, it is defined as:
\[ \epsilon_{N,w}^{Frisch} = \frac{\partial N / N}{\partial w / w} \bigg|_{\lambda \text{ constant}} \]
Given the utility function in this model, what is the Frisch elasticity of labor supply? \answer{D}
\begin{tasks}(1)
    \task 0
    \task 1
    \task -1
    \task Infinite ($\infty$)
\end{tasks}
\details{The Frisch elasticity holds $\lambda$ constant. The optimality condition is $b = \lambda \cdot w$. Since $b$ and $\lambda$ are both treated as constants in this experiment, if $w$ changes even slightly, one side becomes permanently greater than the other, causing an "all-or-nothing" jump in labor supply from 0 to max hours, which is an infinite response.}

\Question The MRS represents the rate at which a consumer is willing to trade consumption for leisure while maintaining the same level of utility. What is the correct formula for the $MRS_{l,C}$ for this model? \answer{A}
\begin{tasks}(1)
    \task $bC^{\sigma}$
    \task $C^{\sigma}/b$
    \task $b/C$
    \task $w/b$
\end{tasks}
\details{The MRS is the ratio of the marginal utility of leisure ($MU_l = b$) to the marginal utility of consumption ($MU_C = C^{-\sigma}$). Therefore, $MRS_{l,C} = \frac{MU_l}{MU_C} = \frac{b}{C^{-\sigma}} = bC^{\sigma}$.}

\subsection*{Part II: The Economy Before the Handout}
For this section, assume the baseline scenario where $\pi - T = 0$.

\Question What is the consumer's optimal consumption ($C_0$) in this pre-policy environment? \answer{C}
\begin{tasks}(1)
    \task 2
    \task 10
    \task $\sqrt{2}$
    \task 5
\end{tasks}
\details{The optimal interior choice occurs where $MRS_{l,C} = w$, so $bC^{\sigma} = w$. Plugging in the parameters: $5 \cdot C_0^2 = 10 \implies C_0^2 = 2 \implies C_0 = \sqrt{2} \approx 1.414$.}

\Question What is the consumer's optimal choice of leisure ($l_0$)? \answer{A}
\begin{tasks}(1)
    \task $1 - \frac{\sqrt{2}}{10}$
    \task 1
    \task 0.5
    \task 0
\end{tasks}
\details{Substitute $C_0 = \sqrt{2}$ into the budget constraint: $\sqrt{2} = 10(1 - l_0) + 0 \implies 10l_0 = 10 - \sqrt{2} \implies l_0 = 1 - \frac{\sqrt{2}}{10} \approx 0.8586$.}

\Question Given the choice of leisure in the previous question, how many hours does the consumer choose to work ($N_0 = h - l_0$)? \answer{B}
\begin{tasks}(1)
    \task $1 - \frac{\sqrt{10}}{2}$
    \task $\frac{\sqrt{2}}{10}$
    \task 0.5
    \task 1
\end{tasks}
\details{Labor supply is total time minus leisure: $N_0 = h - l_0 = 1 - (1 - \frac{\sqrt{2}}{10}) = \frac{\sqrt{2}}{10} \approx 0.1414$.}


\subsection*{Part III: The Economy After the Handout}
The NTD 10,000 handout is now distributed. We normalize this amount so the post-policy value is $\pi - T = 1$.

\Question What is the consumer's new optimal consumption ($C_1$)? \answer{D}
\begin{tasks}(1)
    \task $1 + \sqrt{2}$
    \task 11
    \task 1
    \task $\sqrt{2}$
\end{tasks}
\details{For an interior solution, optimal consumption depends only on the wage and preferences ($C = (w/b)^{1/\sigma}$), not non-wage income. Since the wage is constant and the cash handout is not large enough to push the consumer to a corner solution, their target consumption level remains the same.}

\Question What is the consumer's new optimal leisure ($l_1$)? \answer{B}
\begin{tasks}(1)
    \task It remains unchanged.
    \task $1 - \frac{\sqrt{2}-1}{10}$
    \task 1
    \task $1 - \frac{\sqrt{2}}{10}$
\end{tasks}
\details{Plug the new parameters into the budget constraint: $\sqrt{2} = 10(1 - l_1) + 1 \implies \sqrt{2} - 1 = 10 - 10l_1 \implies 10l_1 = 11 - \sqrt{2} \implies l_1 = \frac{11 - \sqrt{2}}{10} \approx 0.9586$.}

\Question What is the new number of hours worked ($N_1 = h - l_1$)? \answer{A}
\begin{tasks}(1)
    \task $\frac{\sqrt{2}-1}{10}$
    \task $\frac{\sqrt{2}}{10}$
    \task 0
    \task 0.1
\end{tasks}
\details{$N_1 = h - l_1 = 1 - (\frac{11 - \sqrt{2}}{10}) = \frac{10 - 11 + \sqrt{2}}{10} = \frac{\sqrt{2}-1}{10} \approx 0.0414$.}

\Question What is the approximate percentage change in hours worked as a result of the cash handout? \answer{C}
\begin{tasks}(1)
    \task -29.3\%
    \task +70.7\%
    \task -70.7\%
    \task +29.3\%
\end{tasks}
\details{The percentage change is $\frac{N_1 - N_0}{N_0} = \frac{(\sqrt{2}-1)/10 - \sqrt{2}/10}{\sqrt{2}/10} = \frac{\sqrt{2}-1-\sqrt{2}}{\sqrt{2}} = \frac{-1}{\sqrt{2}} \approx -0.707$, or -70.7\%.}

\subsection*{Part IV: Analysis and Interpretation}

\Question The cash handout represents a pure... \answer{C}
\begin{tasks}(1)
    \task Substitution Effect
    \task Technology Shock
    \task Income Effect
    \task Preference Shock
\end{tasks}
\details{A cash transfer that does not depend on hours worked is a pure increase in non-wage income. This shifts the budget constraint outward without changing its slope (the wage), which is the definition of a pure income effect.}

\Question The model's prediction that leisure increases when non-wage income rises implies that leisure is what kind of good? \answer{D}
\begin{tasks}(1)
    \task An inferior good
    \task A Giffen good
    \task A public good
    \task A normal good
\end{tasks}
\details{A normal good is any good for which demand increases when income increases. Since the consumer "buys" more leisure (i.e., works less) when their non-wage income rises, leisure is a normal good.}

\Question This partial equilibrium analysis focuses only on the household labor supply decision. If this were the only effect, what would the model predict is the immediate impact of the cash handout on GDP? \answer{B}
\begin{tasks}(1)
    \task GDP would increase, because consumption rises.
    \task GDP would decrease, because total hours worked in the economy fall.
    \task GDP would remain unchanged, because the money is just a transfer.
    \task The impact cannot be determined.
\end{tasks}
\details{In this simple model, GDP is a function of total labor input. Since the policy causes the representative consumer to work less ($N_1 < N_0$), the total hours worked in the economy would fall, leading to a decrease in output (GDP).}

\subsection*{Part V: Reconciling Theory and Application}

\Question Question 1 established that the Frisch elasticity is infinite, implying an "all-or-nothing" labor response to wage changes. Yet, your answer to Question 5 shows an \textbf{interior solution} for labor supply. Why does the model predict the average person works *some* of the time? \answer{C}
\begin{tasks}(1)
    \task The income effect from the wage is large enough to offset the infinite substitution effect.
    \task The Frisch elasticity is only a theoretical concept and does not apply to this model.
    \task In the full optimization, the consumer adjusts their consumption, which in turn adjusts their marginal utility of wealth, allowing them to find an interior balance.
    \task The specific parameter values ($b=5, w=10$) are uniquely chosen to create an interior solution.
\end{tasks}
\details{This is the key conceptual point. The infinite Frisch elasticity comes from an artificial experiment where the marginal utility of wealth ($\lambda$) is held constant. In the full optimization, the consumer is free to adjust their consumption level ($C$). By choosing $C=\sqrt{2}$, they implicitly choose a marginal utility of wealth ($\lambda = C^{-\sigma} = (\sqrt{2})^{-2} = 0.5$) that makes them perfectly willing to work for the going wage, leading to an interior solution.}

\Question At the benchmark optimum, what is the marginal utility of an additional dollar of income ($\lambda = MU_C$)? \answer{A}
\begin{tasks}(1)
    \task 0.5
    \task 2
    \task 1
    \task 10
\end{tasks}
\details{The marginal utility of wealth ($\lambda$) is equal to the marginal utility of consumption ($MU_C$) at the optimum. $\lambda = MU_C = C_0^{-\sigma} = (\sqrt{2})^{-2} = 1/2 = 0.5$.}

\Question Which of the following is a significant limitation of applying this simple model to the real-world policy? \answer{B}
\begin{tasks}(1)
    \task The model incorrectly assumes people like leisure.
    \task The model uses a fixed wage, ignoring that a widespread drop in labor supply could cause wages to rise, creating a \textbf{general equilibrium effect}.
    \task The model's utility function is too complex to be realistic.
    \task The model does not account for the government's budget.
\end{tasks}
\details{This is a core concept of general vs. partial equilibrium. The model's assumption of a fixed wage is its biggest weakness. In reality, if a large portion of the population reduces their labor supply, firms would likely have to increase wages to attract workers, creating a general equilibrium effect that would counteract some of the initial decline in work.}

\end{Exercise}

\end{document}
