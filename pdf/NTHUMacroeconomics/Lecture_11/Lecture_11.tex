\documentclass[11pt,aspectratio=169,usenames,dvipsnames]{beamer}
\usetheme{SimplePlus}

\usepackage{threeparttable}
\usepackage{booktabs}
\usepackage{xcolor} % For custom colors
\usepackage{tikz} % For styling enumerate numbers
\usepackage{tcolorbox} % For colored box styling
\usepackage{amsmath, amsfonts, amssymb, amsthm} % Math related
\usepackage{natbib}
\usepackage{fontspec}
\usepackage{luatexja}
\usepackage[mathscr]{euscript}

% ---------------- %
% color definition %
% ---------------- %
\definecolor{main}{HTML}{23373B}
\definecolor{pink}{RGB}{180, 50, 110}
\definecolor{orange}{HTML}{FF8000}
\definecolor{red}{HTML}{990000}
\definecolor{blue}{HTML}{004C99}
\definecolor{lightgray}{HTML}{E7E7E7}
\definecolor{gray}{RGB}{90, 90, 90}

\newcommand{\pink}[1]{\textcolor{pink}{#1}}
\newcommand{\orange}[1]{\textcolor{orange}{#1}}
\newcommand{\red}[1]{\textcolor{red}{#1}}
\newcommand{\blue}[1]{\textcolor{blue}{#1}}
\newcommand{\green}[1]{\textcolor{OliveGreen}{#1}}
\newcommand{\magenta}[1]{\textcolor{magenta}{#1}}
\newcommand{\gray}[1]{\textcolor{gray}{#1}}
\newcommand{\purple}[1]{\textcolor{purple}{#1}}
\definecolor{yellow}{HTML}{EDB120}

% \setbeamercolor{alerted text}{fg=blue}

%%% automatically add spaces into enumerate and itemize environment
\let\tempone\itemize
\let\temptwo\enditemize
\renewenvironment{itemize}{\tempone\addtolength{\itemsep}{\fill}}{\temptwo}
\let\tempa\enumerate
\let\tempb\endenumerate
\renewenvironment{enumerate}{\tempa\addtolength{\itemsep}{\fill}}{\tempb}

\usepackage{fontawesome5}
\setbeamertemplate{itemize item}{\faAngleRight}
\setbeamertemplate{itemize subitem}{\faAngleDoubleRight}

\setmainfont{Crimson Pro Light}[
  ItalicFont={* Italic},
  BoldFont={Crimson Pro Medium},
  BoldItalicFont={Crimson Pro Medium Italic}]
\setsansfont{Crimson Pro Light}[
  ItalicFont={* Italic},
  BoldFont={Crimson Pro Medium},
  BoldItalicFont={Crimson Pro Medium Italic}]

\usepackage[mode=tex]{standalone}
\usepackage{tikz}
\usetikzlibrary{decorations}
\usetikzlibrary{decorations.pathreplacing, intersections}
\usepackage{pgfplots}
\usetikzlibrary{calc,positioning}
\usepgfplotslibrary{fillbetween}
\pgfplotsset{compat=newest, scale only axis, width = 10cm}

% --------------------------- %
% Section title page with toc %
% --------------------------- %
\setbeamertemplate{subsection page}{%
    \usebeamertemplate*{section page}
}
\setbeamertemplate{section in toc}[square]
\setbeamertemplate{subsection in toc}[square]
\AtBeginSection[]{
% \sepframe
\begin{frame}[noframenumbering]{Outline}
    % \tableofcontents[currentsection]
    \tableofcontents[currentsection, currentsubsection]
\end{frame}
}
\AtBeginSubsection[]{
  \begin{frame}[noframenumbering]{Outline}
    \tableofcontents[currentsection, currentsubsection]
  \end{frame}
}

% ------------ %
% beamerbutton %
% ------------ %
\newcommand{\goto}[2]{\hyperlink{#2}{\beamergotobutton{#1}}}
\newcommand{\return}[2]{\hyperlink{#2}{\beamerreturnbutton{#1}}}
\newcommand{\extgoto}[2]{\href{#2}{\beamergotobutton{#1}}}

\hypersetup{
    pdfpagemode=UseNone,
    pdftitle = {Macroeconomics I, Lecture 11},
    pdfauthor = {Hui-Jun Chen},
    pdfsubject = {},
    pdfkeywords = {},
}

\title[Lecture 9]{Lecture 11 \\ Distorting Taxes and the Welfare Theorems}
\author[Hui-Jun Chen]{Hui-Jun Chen}
\institute[NTHU]{National Tsing Hua University}
\date{\today}

\begin{document}

% Title Page
\begin{frame}[plain]
    \titlepage
\end{frame}


\begin{frame}{Overview}
\label{slide:Overview}

In previous lectures, all the taxes we are discussing is \alert{lump-sum tax}.

\begin{itemize}
    \item pure \alert{income effect}, no change to consumption-leisure allocation
    \item satisfy both welfare theorems
\end{itemize}

In this lecture, the \alert{distorting taxes} will include \alert{substitution effect}, and thus

\begin{itemize}
    \item creating ``wedges'' to distort consumption-leisure choice
    \item violate the welfare theorems (CE $ \neq $ SPP)
\end{itemize}

\end{frame}

\section{Simplified (but Problematic) Model}
\label{sec:Simplified_Model}

\begin{frame}{SPP in Simplified (but Problematic) Model}
\label{slide:SPP_in_Simplified_Model}
    \begin{columns}
        \begin{column}{0.5\textwidth}
            \begin{figure}
                \includegraphics[width=.7\textwidth]{./figures/Figure_5_14.jpg}
            \end{figure}
        \end{column}
        \begin{column}{0.5\textwidth}
            Assume production is labor-only technology:

            %
            \begin{equation*}
               Y = z N^{d}
            \end{equation*}
            %
            So PPF is
            %
            %
            \begin{equation*}
               C = z( h-l ) - G
            \end{equation*}
            %
            Thus, SPP is
            %
            \begin{align*}
                    & \max_{l} U( z( h-l ) - G, l )
                \\
                \text{FOC:} \quad
                    & \frac{D_{l}U( C, l )}{D_{C}U( C, l )} = MRS_{l, C}
                \\
                    & = MRT_{l, C} = z = MPN
            \end{align*}
            %
        \end{column}
    \end{columns}
\end{frame}

\begin{frame}{Labor Demand in Simplified Model}
\label{slide:Labor_Demand_in_Simplified_Model}
    \begin{columns}
        \begin{column}{0.5\textwidth}
            \begin{figure}
                \caption{\scriptsize Figure 5.15  The Labor Demand Curve in the Simplified Model}
                \includegraphics[width=\textwidth]{./figures/Figure_5_15.jpg}
            \end{figure}
        \end{column}
        \begin{column}{0.5\textwidth}
            %
            \begin{equation*}
                 \max_{N^{d}} z N^{d} - wN^{d}
            \end{equation*}
            %
            FOC would be $ z = w $ (horizontal line)
            \begin{itemize}
                \item if $ z < w $: negative profit for every worker hired, choose $ N^{d} = 0 $
                \item if $ z > w $: positive profit for every worker hired, choose $ N^{d} = \infty $
                \item only $ z = w $ possible, $ \therefore $ linear PPF in previous slide
                \begin{itemize}
                    \item ``infinitely elastic'' $ N^{d} $
                \end{itemize}
            \end{itemize}
        \end{column}
    \end{columns}
\end{frame}

\begin{frame}{Competitive Equilibrium w/ Distorting Tax}
\label{slide:Competitive_Equilibrium_w__Distortionary_Tax}
    A competitive equilibrium, with $ \{ z, \alert{G} \} $ exogenous, is a list of endogenous prices and quantities $ \{ C, l, N^{s}, N^{d}, Y, \pi, w, \alert{t} \} $ such that:
    \begin{enumerate}
        \item  taking $ \{ w, \pi \} $ as given, the consumer solves
        %
        %
        %
        \begin{equation*}
            \max_{C, l, N^{s}} U( C, l )
            \quad \text{subject to} \quad
            C = w\alert{( 1-t )}N^{s} + \pi
            \quad \text{and} \quad
            N^{s} + l = h
        \end{equation*}
        %
        \item taking $ w $ as given, the firm solves:
        %
        \begin{equation*}
            \max_{N^{d}, Y, \pi} \pi
            \quad \text{subject to} \quad
            \pi = Y - w N^{d}
            \quad \text{and} \quad
            Y = z N^{d}
        \end{equation*}
        %
        \item the government spends $ G = w t N^{s} $
        \item the labor market clears at the equilibrium wage, i.e. $ N^{s} = N^{d} $
    \end{enumerate}
\end{frame}

\begin{frame}{Effect of Distorting Tax}
\label{slide:Effect_of_Distorting_Tax}
    Since the tax is imposed on consumers/workers, it distorted the consumption-leisure decision:
    %
    \begin{equation*}
        MRS_{l, C} = w( 1-t )
    \end{equation*}
    %
    So in the equilibrium, it deviates from SPP:
    %
    \begin{equation*}
        MRS_{l, C} = w( 1-t ) < w = z = MPN = MRT_{l, C}
    \end{equation*}
    %
    \textbf{Result}: CE and SPP lead to different allocation!
\end{frame}

\begin{frame}{Graphical Representation}
\label{slide:Graphical_Representation}
    \begin{columns}
        \begin{column}{0.5\textwidth}
            \begin{figure}
                \caption{\scriptsize Figure 5.16  Competitive Equilibrium in the Simplified Model with a Proportional Tax on Labor Income}
                \includegraphics[width=.7\textwidth]{./figures/Figure_5_16.jpg}
            \end{figure}
        \end{column}
        \begin{column}{0.5\textwidth}
            SPP solution lies at point E:
            \begin{itemize}
                \item $\overline{AB}$: PPF, slope  $ -z $
                \item can reach indifference curve $ I_{1} $
            \end{itemize}
            CE solution lies at point H:
            \begin{itemize}
                \item $\overline{DF}$: consumer’s budget line
                \item can only reach $ I_{2} $
                \item proportional tax $ \Rightarrow  $ $ N^{s} $ $ \downarrow  $
                \item $ N^{s} $$ \downarrow $$ \Rightarrow $$ Y $$\downarrow  $, but still need to meet $ G $, so $ C \downarrow  $: gov’t budget critical!
            \end{itemize}
        \end{column}
    \end{columns}
\end{frame}

\begin{frame}{How Much Tax Revenue can be Generated?}
\label{slide:How_Much_Tax_Revenue_can_be_Generated_}
    \begin{columns}
        \begin{column}{0.5\textwidth}
            \begin{figure}
                \caption{\scriptsize Figure 5.17  A Laffer Curve}
                \includegraphics[width=\textwidth]{./figures/Figure_5_17.jpg}
            \end{figure}
            \goto{Back}{slide:Multiple_Competitive_Equilibria_Possible}
        \end{column}
        \begin{column}{0.5\textwidth}
            \small
            equilibrium wage: $ w = z $, implies \alert{total tax revenue} by solve consumer problem:
            %
            \begin{equation*}
                R( t ) = tz( h - l^{*}( t ) )
            ,\end{equation*}
            %
            What $ t $ maximizes? Solve
            %
            \begin{equation*}
               \max_{t} R( t ) = \max_{t} tz( h - l^{*}( t ) )
            ,\end{equation*}
            %
            \begin{itemize}
                \item not just $ t = 1 $! tax \alert{rate} vs tax \alert{base}
                \item $ t = 0 $: no revenue because no tax
                \item $ t = 1 $: no revenue because no incentive to work
            \end{itemize}
        \end{column}
    \end{columns}
\end{frame}

\begin{frame}{Full Model Elaboration}
\label{slide:Full_Model_Elaboration}
    Let $ U( C, l ) = \ln C + \ln l $, and $ h = z = 1 $, by firm's problem we know $ w = z = 1 $.
    Consumer has some non-labor income denoted as $ x > 0 $. FOC leads to
    %
    \begin{align*}
        MRS_{l, C}
            & = \frac{C}{l}
        \\
            & = \frac{( 1-t )( 1-l ) + \pi}{l} = 1 - t < 1 = MRT_{l, C}
        \\
            & \Rightarrow ( 1-t )( 1-l ) + \pi = ( 1-t )l
        \\
            & \Rightarrow 1-l + \frac{\pi}{1-t} = l \Rightarrow 2 l = 1 + \frac{\pi}{(1-t)}
        \\
            & \Rightarrow l = \frac{1}{2} + \frac{\pi}{2(1-t)}
        \\
            & \red{ \Rightarrow N^{s} ( t ) = 1-l = \frac{1}{2} - \frac{\pi}{2(1-t)} }
    \end{align*}
    %
\end{frame}

\begin{frame}{Maximize Tax Revenue}
\label{slide:Maximize_Tax_Revenue}
    Total tax revenue is
    %
    \begin{equation*}
       R( t ) = t N^{s}( t )
    ,\end{equation*}
    %
    and thus government's problem is
    %
    \begin{equation*}
       \max_{t} \frac{1}{2}t - \frac{t\pi}{2(1-t)}
    .\end{equation*}
    %
    FOC leads to
    %
    \begin{align*}
        \frac{1}{2} - \frac{\pi(1-t) + t\pi}{2(1-t)^{2}}
            & = 0 \Rightarrow \frac{1}{2} - \frac{\pi}{2(1-t)^{2}} = 0
        \\
        \frac{1}{2}
            & = \frac{\pi}{2(1-t)^{2}} \Rightarrow 1 = \frac{\pi}{(1-t)^{2}}
        \\
        t
            & = 1 - \sqrt{\pi}
    \end{align*}
    %
\end{frame}

\begin{frame}{Visualization}
\label{slide:Visualization}
    \begin{columns}
        \begin{column}{0.5\textwidth}
            \begin{figure}
                \includegraphics[width=\textwidth]{./figures/lafferCurve.png}
            \end{figure}


        \end{column}
        \begin{column}{0.5\textwidth}
            Consider two cases:
            \begin{enumerate}
                \item consumer is poor (low $ \pi $)
                \item consumer is rich (high $ \pi $)
            \end{enumerate}
            For a given after tax-wage , rich consumer supplies less labor
            \begin{itemize}
                \item tax revenue shifts down
                \item Laffer peak shifts left
                \item many other conditions also impact this analysis!
            \end{itemize}
        \end{column}
    \end{columns}
\end{frame}

\begin{frame}{Multiple Competitive Equilibria Possible}
\label{slide:Multiple_Competitive_Equilibria_Possible}
    \begin{columns}
        \begin{column}{0.5\textwidth}
            \begin{figure}
                \caption{Figure 5.18  Two Competitive Equilibria}
                \includegraphics[width=.7\textwidth]{./figures/Figure_5_18.jpg}
            \end{figure}
        \end{column}
        \begin{column}{0.5\textwidth}
            Previous slide logic implies the government can choose 2 tax rates for a given required level of $ G $
            \begin{itemize}
                \item both $ t_{1} $ and $ t_{2} $ yield the same revenue
                \item consumer strictly better off under lower tax rate $ t_{1} $
            \end{itemize}
            % \jump{Tax Revenue}{slide:How_Much_Tax_Revenue_can_be_Generated_}
        \end{column}
    \end{columns}
\end{frame}


\begin{frame}{What's wrong with this model?}
\label{slide:What_s_wrong_with_this_model_}

Recall that $ Y = z N^{d} $ implies labor supply $ N^{s}(t) $ equals to

%
\begin{equation}
\label{eq:laborSupplyTaxRate}
    N^{s}(t) = 1 - l = \frac{1}{2} - \frac{\pi}{2(1-t)}
,\end{equation}
%
and the total tax revenue is given by
%
\begin{equation}
\label{eq:govTaxRevenue}
    R(t) = w t N^{s}(t)
.\end{equation}
%
In equilibrium $ w = z = 1 $, so $ \pi = z N^{d} - w N^{d} = 0 $, so this question is trivial...Stay tuned with Problem Sets \faSmileWink[regular]

\end{frame}

\begin{frame}{Conclusion}
\label{slide:Conclusion}
    We've focused on the simple case to keep analysis straightforward, but logic applies more broadly.
    \begin{itemize}
        \item SPP: $ MRS_{l, C} = MRT_{l, C} = MPN $, since PPF is $ C = z F( K, N ) - G $
        \item CE: same distortion as our simple case:
        \begin{itemize}
            \item consumer problem implies $ MRS_{l, C} = w ( 1-t ) $
            \item firm problem implies $ MRT_{l, C} = w $
            \item same result as simplified model: $ MRS_{l, C} \neq MRT_{l, C} $, unlike SPP
            \item only difference from simplified model: $ MPN = D_{N}F( K, N ) \neq z $
        \end{itemize}
    \end{itemize}
\end{frame}

\end{document}
