\documentclass[14pt]{extarticle}
\usepackage{natbib}

\usepackage{xcolor}
\usepackage{amsmath}
\newcommand{\tuple}[1]{ \langle #1 \rangle }
%\usepackage{automata}
\usepackage{times}
\usepackage{ltablex}
\usepackage{tasks}
\usepackage{threeparttable}
\usepackage{booktabs}
\usepackage{xcolor} % For custom colors
\usepackage{tikz} % For styling enumerate numbers
\usepackage{tcolorbox} % For colored box styling
\usepackage{amsmath, amsfonts, amssymb, amsthm} % Math related
\usepackage{natbib}
\usepackage{fontspec}
\usepackage{luatexja}
\usepackage[mathscr]{euscript}

%%%%%% Template
\usepackage{hyperref}
\hypersetup{colorlinks=true,allcolors=blue}

\usepackage{vmargin}
\setpapersize{USletter}
\setmarginsrb{1.0in}{1.0in}{1.0in}{0.6in}{0pt}{0pt}{0pt}{0.4in}

% HOW TO USE THE ABOVE:
%\setmarginsrb{leftmargin}{topmargin}{rightmargin}{bottommargin}{headheight}{headsep}{footheight}{footskip}
%\raggedbottom
% paragraphs indent & skip:
\parindent  0.3cm
\parskip    -0.01cm

\usepackage{tikz}
\usetikzlibrary{backgrounds}

% hyphenation:
% \hyphenpenalty=10000 % no hyphen
% \exhyphenpenalty=10000 % no hyphen
\sloppy

% notes-style paragraph spacing and indentation:
\usepackage{parskip}
\setlength{\parindent}{0cm}

% let derivations break across pages
\allowdisplaybreaks

\newcommand{\orange}[1]{\textcolor{orange}{#1}}
\newcommand{\blue}[1]{\textcolor{blue}{#1}}
\newcommand{\red}[1]{\textcolor{red}{#1}}
\newcommand{\freq}[1]{{\bf \sf F}(#1)}
\newcommand{\datafreq}[2]{{{\bf \sf F}_{#1}(#2)}}

\def\qqquad{\quad\qquad}
\def\qqqquad{\qquad\qquad}

%%%%%%%%%%%%%%%%%%%%%%%%%%%%%%%%%%%%%%%%%%%%%%%%%%%%%%%%%%%%%%%%%%%%%%%%%%%%%%%%
%%%%%%%%%%%%%%%%%%%%%%%%%%%%%%%%%%%%%%%%%%%%%%%%%%%%%%%%%%%%%%%%%%%%%%%%%%%%%%%%

% fill-in-blank question style, found in https://tex.stackexchange.com/a/505089

\usepackage{ifthen}
\usepackage{tocloft}
\usepackage{exercise}
% \usepackage{xcolor}

% Set the Show Answers Boolean
\newboolean{showAns}
\setboolean{showAns}{false}
\newcommand{\showAns}{\setboolean{showAns}{true}}

% The length of the Answer line
\newlength{\answerlength}
\newcommand{\anslen}[1]{\settowidth{\answerlength}{#1}}

% ans command that indicates space for an answer or shows the answer in red
\newcommand{\ans}[1]{\settowidth{\answerlength}{\hspace{2ex}#1\hspace{2ex}}%
    \ifthenelse{\boolean{showAns}}%
        {\textcolor{red}{\underline{\hspace{2ex}#1\hspace{2ex}}}}%
        {\underline{\hspace{\answerlength}}}}%

\newcommand{\details}[1]{\settowidth{\answerlength}{#1}%
    \ifthenelse{\boolean{showAns}}%
        {\begin{tcolorbox}[colback=blue!5!white,colframe=blue!75!black,title=Solution] #1 \end{tcolorbox}}%
        {}}%

% Formatting how multiple choices Questions are formated.
\settasks{label=(\Alph*), label-width=30pt}


% Some commands for the Exercise Question package
\renewcommand{\QuestionNB}{\Large\protect\textcircled{\small\bfseries\arabic{Question}}\ }
\renewcommand{\ExerciseHeader}{} %no header
\renewcommand{\QuestionBefore}{3ex} %Space above each Q
\setlength{\QuestionIndent}{8pt} % Indent after Q number


% To create the list of answers with tocloft...
\newcommand{\listanswername}{Answers}
\newlistof[Question]{answer}{Answers}{\listanswername}

% Creates a TOC for Answers
\newcounter{prevQ}
\newcommand{\answer}[1]{\refstepcounter{answer}%
\ans{#1}%
\ifnum\theQuestion=\theprevQ%
        \addcontentsline{Answers}{answer}{\protect\numberline{}#1}% don't include the Q number
        \else%
        \addcontentsline{Answers}{answer}{\protect\numberline{\theQuestion}#1}%
        \setcounter{prevQ}{\value{Question}}%
        \fi%
        }%

% \hyphenpenalty=10000 % no hyphen
% \exhyphenpenalty=10000 % no hyphen
\sloppy              % hyphen

\newcommand{\HRule}{\rule{\linewidth}{0.5mm}}
\newcommand{\Hrule}{\rule{\linewidth}{0.3mm}}

%tocloft formatting listofanswers
\renewcommand{\cftAnswerstitlefont}{\bfseries\large}
\renewcommand{\cftanswerdotsep}{\cftnodots}
\cftpagenumbersoff{answer}
\addtolength{\cftanswernumwidth}{10pt}

\makeatletter% since there's an at-sign (@) in the command name
\renewcommand{\@maketitle}{%
  \parindent=0pt% don't indent paragraphs in the title block
  \centering
  {\Large \bfseries\textsc{\@title}} \\
  \vspace{5pt}
  {\large \textit{\@author}} \\
  \HRule \\
  \vspace{1em}
}
\makeatother% resets the meaning of the at-sign (@)

\setmainfont{Crimson Pro Light}[
  ItalicFont={* Italic},
  BoldFont={Crimson Pro Medium},
  BoldItalicFont={Crimson Pro Medium Italic}]
\setsansfont{Crimson Pro Light}[
  ItalicFont={* Italic},
  BoldFont={Crimson Pro Medium},
  BoldItalicFont={Crimson Pro Medium Italic}]

\title{Midterm Exam II}
\author{Macroeconomics I \\ Hui-Jun Chen}

\begin{document}

\maketitle

\showAns
% \listofanswer

% \section*{Instruction}

% % Due at 11:59 PM (Eastern Time) on Sunday, June 14, 2022.

% Please answer this problem set on Carmen quizzes ``Problem Set 2''. In the following problems, the part that is in \textbf{\red{red and bold}} are the order of questions that should be answered on Carmen quizzes.

\textbf{Each question in Problem 1 worth 2.5 points, each question in Problem 2 worth 3.5 points, and writing down your name and ID number is worth 1 point. The total is 100.}

\begin{Exercise}

%%%%%%%%%%%%%%%%%%%%%%%%%%%%%%%%%%%%%%%%%%%%%%%%%%%%%
\section{Problem 1: Distorting Taxes with Cobb--Douglas Production}

Recent U.S. legislation---including the Tax Cuts and Jobs Act (2017) and the One Big Beautiful Bill (2025)---has changed marginal household income tax rates and deduction caps. In our model, these appear as a proportional labor-income tax $t$ that alters the after-tax wage $w(1-t)$ in the production function $Y=zN^{\alpha}$. The distortion causes
\[
MRS_{l,C} = w(1-t) \neq MRT_{l,C} = MPN = z\alpha N^{\alpha-1}.
\]

% --- Conceptual Questions ---
\Question In an undistorted economy, efficiency requires equality between
\answer{B}
\begin{tasks}(1)
    \task $MRS_{l,C} = w(1-t)$
    \task $MRS_{l,C} = MRT_{l,C} = MPN$
    \task $MPN > w$
    \task $MRT_{l,C} = w(1+t)$
\end{tasks}

\Question A proportional labor-income tax primarily
\answer{B}
\begin{tasks}(1)
    \task Shifts the PPF outward
    \task Reduces the after-tax wage and labor supply
    \task Raises both consumption and leisure
    \task Increases productivity directly
\end{tasks}

\Question A lump-sum tax is efficient because
\answer{B}
\begin{tasks}(1)
    \task It alters the slope of indifference curves
    \task It does not change the relative price of leisure
    \task It raises marginal utility of income
    \task It lowers government revenue
\end{tasks}

\Question The First Welfare Theorem fails when
\answer{C}
\begin{tasks}(1)
    \task Markets are competitive
    \task Preferences are identical
    \task Distorting taxes create wedges
    \task Government spending = 0
\end{tasks}

\Question The tax wedge measures
\answer{B}
\begin{tasks}(1)
    \task Difference between disposable and gross income
    \task Gap between $MRS_{l,C}$ and $MRT_{l,C}$
    \task Labor elasticity of substitution
    \task Capital income share
\end{tasks}

\Question When $w(1-t)$ falls, equilibrium labor supply
\answer{B}
\begin{tasks}(1)
    \task Rises
    \task Falls due to substitution and income effects
    \task Unchanged
    \task Indeterminate
\end{tasks}

\Question Holding $z$ fixed, a higher $t$ implies
\answer{B}
\begin{tasks}(1)
    \task Higher output
    \task Lower output and consumption
    \task Constant GDP
    \task Larger profit
\end{tasks}

\Question Which policy minimizes efficiency loss for a given revenue?
\answer{B}
\begin{tasks}(1)
    \task Raise proportional tax
    \task Lump-sum taxation
    \task Raise sales tax
    \task Subsidize leisure
\end{tasks}

\Question Raising the standard deduction while lowering credits likely
\answer{B}
\begin{tasks}(1)
    \task Raises effective $t$
    \task Lowers effective $t$ for middle earners
    \task Leaves $t$ unchanged
    \task Has no effect
\end{tasks}

\Question The Laffer curve shows
\answer{B}
\begin{tasks}(1)
    \task Tradeoff between inflation and unemployment
    \task Relationship between $t$ and total tax revenue
    \task Government multipliers
    \task Wage rigidity
\end{tasks}

% --- Theory Questions ---
\Question The firm's wage equals
\answer{A}
\begin{tasks}(1)
    \task $w=z\alpha N^{\alpha-1}$
    \task $w=z(1-\alpha)N^{\alpha}$
    \task $w=zN^{\alpha}$
    \task $w=zN^{1-\alpha}$
\end{tasks}

\Question Profits satisfy
\answer{A}
\begin{tasks}(1)
    \task $\pi=z(1-\alpha)N^{\alpha}$
    \task $\pi=zN^{\alpha-1}$
    \task $\pi=wN-zN^{\alpha}$
    \task $\pi=zN-wN$
\end{tasks}

\Question Equilibrium labor is given by
\answer{A}
\begin{tasks}(1)
    \task $N=\dfrac{A}{1+A}$, $A=\dfrac{\alpha(1-t)}{\alpha(1-t)+1-\alpha}$
    \task $N=(1+A)/A$
    \task $N=A(1+A)$
    \task $N=(1-A)/(1+A)$
\end{tasks}

\Question Government revenue equals
\answer{A}
\begin{tasks}(1)
    \task $R=t w N$
    \task $R=(1-t)wN$
    \task $R=t zN^{\alpha-1}$
    \task $R=z(1-t)N^{\alpha}$
\end{tasks}

\Question The peak of the Laffer curve occurs where
\answer{A}
\begin{tasks}(1)
    \task $\partial R/\partial t = 0$
    \task $R=Y$
    \task $R=w$
    \task $MRS_{l,C}=MRT_{l,C}$
\end{tasks}

\Question Suppose $Y=zN^{\alpha}$ with $z=1$ and $\alpha=0.5$.
If the government imposes a proportional labor-income tax $t=0.2$,
the wage is $w=\alpha z N^{\alpha-1}$, and labor supply satisfies
$MRS_{l,C}=w(1-t)$.  Compute the equilibrium labor $N$.
\answer{C}
\begin{tasks}(1)
    \task $N=0.36$
    \task $N=0.50$
    \task $N=0.64$
    \task $N=0.80$
\end{tasks}

\Question For $z=1$, $\alpha=0.33$, $t=0.3$, equilibrium $N$ is approximately
\answer{C}
\begin{tasks}(1)
    \task 0.25
    \task 0.33
    \task 0.40
    \task 0.50
\end{tasks}

\Question A rise in productivity $z$ causes the revenue-maximizing tax rate $t^{*}$ to
\answer{A}
\begin{tasks}(1)
    \task Increase
    \task Decrease
    \task Stay constant
    \task Drop to zero
\end{tasks}

\Question Graphically, a higher $t$ moves equilibrium
\answer{A}
\begin{tasks}(1)
    \task From planner optimum to distorted CE below same IC
    \task To higher IC
    \task Vertical shift
    \task Rightward along PPF
\end{tasks}

\Question In the U.S. tax code, the state and local tax (SALT) \emph{deduction cap}
limits how much state/local taxes households can deduct from federal taxable income.
When the SALT cap binds in high-tax states, households face a higher \emph{effective}
labor-income tax rate $t$ on the margin. In the Cobb–Douglas model $Y=zN^{\alpha}$ with
$MRS_{l,C}=w(1-t)$ and $w=\alpha z N^{\alpha-1}$, equilibrium employment in those states will

\answer{B}
\begin{tasks}(1)
    \task rise, because the deduction cap increases work incentives.
    \task fall, due to a higher effective marginal tax rate on labor income.
    \task remain unchanged, since SALT is unrelated to labor supply.
    \task be indeterminate without capital taxation.
\end{tasks}

%%%%%%%%%%%%%%%%%%%%%%%%%%%%%%%%%%%%%%%%%%%%%%%%%%%%%
\newpage

\section{Problem 2: Lucas Human Capital Accumulation}
\label{sec:Problem_2__Lucas_Human_Capital_Accumulation}

Reference: Lucas human capital accumulation model (1988 JME)

Credit: Julia K. Thomas

Consider a two-period model, where human capital are accumulated by \blue{spending time in education} rather than \blue{purchasing using output goods}.
\begin{itemize}
    \item the utility function is given by $ U(C, C') = u(C) + u(C') $, i.e., consumer doesn't value leisure.
    \item households are endowed with $ H $ of current human capital at date $ 0 $, and they accumulate future human capital $ H' $ by \blue{spending $ 1-\phi $ fraction of their time endowment to education}. The law of motion for human capital is given by
    %
    \begin{equation}
    \label{eq:HlawofMotion}
        H' = H + (1 - \phi )H
    ,\end{equation}
    %
    where $ 1-\phi $ is the fraction of the time endowment that goes to education so that households can accumulate human capital.

    \item households are endowed with $ K $ amount of capital, and determine the investment at date $ 0 $ to determine their future capital $ K' $ at date $ 1 $. The usage of these capital for consumer is to rent to the firm and earn the per-unit rent $ r $.
        The law of motion for physical capital is given by
        %
        \begin{equation}
        \label{eq:KlawofMotion}
            K' = (1-\delta)K + I
        .\end{equation}
    \item Firm's production function is given by
    %
    \begin{equation}
    \label{eq:FirmProduction}
        Y = K^{\alpha} (\phi H)^{1-\alpha};
        Y' = K'^{\alpha} (\phi' H')^{1-\alpha};
    \end{equation}
    %
    Firm pays the per-unit wage $ w $ for the labor times the level of human capital supplied by the households and pays per-unit capital renting fee $ r $ to consumers.
    \item Consumer owns the whole firm, and claims the whole profit $ \pi $.
    \item There's no government in this model, i.e., $ G = G' = T = T' = B = 0 $.
\end{itemize}

First, let's construct the budget constraint for consumers.

\Question Consider the current budget constraint, what is the labor income for consumer? \answer{B}
\begin{tasks}(4)
    \task $ r K $
    \task $ w \phi H $
    \task $ w K $
    \task $ r \phi H $
\end{tasks}

\Question Consider the current budget constraint, what is the capital income for consumer? \answer{A}
\begin{tasks}(4)
    \task $ r K $
    \task $ w \phi H $
    \task $ w K $
    \task $ r \phi H $
\end{tasks}

\Question \label{budgetConstrant} What is the current budget constraint for consumer? \answer{D}
\begin{tasks}(2)
    \task $ C \le w H + r \phi K - I + \pi $
    \task $ C \le w \phi H + r \phi K - I + \pi $
    \task $ C \le w H + r K - I + \pi $
    \task $ C \le w \phi H + r K - I + \pi $
\end{tasks}

\Question \label{profit} What is the profit for the firm? \answer{C}
\begin{tasks}(2)
    \task $ \pi = Y - wH - rK $
    \task $ \pi = Y - w\phi H - r \phi K $
    \task $ \pi = Y - w\phi H - rK $
    \task $ \pi = Y - w\phi H - r \phi K $
\end{tasks}

\Question In this economy, does the competitive equilibrium and social planner's problem generate the same result? Why? \answer{B}
\begin{tasks}(1)
    \task No, because the first welfare theorem doesn't holds.
    \task Yes, because the first welfare theorem holds.
    \task Yes, because the first welfare theorem don't holds.
    \task No, because the first welfare theorem holds.
\end{tasks}

Let's solve this model using the social planner's problem.

\Question \label{SPPConsumption} Combine your answers in \ref{budgetConstrant} and \ref{profit}, in the perspective of social planner, we can rewrite household's current budget constraint as \answer{A}
\begin{tasks}(2)
    \task $ C \le Y - I $
    \task $ C \le Y - rK - w\phi H - I $
    \task $ C \le Y - r\phi K - w H - I $
    \task $ C \le Y - r\phi K - w\phi H - I $
\end{tasks}

Since the budget constraint is binding, we can replace consumption as your answer in \ref{SPPConsumption}.

Social planner's problem is then given by

%
%
\begin{align}
        \max_{C, C', \phi, K', H'}
            & u(C) + u(C')
        \\
        \text{ s.t. }
            & C = \text{ your answer in \ref{SPPConsumption} }
        \\
            & C' = Y'
        \\
            & H' = H + (1-\phi)H \label{HLaw}
        \\
            & K' = (1-\delta) K + I \label{KLaw}
\end{align}
%

Replace consumption with your answer in \ref{SPPConsumption} and investment with equation \ref{KLaw}, we can rewrite social planner's problem as
%
\begin{equation*}
    \begin{split}
        \max_{
            \underbrace{
            \text{\answer{C}}
            }_{\text{\ref{argument}}}
        }
            & u(
                \underbrace{
                \text{\answer{A}}
                }_{\text{\ref{Uconsumption}}}
               ) + u(K'^{\alpha} (\phi' H')^{1-\alpha})
        \\
        \text{ s.t. }
            & H' = H + (1-\phi)H
        \\
    \end{split}
\end{equation*}
%

\Question \label{argument}
\begin{tasks}(2)
    \task $\phi, K', H', C'$
    \task $\phi, K', C'$
    \task $\phi, K', H'$
    \task $K', H', C'$
\end{tasks}

\Question \label{Uconsumption}
\begin{tasks}(1)
    \task $ K^{\alpha} (\phi H)^{1-\alpha} + (1-\delta) K - K' $
    \task $ K^{\alpha} (\phi H)^{1-\alpha} $
    \task $ K^{\alpha} (\phi H)^{1-\alpha} + (1-\delta) K - K' + r K $
    \task $ K^{\alpha} (\phi H)^{1-\alpha} + (1-\delta) K - K' + w \phi H $
\end{tasks}

\Question \label{phi'} There's one result directly from our model assumption. Since this is a two-period model, and agents don't live to the third period, we know that $ \phi' =$ \answer{D}
\begin{tasks}(4)
    \task $ 0.3 $
    \task $ 0.5 $
    \task $ 0 $
    \task $ 1 $
\end{tasks}

Using your answer in \ref{phi'} as well as substitute $ H' = (2-\phi)H $ into the utility function, we can write the social planner's problem as
%
\begin{equation*}
    \max_{\phi, K'}
            u(
                \underbrace{
                \text{\answer{A}}
                }_{\text{\ref{Uconsumption}}}
               ) + u(
                \underbrace{
                \text{\answer{C}}
                }_{\text{\ref{UconsumptionPrime}}}
                   )
.\end{equation*}
%

\Question \label{UconsumptionPrime}

\begin{tasks}(2)
    \task $ K'^{\alpha} ( (2-\phi) H)^{-\alpha} $
    \task $ K'^{\alpha} ( (2-\phi) H) $
    \task $ K'^{\alpha} ( (2-\phi) H)^{1-\alpha} $
    \task $ K'^{\alpha} H^{1-\alpha} $
\end{tasks}

\Question \label{FOCK'} The first order condition with respect to $ K' $ would leads to \answer{D}

\begin{tasks}(1)
    \task $ \frac{u'(C)}{u'(C')} = K'^{\alpha} ( (2-\phi) H )^{1-\alpha} $
    \task $ \frac{u'(C)}{u'(C')} = \alpha K'^{\alpha-1} H^{1-\alpha} $
    \task $ \frac{u'(C)}{u'(C')} = K'^{\alpha} ( (2-\phi) H )^{1-\alpha} $
    \task $ \frac{u'(C)}{u'(C')} = \alpha K'^{\alpha-1} ( (2-\phi) H )^{1-\alpha} $
\end{tasks}

\Question \label{FOCphi} The first order condition with respect to $ \phi $ would leads to \answer{B}

\begin{tasks}(1)
    \task $ \frac{u'(C)}{u'(C')} = \left(
        \frac{K'}{K}
    \right)^{1-\alpha} \left(
        \frac{2-\phi}{\phi}
    \right)^{-\alpha}$
    \task $ \frac{u'(C)}{u'(C')} = \left(
        \frac{K'}{K}
    \right)^{\alpha} \left(
        \frac{2-\phi}{\phi}
    \right)^{-\alpha}$
    \task $ \frac{u'(C)}{u'(C')} = \left(
        \frac{K'}{K}
    \right)^{\alpha} $
    \task $ \frac{u'(C)}{u'(C')} = \left(
        \frac{K'}{K}
    \right)^{\alpha-1} \left(
        \frac{2-\phi}{\phi}
    \right)^{-\alpha}$
\end{tasks}

Remember that $ MRS_{C, C'} = \frac{u'(C)}{u'(C')} $, and thus your answer in \ref{FOCK'} and \ref{FOCphi} should equal to each other.

\Question Simplify the above equation and we can get \answer{A}

\begin{tasks}(2)
    \task $ \frac{\phi^{\alpha}}{2-\phi} K'  = \alpha K^{\alpha} H^{1-\alpha}$
    \task $ \frac{\phi^{\alpha}}{2-\phi} K'^{\alpha-1}  = \alpha K^{\alpha} H^{1-\alpha}$
    \task $ \frac{\phi}{2-\phi} K'  = \alpha K^{\alpha} H^{1-\alpha}$
    \task $ \frac{\phi^{1-\alpha}}{2-\phi} K'  = \alpha K^{\alpha} H^{1-\alpha}$
\end{tasks}

From the above equation, we can see that the choice variables, $ \phi $ and $ K' $, are equal to $ \alpha K^{\alpha} H^{1-\alpha} $.
Remember that both $ K $ and $ H $ are the endowments and $ \alpha $ is the parameter of production function.

\Question What is the economics intuition of the this equation? \answer{C}

\begin{tasks}(1)
    \task The investment on human capital is a more favorable option than the investment on physical capital in equilibrium
    \task The investment on human capital is a less favorable option than the investment on physical capital in equilibrium
    \task The investment on human capital and the investment on physical capital are equally favorable options in equilibrium
    \task We cannot determine which investment is more favorable in equilibrium
\end{tasks}







% \section{Problem 2: Intertemporal Substitution with CRRA Utility}

% Between 2020 and 2024, fiscal and monetary policy created natural tests for intertemporal substitution: stimulus checks (temporary $y$), student-loan forgiveness (higher lifetime wealth), and interest-rate hikes (higher $r$).
% Consumers choose how to allocate consumption over time according to

% \[
% \max_{c,\,c'} \; u(c) + \frac{u(c')}{1+r},
% \quad
% \text{s.t.}
% \quad
% c + \frac{c'}{1+r} = y - t + \frac{y'-t'}{1+r},
% \]
% where $u(c)=\frac{c^{1-\sigma}}{1-\sigma}$.

% The optimality condition (Euler equation) equates marginal utilities:
% \[
% u'(c) = \frac{u'(c')}{1+r}
% \quad \Rightarrow \quad
% \left( \frac{c'}{c} \right)^{\sigma} = \frac{1}{1+r}.
% \]

% % When $r=0$, the consumer perfectly smooths consumption ($c=c'$); a higher $r$ makes future consumption relatively cheaper, encouraging saving today.

% % --- Conceptual Questions ---
% \Question The Euler equation represents
% \answer{A}
% \begin{tasks}(1)
%     \task The tradeoff between current and future consumption.
%     \task Labor-leisure condition.
%     \task Government solvency.
%     \task Price rigidity.
% \end{tasks}

% \Question When $r$ increases, the budget line
% \answer{A}
% \begin{tasks}(1)
%     \task Rotates clockwise (steeper).
%     \task Shifts outward.
%     \task Flattens.
%     \task Remains unchanged.
% \end{tasks}

% \Question A consumer is a borrower if
% \answer{B}
% \begin{tasks}(1)
%     \task $s>0$.
%     \task $s<0$.
%     \task $c'=0$.
%     \task $y'=0$.
% \end{tasks}

% \Question Consumption smoothing implies
% \answer{A}
% \begin{tasks}(1)
%     \task $c=c'$ under $r=0$ (no intertemporal incentive to save).
%     \task $s=0$ always.
%     \task $c$ rises one-for-one with current income $y$.
%     \task $c'=(1+r)c$.
% \end{tasks}

% \Question Under the Permanent Income Hypothesis, a temporary income rise causes
% \answer{A}
% \begin{tasks}(1)
%     \task A small rise in $c$, larger rise in saving.
%     \task Equal rises in $c$ and $c'$.
%     \task Zero saving.
%     \task A big jump in $c$ only.
% \end{tasks}

% \Question Permanent income increases cause
% \answer{A}
% \begin{tasks}(1)
%     \task Larger effects on both $c$ and $c'$.
%     \task No change.
%     \task Opposite effects.
%     \task A fall in $c'$.
% \end{tasks}

% \Question A higher $\sigma$ implies consumers
% \answer{B}
% \begin{tasks}(1)
%     \task Are more willing to substitute across time.
%     \task Are less willing to substitute across time.
%     \task Are neutral.
%     \task Always borrow more.
% \end{tasks}

% \Question Ricardian Equivalence assumes
% \answer{B}
% \begin{tasks}(1)
%     \task Proportional taxes.
%     \task Lump-sum taxes and perfect credit markets.
%     \task Credit constraints.
%     \task Progressive brackets.
% \end{tasks}

% \Question Ricardian Equivalence fails when
% \answer{C}
% \begin{tasks}(1)
%     \task Consumers live forever.
%     \task Markets are perfect.
%     \task Credit markets are imperfect.
%     \task Government debt $=0$.
% \end{tasks}

% \Question Aggregate consumption is smoother than GDP because
% \answer{B}
% \begin{tasks}(1)
%     \task Firms hoard capital.
%     \task Households smooth through saving.
%     \task Taxes vary.
%     \task Inflation stabilizes demand.
% \end{tasks}

% % --- Theory Questions ---
% \Question The lifetime budget constraint is
% \answer{A}
% \begin{tasks}(1)
%     \task $c + \frac{c'}{1+r} = y - t + \frac{y'-t'}{1+r}$.
%     \task $c=c'$.
%     \task $c+c'=y+y'$.
%     \task $\frac{c'}{c}=1+r$.
% \end{tasks}

% \Question The optimality condition for CRRA utility is
% \answer{A}
% \begin{tasks}(1)
%     \task $(1+r) u'(c) = u'(c')$.
%     \task $u'(c)=u'(c')$.
%     \task $(c'/c) = 1+r$.
%     \task $c'=c(1+r)^{1/\sigma}$.
% \end{tasks}

% \Question If $r>0$, then
% \answer{B}
% \begin{tasks}(1)
%     \task $c=c'$.
%     \task $c>c'$.
%     \task $c<c'$.
%     \task $s<0$.
% \end{tasks}

% \Question Given $y=1.1$, $y'=1.2$, $r=0.1$, $t=t'=0$, $\sigma=2$, the optimal $c$ is approximately
% \answer{B}
% \begin{tasks}(1)
%     \task $0.95$.
%     \task $1.00$.
%     \task $1.05$.
%     \task $1.10$.
% \end{tasks}

% \Question A one-time stimulus ($\Delta y>0$ today only) implies
% \answer{B}
% \begin{tasks}(1)
%     \task A large $\Delta c$.
%     \task A small $\Delta c$, large $\Delta s$.
%     \task No change in saving.
%     \task Lower $c'$ immediately.
% \end{tasks}

% \Question For a borrower, higher $r$
% \answer{C}
% \begin{tasks}(1)
%     \task Raises $c$.
%     \task Lowers both $c$ and $c'$.
%     \task Lowers $c$, ambiguous $c'$.
%     \task Raises both.
% \end{tasks}

% \Question Student-loan forgiveness (lower $t'$) causes
% \answer{B}
% \begin{tasks}(1)
%     \task Lower lifetime wealth.
%     \task Higher lifetime wealth, higher $c$ and $c'$.
%     \task Higher $r$.
%     \task No change.
% \end{tasks}

% \Question If $\sigma=1.5$, $r$ rises from $0.02$ to $0.04$, then $(c'/c)$
% \answer{B}
% \begin{tasks}(1)
%     \task Increases slightly.
%     \task Decreases slightly.
%     \task Remains unchanged.
%     \task Falls to zero.
% \end{tasks}

% \Question Graphically, a higher $r$ shifts the optimum
% \answer{B}
% \begin{tasks}(1)
%     \task Toward higher $c$, lower $c'$.
%     \task Toward lower $c$, higher $c'$.
%     \task No movement.
%     \task Outward.
% \end{tasks}

% \Question According to the Permanent Income Hypothesis, saving spikes in 2020--21 occurred because
% \answer{B}
% \begin{tasks}(1)
%     \task Stimulus was permanent income.
%     \task Stimulus was temporary income, mostly saved.
%     \task Rate cuts forced borrowing.
%     \task Inflation expectations.
% \end{tasks}

\end{Exercise}


\end{document}

