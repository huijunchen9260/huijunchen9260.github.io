\documentclass[14pt]{extarticle}
% \documentclass[14pt]{article}

% \usepackage[style=authoryear,maxbibnames=9,maxcitenames=2,uniquelist=false,backend=biber,doi=false,url=false]{biblatex}
% \addbibresource{$BIB} % bibtex location
% \renewcommand*{\nameyeardelim}{\addcomma\space} % have comma in parencite
\usepackage{natbib}

\usepackage{xcolor}
\usepackage{amsmath}
\newcommand{\tuple}[1]{ \langle #1 \rangle }
%\usepackage{automata}
\usepackage{times}
\usepackage{ltablex}
\usepackage{tasks}

%%%%%% Template
\usepackage{hyperref}
\hypersetup{colorlinks=true,allcolors=blue}

\usepackage{vmargin}
\setpapersize{USletter}
\setmarginsrb{1.0in}{1.0in}{1.0in}{0.6in}{0pt}{0pt}{0pt}{0.4in}

% HOW TO USE THE ABOVE:
%\setmarginsrb{leftmargin}{topmargin}{rightmargin}{bottommargin}{headheight}{headsep}{footheight}{footskip}
%\raggedbottom
% paragraphs indent & skip:
\parindent  0.3cm
\parskip    -0.01cm

\usepackage{tikz}
\usetikzlibrary{backgrounds}

% hyphenation:
% \hyphenpenalty=10000 % no hyphen
% \exhyphenpenalty=10000 % no hyphen
\sloppy

% notes-style paragraph spacing and indentation:
\usepackage{parskip}
\setlength{\parindent}{0cm}

% let derivations break across pages
\allowdisplaybreaks

\newcommand{\orange}[1]{\textcolor{orange}{#1}}
\newcommand{\blue}[1]{\textcolor{blue}{#1}}
\newcommand{\red}[1]{\textcolor{red}{#1}}
\newcommand{\freq}[1]{{\bf \sf F}(#1)}
\newcommand{\datafreq}[2]{{{\bf \sf F}_{#1}(#2)}}

\def\qqquad{\quad\qquad}
\def\qqqquad{\qquad\qquad}

%%%%%%%%%%%%%%%%%%%%%%%%%%%%%%%%%%%%%%%%%%%%%%%%%%%%%%%%%%%%%%%%%%%%%%%%%%%%%%%%
%%%%%%%%%%%%%%%%%%%%%%%%%%%%%%%%%%%%%%%%%%%%%%%%%%%%%%%%%%%%%%%%%%%%%%%%%%%%%%%%

% fill-in-blank question style, found in https://tex.stackexchange.com/a/505089

\usepackage{ifthen}
\usepackage{tocloft}
\usepackage{exercise}
% \usepackage{xcolor}

% Set the Show Answers Boolean
\newboolean{showAns}
\setboolean{showAns}{false}
\newcommand{\showAns}{\setboolean{showAns}{true}}

% The length of the Answer line
\newlength{\answerlength}
\newcommand{\anslen}[1]{\settowidth{\answerlength}{#1}}

% ans command that indicates space for an answer or shows the answer in red
\newcommand{\ans}[1]{\settowidth{\answerlength}{\hspace{2ex}#1\hspace{2ex}}%
    \ifthenelse{\boolean{showAns}}%
        {\textcolor{red}{\underline{\hspace{2ex}#1\hspace{2ex}}}}%
        {\underline{\hspace{\answerlength}}}}%

\newcommand{\details}[1]{\settowidth{\answerlength}{#1}%
    \ifthenelse{\boolean{showAns}}%
        {\\ \textcolor{blue}{#1}}%
        {}}%

% Formatting how multiple choices Questions are formated.
\settasks{label=(\Alph*), label-width=30pt}


% Some commands for the Exercise Question package
\renewcommand{\QuestionNB}{\Large\protect\textcircled{\small\bfseries\arabic{Question}}\ }
\renewcommand{\ExerciseHeader}{} %no header
\renewcommand{\QuestionBefore}{3ex} %Space above each Q
\setlength{\QuestionIndent}{8pt} % Indent after Q number


% To create the list of answers with tocloft...
\newcommand{\listanswername}{Answers}
\newlistof[Question]{answer}{Answers}{\listanswername}

% Creates a TOC for Answers
\newcounter{prevQ}
\newcommand{\answer}[1]{\refstepcounter{answer}%
\ans{#1}%
\ifnum\theQuestion=\theprevQ%
        \addcontentsline{Answers}{answer}{\protect\numberline{}#1}% don't include the Q number
        \else%
        \addcontentsline{Answers}{answer}{\protect\numberline{\theQuestion}#1}%
        \setcounter{prevQ}{\value{Question}}%
        \fi%
        }%

% \hyphenpenalty=10000 % no hyphen
% \exhyphenpenalty=10000 % no hyphen
\sloppy              % hyphen

\newcommand{\HRule}{\rule{\linewidth}{0.5mm}}
\newcommand{\Hrule}{\rule{\linewidth}{0.3mm}}

%tocloft formatting listofanswers
\renewcommand{\cftAnswerstitlefont}{\bfseries\large}
\renewcommand{\cftanswerdotsep}{\cftnodots}
\cftpagenumbersoff{answer}
\addtolength{\cftanswernumwidth}{10pt}

\makeatletter% since there's an at-sign (@) in the command name
\renewcommand{\@maketitle}{%
  \parindent=0pt% don't indent paragraphs in the title block
  \centering
  {\Large \bfseries\textsc{\@title}} \\
  \vspace{5pt}
  {\large \textit{\@author}} \\
  \HRule \\
  \vspace{1em}
}
\makeatother% resets the meaning of the at-sign (@)

\title{Problem Set 1}
\author{Hui-Jun Chen}


%%%%%%%%%%%%%%%%%%%%%%%%%%%%%%%%%%%%%%%%%%%%%%%%%%%%%%%%%%%%%%%%%%%%%%%%%%%%%%%%
%%%%%%%%%%%%%%%%%%%%%%%%%%%%%%%%%%%%%%%%%%%%%%%%%%%%%%%%%%%%%%%%%%%%%%%%%%%%%%%%
\begin{document}

\maketitle

% \showAns
% \listofanswer

\begin{Exercise}

\section{Inflation}
\label{sec:Inflation}

Credit: Sungmin Park

Consider the following list of all final goods produced in an economy during years 2018-2020.

\begin{center}

\begin{tabular}{|c|c|c|c|c|c|c|}
\hline & \multicolumn{2}{|c|}{2018} & \multicolumn{2}{|c|}{2019} & \multicolumn{2}{|c|}{2020} \\
\hline & Quantity & Price & Quantity & Price & Quantity & Price \\
\hline Apples & 100 & $\$ 1.00$ & 120 & $\$ 2.00$ & 150 & $\$ 2.50$ \\
\hline Bananas & 100 & $\$ 0.50$ & 150 & $\$ 0.75$ & 200 & $\$ 1.00$ \\
\hline Cupcakes & 50 & $\$ 2.00$ & 100 & $\$ 2.50$ & 150 & $\$ 3.00$ \\
\hline
\end{tabular}

\end{center}

\subsection{GDP deflator}
\label{sub:GDP_deflator}

Compute this economy's nominal gross domestic product (GDP) and the real GDP in each year, using 2018 as the base year.

Notice that here the Real GDP is calculated using the GDP deflator method.

\Question
Nominal GDP in 2018 is \answer{D}
\begin{tasks}(4)
    \task 235
    \task 240
    \task 245
    \task 250
\end{tasks}

\Question
Nominal GDP in 2019 is \answer{C}
\begin{tasks}(4)
    \task 582.5
    \task 592.5
    \task 602.5
    \task 612.5
\end{tasks}

\Question
Nominal GDP in 2020 is \answer{A}
\begin{tasks}(4)
    \task 1025
    \task 1030
    \task 1035
    \task 1040
\end{tasks}

\Question
Real GDP in 2018 is \answer{D}
\begin{tasks}(4)
    \task 235
    \task 240
    \task 245
    \task 250
\end{tasks}

\Question
Real GDP in 2019 is \answer{B}
\begin{tasks}(4)
    \task 390
    \task 395
    \task 400
    \task 405
\end{tasks}

\Question
Real GDP in 2020 is \answer{C}
\begin{tasks}(4)
    \task 540
    \task 545
    \task 550
    \task 555
\end{tasks}

\Question
GDP deflator in 2019 is \answer{A}
\begin{tasks}(4)
    \task 152.53
    \task 155.63
    \task 152.63
    \task 163.52
\end{tasks}

\Question
GDP deflator in 2020 is \answer{C}
\begin{tasks}(4)
    \task 166.36
    \task 186.42
    \task 186.36
    \task 166.42
\end{tasks}

\subsection{CPI}
\label{sub:CPI}

Continuing to using 2018 as the base year, What is the Consumer Price Index (CPI) in 2019 and 2020?

\Question
CPI in 2019 is \answer{B}
\begin{tasks}(4)
    \task 155
    \task 160
    \task 165
    \task 170
\end{tasks}

\Question
CPI in 2020 is \answer{B}
\begin{tasks}(4)
    \task 100
    \task 200
    \task 300
    \task 400
\end{tasks}

After calculating the CPI, the inflation formula is:
%
\begin{equation*}
    \text{Inflation} = \text{percentage change in CPI} = \frac{CPI_{t} - CPI_{t-1}}{CPI_{t-1}}\times 100
.\end{equation*}
%
What are the inflation rates in 2019 and 2020 based on the CPI?

\Question
Inflation rate in 2019 is \answer{D}
\begin{tasks}(4)
    \task 30\%
    \task 40\%
    \task 50\%
    \task 60\%
\end{tasks}

\Question
Inflation rate in 2020 is \answer{C}
\begin{tasks}(4)
    \task 15\%
    \task 20\%
    \task 25\%
    \task 30\%
\end{tasks}

\subsection{COVID shock}
\label{sub:COVID_shock}

\begin{center}

\begin{tabular}{|l|c|c|}
\hline  & \multicolumn{2}{|c|}{2021} \\
\cline { 2 - 3 } & Quantity & Price \\
\hline Apples & 150 & $\$ 3.00$ \\
\hline Bananas & 200 & $\$ 5.60$ \\
\hline Cupcakes & 150 & $\$ 7.00$ \\
\hline
\end{tabular}

\end{center}

What is the CPI and inflation rate in 2021 using 2018 as base year?

\Question
CPI in 2021 is \answer{A}
\begin{tasks}(4)
    \task 484
    \task 565
    \task 584
    \task 465
\end{tasks}

\Question
Inflation rate in 2021 is \answer{B}
\begin{tasks}(4)
    \task 120\%
    \task 142\%
    \task 20\%
    \task 42\%
\end{tasks}

\Question
Comparing Inflation in 2021 and 2020. Is 2021 experiencing an inflation or deflation? \answer{C}
\begin{tasks}(3)
    \task deflation
    \task stagflation
    \task inflation
\end{tasks}

\section{Employment}
\label{sec:Employment}

Credit: Sungmin Park

\Question
Let $ u $ denote the unemployment rate of an economy. Let $ e  $ denote the fraction of adult population that is employed. What is the labor-force participation rate written in terms of $ u $ and $ e $? \answer{C}
\begin{tasks}(4)
    \task $ \displaystyle \frac{1 - u}{e} $
    \task $ \displaystyle \frac{1 - e}{u} $
    \task $ \displaystyle \frac{e}{1 - u} $
    \task $ \displaystyle \frac{u}{1 - e} $
\end{tasks}

\pagebreak

\section{Computer Exercise}
\label{sec:Computer_Exercise}

Credit: Mike Carter

One of the most important measurements of economic output is Gross Domestic Product. This question asks you to find information about GDP for a few selected time periods to get you some practice using official data. The data we will use is accessible at \url{http://FRED.StLouisFed.org} .

\begin{itemize}
    \item To get to Real GDP, click ``\textbf{CATEGORY}'' $ \Rightarrow  $ ``\textbf{NATIONAL ACCOUNTS}'' $ \Rightarrow $ ``\textbf{NATIONAL INCOME \& PRODUCT ACCOUNTS}'' $ \Rightarrow $ ``\textbf{GDP/GNP}'', then find the data series labeled ``\textbf{Billions of Chained 2012 Dollars, Not Seasonally Adjusted}''.
    \begin{itemize}
        \item I think it's easier to view this data in table form. To do that, click the link halfway down the page to ``\textbf{Table 1.1.6 Real Gross Domestic Product, Chained Dollars: Annual}''.
        \item Be sure you’ve selected ``\textbf{chained dollars}'' to get real GDP
        \item Also be sure you’ve selected ``\textbf{annual}'' so you can see GDP for the whole year
    \end{itemize}
    \item For nominal GDP, click ``\textbf{CATEGORY}'' $ \Rightarrow $ ``\textbf{NATIONAL ACCOUNTS}'' $ \Rightarrow $ ``\textbf{NATIONAL INCOME \& PRODUCT ACCOUNTS}'' $ \Rightarrow $ ``\textbf{GDP/GNP}'', then find the data series labeled ``\textbf{Billions of Dollars, Annual, Not Seasonally Adjusted}''.
    \begin{itemize}
        \item Again, I think this data is easier to use in table form. To find that, click the link halfway down the page to ``\textbf{Table 1.1.5 Gross Domestic Product: Annual}''
        \item For nominal GDP, make sure you don’t see ``\textbf{real}'' or ``\textbf{chained}'' labels
        \item Also be sure to select ``\textbf{annual}'' to find GDP for the whole year
    \end{itemize}
    \item To find population data, click ``\textbf{CATEGORY}'' $ \Rightarrow $ ``\textbf{POPULATION, EMPLOYMENT, \& LABOR MARKETS}'' $ \Rightarrow $ ``\textbf{POPULATION}''. The annual population should be toward the top of the list on that page.
    \begin{itemize}
        \item Unfortunately this series doesn’t have a nice table linked at the bottom of the page. But you can click the ``\textbf{DOWNLOAD}'' button near the top of the page to see values for every year.
    \end{itemize}
\end{itemize}

fill in the table below

\begin{center}

\newlength\q
\setlength\q{\dimexpr .2\textwidth -2\tabcolsep}
\newlength\p
\setlength\p{\dimexpr .4\textwidth -2\tabcolsep}
\begin{tabular}{|p{\p}|p{\q}|p{\q}|p{\q}|}
\hline & 2019 & 1989 & 1956 \\
\hline Nominal GDP & 21372582 million & Q17 & Q18 \\
\hline Real GDP & 19032672 million & Q19 & Q20 \\
\hline Population & Q21 & Q22 & Q23 \\
\hline Nominal GDP per capita & Q24 & Q25 & Q26 \\
\hline Real GDP per capita & Q27 & Q28 & Q29 \\
\hline Implied Deflator & Q30 & Q31 & Q32 \\
\hline
\end{tabular}

\end{center}

*(FRED has updated their numbers for Nominal and Real GDP for 2019. I am following the same numbers as before)

*$ \displaystyle \text{Implied Deflator} = \frac{\text{Nominal GDP per capita}}{\text{Real GDP per capita}} \times  100$

\Question \answer{B}
\begin{tasks}(4)
    \task 5461580 million
    \task 5641580 million
    \task 5645180 million
    \task 5156480 million
\end{tasks}

\Question \answer{C}
\begin{tasks}(4)
    \task 443953 million
    \task 445393 million
    \task 449353 million
    \task 449335 million
\end{tasks}

\Question \answer{A}
\begin{tasks}(4)
    \task 9197997 million
    \task 9179997 million
    \task 9199779 million
    \task 9791997 million
\end{tasks}

\Question \answer{D}
\begin{tasks}(4)
    \task 2939134 million
    \task 2439193 million
    \task 2933941 million
    \task 2934391 million
\end{tasks}

\Question \answer{A}
\begin{tasks}(4)
    \task 330513000
    \task 330351000
    \task 331305000
    \task 351330000
\end{tasks}

\Question \answer{D}
\begin{tasks}(4)
    \task 248773000
    \task 238747000
    \task 248773000
    \task 247387000
\end{tasks}

\Question \answer{B}
\begin{tasks}(4)
    \task 122168000
    \task 168221000
    \task 168212000
    \task 122681000
\end{tasks}

\Question \answer{C}
\begin{tasks}(4)
    \task 66446.88
    \task 68846.64
    \task 64664.88
    \task 64466.88
\end{tasks}

\Question \answer{B}
\begin{tasks}(4)
    \task 20428.67
    \task 22804.67
    \task 22467.80
    \task 20467.28
\end{tasks}

\Question \answer{A}
\begin{tasks}(4)
    \task 2671.21
    \task 2121.67
    \task 2621.71
    \task 2761.21
\end{tasks}

\Question \answer{D}
\begin{tasks}(4)
    \task 57558.24
    \task 57524.58
    \task 58524.75
    \task 57585.24
\end{tasks}

\Question \answer{A}
\begin{tasks}(4)
    \task 37180.60
    \task 37160.80
    \task 38060.71
    \task 37060.18
\end{tasks}

\Question \answer{C}
\begin{tasks}(4)
    \task 17367.44
    \task 17467.43
    \task 17443.67
    \task 14367.74
\end{tasks}

\Question \answer{B}
\begin{tasks}(4)
    \task 121.29
    \task 112.29
    \task 112.92
    \task 121.92
\end{tasks}

\Question \answer{C}
\begin{tasks}(4)
    \task 64.33
    \task 63.33
    \task 61.33
    \task 62.33
\end{tasks}

\Question \answer{B}
\begin{tasks}(4)
    \task 15.13
    \task 15.31
    \task 13.15
    \task 11.53
\end{tasks}

\end{Exercise}

\end{document}
