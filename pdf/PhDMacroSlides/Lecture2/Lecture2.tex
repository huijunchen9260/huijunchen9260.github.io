%----------------------------------------------------------------------------------------
%    PACKAGES AND THEMES
%----------------------------------------------------------------------------------------

\documentclass[aspectratio=169,xcolor=dvipsnames]{beamer}
\usetheme{SimpleDarkBlue}

\usepackage{hyperref}
\usepackage{graphicx} % Allows including images
\usepackage{booktabs} % Allows the use of \toprule, \midrule and \bottomrule in tables

\usepackage{fontspec}
\usepackage{luatexja}

\usepackage{appendixnumberbeamer}

\setmainfont{Alegreya Sans Light}[
  ItalicFont={* Italic},
  BoldFont={Alegreya Sans Medium},
  BoldItalicFont={Alegreya Sans Medium Italic}]
\setsansfont{Alegreya Sans Light}[
  ItalicFont={* Italic},
  BoldFont={Alegreya Sans Medium},
  BoldItalicFont={Alegreya Sans Medium Italic}]

% ------------ %
% beamerbutton %
% ------------ %
\newcommand{\goto}[2]{\hyperlink{#2}{\beamergotobutton{#1}}}
\newcommand{\return}[2]{\hyperlink{#2}{\beamerreturnbutton{#1}}}
\newcommand{\extgoto}[2]{\href{#2}{\beamergotobutton{#1}}}

% % --------------------------- %
% % Section title page with toc %
% % --------------------------- %
% \setbeamertemplate{subsection page}{%
%     \usebeamertemplate*{section page}
% }
% \setbeamertemplate{section in toc}[square]
% \setbeamertemplate{subsection in toc}[square]
% \AtBeginSection[]{
% % \sepframe
% \begin{frame}[noframenumbering]{Outline}
%     % \tableofcontents[currentsection]
%     \tableofcontents[currentsection, currentsubsection]
% \end{frame}
% }
% \AtBeginSubsection[]{
%   \begin{frame}[noframenumbering]{Outline}
%     \tableofcontents[currentsection, currentsubsection]
%   \end{frame}
% }

% ------------------------------------ %
% Section title page with Huge bf text %
% ------------------------------------ %
\AtBeginSection[]{
  \begin{frame}[noframenumbering, plain]
    \Huge{\centerline{\textbf{\insertsection}}}
  \end{frame}
}

%%% automatically add spaces into enumerate and itemize environment
\let\tempone\itemize
\let\temptwo\enditemize
\renewenvironment{itemize}{\tempone\addtolength{\itemsep}{\fill}}{\temptwo}
\let\tempa\enumerate
\let\tempb\endenumerate
\renewenvironment{enumerate}{\tempa\addtolength{\itemsep}{\fill}}{\tempb}

%%=============================================================
%%  FOOTLINE TEMPLATE
%%=============================================================

% % --- Footline depends on onesec option: show only current section or full bar ---
% \if@Optiononesec
% % Minimalist footline: only section name and frame number
% \defbeamertemplate*{footline}{shadow theme}{%
%     \leavevmode\hbox{%
%         \hypersetup{linkcolor=white,urlcolor=white,citecolor=white}
%         \begin{beamercolorbox}[wd=1.01\paperwidth, ht=2.5ex, dp=1.125ex, leftskip=.3cm plus1fil, rightskip=0.3cm]{author in head/foot}%
%             \insertshorttitle \insertsection \hfill \insertframenumber \ / \ \inserttotalframenumber%
%         \end{beamercolorbox}%
%     }%
% }
% \else
% % Standard footline: section bar and frame number
\defbeamertemplate*{footline}{shadow theme}{%
    \leavevmode\hbox{%
        \hypersetup{linkcolor=white,urlcolor=white,citecolor=white}
        \begin{beamercolorbox}[wd=1.02\paperwidth, ht=2.5ex, dp=1.125ex, leftskip=.3cm plus1fil, rightskip=0.3cm]{author in head/foot}%
            \insertsectionnavigationhorizontal{.80\textwidth}{}{} \hspace{.3cm} \hfill \#\ \insertframenumber \ / \ \inserttotalframenumber%
        \end{beamercolorbox}%
    }%
}
% \fi

% --- Activate the custom footline ---
\setbeamertemplate{footline}[shadow theme]


%----------------------------------------------------------------------------------------
%    TITLE PAGE
%----------------------------------------------------------------------------------------

% \title[]{}
% \author{ Hui-Jun Chen\inst{1}
% \and Min Fang\inst{2}
% \and Po-Hsuan Hsu\inst{1}
% \and Chi-Yang Tsou\inst{3}
% \institute[]{
%     \inst{1} National Tsing Hua University \and
%     \inst{2} University of Florida \and
%     \inst{3} University of Manchester \and
%     }\vspace{2pt}
% \date{\today} % Date, can be changed to a custom date


\title[Lecture 2]{Graduate Macro Sequence: \\ Three ways to represent a model}
\author[Hui-Jun Chen]{Hui-Jun Chen}
\institute[]{
    National Tsing Hua University\\
    Department of Economics
    }\vspace{2pt}
\date{\today}

%----------------------------------------------------------------------------------------
%    PRESENTATION SLIDES
%----------------------------------------------------------------------------------------

\begin{document}

% \setbeamercovered{transparent}
%------------------------------------------------
\begin{frame}[plain,noframenumbering]
    % Print the title page as the first slide
    \titlepage
\end{frame}

\begin{frame}{Three ways to represent a model}
\label{slide:L2_Title}
\begin{itemize}
    \item Same economics, three ``lenses'':
    \begin{enumerate}
        \item \textcolor{RoyalBlue}{Date-0 (Arrow--Debreu / planning)}: choose an entire plan at time 0
        \item \textcolor{RoyalBlue}{Sequential (markets each period)}: choose each period subject to a per-period budget
        \item \textcolor{RoyalBlue}{Recursive (dynamic programming)}: choose a \emph{rule} using a state variable
    \end{enumerate}
    \item Key message for beginners:
    \[
        \textbf{Plan} \;\;(\{c_t,k_{t+1}\}_{t\ge0}) \quad \Longleftrightarrow \quad
        \textbf{Rule} \;\;(c=g(x),\; k'=h(x))
    \]
    \item We use the same neoclassical growth environment as an example
\end{itemize}

\end{frame}

%------------------------------------------------

\begin{frame}{Roadmap}
\label{slide:L2_Roadmap}
\begin{enumerate}
    \item Warm-up: \textcolor{RoyalBlue}{Plan vs Rule} in a 2-period problem
    \item Infinite horizon: why ``plan form'' becomes an \textcolor{RoyalBlue}{infinite-dimensional} object
    \item The three representations with Neoclassical Growth Model as an example
    \item Why recursion is \textcolor{RoyalBlue}{computationally} powerful (fixed point / iteration)
\end{enumerate}

\end{frame}

%========================================================================================
\section[Warm-up]{Warm-up: plan vs rule}
%========================================================================================

\begin{frame}{Warm-up: a 2-period consumption--saving problem}
\label{slide:L2_Warmup_2period}
\small
Suppose you live for two periods \(t=0,1\).

\begin{itemize}
    \item Resources:
    \[
        c_0 + a_1 = y_0 + (1+r)a_0,
        \qquad
        c_1 = y_1 + (1+r)a_1.
    \]
    \item Preferences:
    \[
        u(c_0) + \beta u(c_1), \quad \beta\in(0,1).
    \]
\end{itemize}

\vspace{0.5em}
\textbf{Two ways to think about it:}
\begin{enumerate}
    \item \textcolor{RoyalBlue}{Plan:} choose \((c_0,a_1,c_1)\) today.
    \item \textcolor{RoyalBlue}{Rule:} choose \(a_1\) today, and tomorrow consume whatever is feasible.
\end{enumerate}

\end{frame}

%------------------------------------------------

\begin{frame}{The only new idea: continuation value}
\label{slide:L2_ContinuationValue}
\small
At \(t=0\), write the objective as
\[
u(c_0) \;+\; \beta \underbrace{u(c_1)}_{\text{everything after today}}.
\]

In longer horizons, ``everything after today'' is a long tail:
\[
u(c_0) + \beta u(c_1) + \beta^2 u(c_2)+\cdots
\]

\begin{itemize}
    \item DP names this tail: \textcolor{RoyalBlue}{continuation value}.
    \item Continuation value depends on what you carry into tomorrow:
    \[
        \text{tomorrow's situation} \;\approx\; \text{state}.
    \]
\end{itemize}

\vspace{0.6em}
\textbf{Transition:} For infinite horizon, we cannot treat the tail as ``a finite list.''
We compress it into a function \(V(\cdot)\).

\end{frame}

%========================================================================================
\section[Why?]{Why infinite horizon motivates recursion}
%========================================================================================

\begin{frame}{Why infinite horizon is hard in plan form}
\label{slide:L2_InfiniteDim}
\small
Date-0 / sequential formulations ask you to pick an \textcolor{RoyalBlue}{entire sequence}:
\[
\{c_t, k_{t+1}\}_{t=0}^{\infty}.
\]

\begin{itemize}
    \item That is an \textcolor{RoyalBlue}{infinite-dimensional} object.
    \item You can derive elegant \textcolor{RoyalBlue}{conditions} (Euler equation, transversality conditions),
    \item But you still need a way to \textcolor{RoyalBlue}{compute} the policy rules.
\end{itemize}

\vspace{0.6em}
\textbf{DP's goal:} replace an infinite sequence with two \textcolor{RoyalBlue}{functions}:
\[
c_t = g(x_t), \qquad x_{t+1} = f(x_t, g(x_t), \varepsilon_{t+1}).
\]

\end{frame}

%------------------------------------------------

\begin{frame}{What is a state?}
\label{slide:L2_State_Def}
\small
A \textcolor{RoyalBlue}{state variable} \(x_t\) is a summary of ``where you are'' today that is sufficient for:
\begin{enumerate}
    \item choosing optimally today,
    \item predicting the distribution of tomorrow.
\end{enumerate}

\textbf{In the neoclassical growth model (no labor):}
\[
x_t = k_t \quad \text{(current capital)}.
\]
Given \(k_t\), today's choice \(k_{t+1}\) pins down today's consumption:
\[
c_t = f(k_t) + (1-\delta)k_t - k_{t+1}.
\]

\begin{itemize}
    \item Past history matters only through \(k_t\).
    \item So the optimal decision can be written as a \textcolor{RoyalBlue}{rule}:
    \[
        k_{t+1} = h(k_t), \qquad c_t = g(k_t).
    \]
\end{itemize}

\end{frame}

\section[Neoclassical]{Three Representation}
\label{sec:Neoclassical_Growth_Model}

\begin{frame}{Neoclassical Growth Model: Set up}
\label{slide:Optimal_Growth_Model}
    \begin{itemize}
        % \item Difficulties with Solow Model: \blue{exogenous} saving rate.
        % \begin{itemize}
        %     \item \textbf{how} arrived at $ s $? Is $ s $ \textbf{optimal}?
        % \end{itemize}
        \item Micro-foundation: rep. consumer makes consumption-saving decision.
        \item No externalities, and thus can solve in Social planner's problem.
        \item Assume rep. consumer lives for $ \infty $ period with \textbf{additive} separability:
        %
        \begin{equation}
        \label{eq:rep_consumer_u}
            U(C_{0}, C_{1}, \ldots) = \sum_{t=0}^{\infty} \beta^{t} u(C_{t})
        ,\end{equation}
        %
        where function $ u(\cdot) $ is the same for every period, and $ \beta $ is \blue{subjective discount factor}.
        \item Assumes no labor (for the sake of sanity)
        \item Two goods are trading:
        \begin{itemize}
            \item firm $ \rightarrow  $ consumer: consumption goods ($c_{t}$) with price $ p_{t} $
            \item consumer $ \rightarrow  $ firm: capital accumulation ($k_{t}$) with price $ r_{t} $
        \end{itemize}
    \end{itemize}
\end{frame}


\begin{frame}{Date 0 Representation}
\label{slide:Date_0_Representation}
A \blue{Date 0 C.E.} is \textbf{prices} $ \left\{p_{t}, r_{t}\right\}_{t=0}^{\infty} $ and \textbf{quantities} $ \{c_{t}^{*}, k_{t+1}^{*}\}_{t=0}^{\infty} $ such that
\begin{enumerate}
    \item $\{c_{t}^{*}, k_{t+1}^{*}\}_{t=0}^{\infty}$ solves household's problem,
     \begin{align}
             & \max_{\{c_{t}^{*}, k_{t+1}^{*}\}_{t=0}^{\infty}} \sum_{t=0}^{\infty} \beta^{t} u(c_{t})
         \\
         \text{subject to} \quad
             & c_{t} \ge 0, \forall t = 0, 1, \ldots
         \\
             & \sum_{t=0}^{\infty} \red{p_{t}} (c_{t} + k_{t+1}) \le \sum_{t=0}^{\infty} \red{p_{t}} (r_{t} k_{t} + (1-\delta)k_{t}), \forall t
     \end{align}
    \item $\{k_{t+1}^{*}\}_{t=0}^{\infty}$ solves firm's problem at each $ t = 0, 1, \ldots $
    %
    \begin{align}
        \max_{k_{t}} p_{t} f(k_{t}) - p_{t}r_{t}k_{t}
    \end{align}
    %
    \item Goods market clear: $ c_{t}^{*} + k_{t+1}^{*} = f(k_{t}^{*}) + (1-\delta) k_{t}^{*} $
\end{enumerate}

\end{frame}

\begin{frame}{Discussion on Date 0 Representation}
\label{slide:Discussion_on_Date_0_Representation}

\begin{itemize}
    \item $ p_{t} $ is the relative price of $ c_{t} $ \textbf{in units of $ c_{0} $} $ \Rightarrow  $ $ p_{0} = 1 $.
    \item $ p_{t} r_{t} $ is the relative price of capital \textbf{in units of $ c_{0} $}
    \item Firm's problem is static, implies $ r_{t} = D_{k}f(k_{t}) $
    \item Use \textbf{LaGrange multiplier} $ \lambda $, we derive the FOC for $ c_{t} $ and $ k_{t+1} $ are
    %
    \begin{align*}
        [c_{t}]: \quad
            &\beta^{t} u'(c_t) = \lambda p_{t}
        \\
        [k_{t+1}]: \quad
            & p_{t} = p_{t+1} (r_{t+1} + 1 - \delta )
    \end{align*}
    %
    %
    \item If we divide both $ p_{t} $ and $ p_{t+1} $, we get \textbf{Euler equation}:
    %
    \begin{equation*}
            \frac{p_{t}}{p_{t+1}} = \frac{u'(c_{t})}{\beta u'(c_{t+1})} = (r_{t+1} + 1 - \delta)
            \Rightarrow
            u'(c_{t}) = \beta u'(c_{t+1}) (r_{t+1} + 1 - \delta)
    \end{equation*}
    %

\end{itemize}
\end{frame}

\begin{frame}{Sequential Representation}
\label{slide:Sequential_Representation}
    \small
    A \blue{sequential C.E.} is \textbf{prices} $ \{r_{t}\}_{t=0}^{\infty} $ and \textbf{quantities} $ \{c_{t}^{*}, k_{t+1}^{*}\}_{t=0}^{\infty} $ such that
\begin{enumerate}
    \item $\{c_{t}^{*}, k_{t+1}^{*}\}_{t=0}^{\infty}$ solves household's problem,
     \begin{align}
             & \max_{\{c_{t}^{*}, k_{t+1}^{*}\}_{t=0}^{\infty}} \sum_{t=0}^{\infty} \beta^{t} u(c_{t})
         \\
         \text{subject to} \quad
             & c_{t} \ge 0, \forall t = 0, 1, \ldots
         \\
            & c_{t} + k_{t+1} \le r_{t} k_{t} + (1-\delta) k_{t}, \red{\forall t = 0, 1, \ldots}
         \\
            & \label{eq:transversality} \lim_{t\rightarrow \infty} \left(\prod_{s=1}^{t} (r_{t} + 1 - \delta)\right)^{-1} k_{t+1} = 0
     \end{align}
    \item $\{k_{t+1}^{*}\}_{t=0}^{\infty}$ solves firm's problem at each $ t = 0, 1, \ldots $
    %
    \begin{align}
        \max_{k_{t}} f(k_{t}) - r_{t}k_{t}
    \end{align}
    %
    \item Goods market clear: $ c_{t}^{*} + k_{t+1}^{*} = f(k_{t}^{*}) + (1-\delta) k_{t}^{*} $
\end{enumerate}
\end{frame}

\begin{frame}{Discussion on Sequential Representation}
\label{slide:Discussion_on_Sequential_Representation}
    \begin{itemize}
        \item Here we have budget constraint at every possible $ t $, rather than one.
        \item Need \textbf{LaGrange multiplier} $ \lambda_{t} $ for each budget constraint!
        \item FOC for $ c_{t} $ and $ k_{t+1} $ are
            \begin{align*}
                [c_{t}]: \quad
                    &\beta^{t} u'(c_t) = \beta^{t} \lambda_{t} \Rightarrow u'(c_{t}) = \beta \lambda_{t}
                \\
                [k_{t+1}]: \quad
                    & \beta^{t} \lambda_{t} = \beta^{t+1} \lambda_{t+1} (r_{t+1} + 1 - \delta)
                    \Rightarrow \lambda_{t} = \beta \lambda_{t+1} (r_{t+1} + 1 - \delta)
            \end{align*}
        \item and still, we can the same \textbf{Euler equation}:
        %
        \begin{equation*}
                u'(c_{t}) = \beta u'(c_{t+1}) (r_{t+1} + 1 - \delta)
        \end{equation*}
        %
        \item Equation \eqref{eq:transversality} is the \blue{transversality condition}: avoid Ponzi scheme
    \end{itemize}

\end{frame}

\begin{frame}{Motivating Recursive Representation}
\label{slide:Intro__Recursive_Representation}
\small
    \begin{itemize}
        \item In the sequential representation, at each date $ t $, household is solving \textbf{exactly the same} utility optimization problem, so we can write it as:
        %
        \begin{align}
                & \max_{c_{t}, k_{t+1}} u(c_{t}) + \overbrace{\sum_{s = t+1}^{\infty} \beta^{s} u(c_{s})}^{\text{not related to } c_{t}}
            \\
            \text{subject to} \quad
                & c_{t} + k_{t+1} \le r_{t} k_{t} + (1-\delta) k_{t}
            \\
                & c_{t+1} + k_{t+2} \le r_{t+1} k_{t+1} + (1-\delta)k_{t+1}
        \end{align}
        %
        \item Observing this, instead of finding the \textbf{level} of the prices and quantities, we find the \textbf{function} of prices and quantities that express the same problem that household is solving \textbf{at each $t$}.
        \item Note that HH cannot change prices, and thus prices depends on the \textbf{aggregate} state variable, i.e., aggregate capital $ \bar{K} $. In equilibrium $ \bar{K} = k $.
    \end{itemize}

\end{frame}

\begin{frame}{Recursive Representation}
\label{slide:Recursive_Representation}
\small
    A \blue{recursive C.E.} is a set of \blue{functions} for \textbf{prices} $ \{r(\bar{K})\} $ and \textbf{quantities} $ \{G(\bar{K}), g(k, \bar{K})\}$ and value $ V(k, \bar{K}) $ such that
    \begin{enumerate}
        \item $ V(k, \bar{K}) $ solves household's problem,
        %
        \begin{align}
                & V(k, \bar{K}) = \max_{c, k' \ge 0} \left(
                    u(c) + \beta V(k', \bar{K}')
                \right)
            \\
            \text{subject to} \quad
                & c + k' = (r(\bar{K}) + 1 - \delta) k
            \\
                & \bar{K}' = G(\bar{K})
        \end{align}
        %
        \item Prices are competitively determined, i.e., firm's problem implies
        %
        \begin{equation*}
            r(\bar{K}) = f'(\bar{k})
        ,\end{equation*}
        %
        \item Individual decisions are consistent with aggregates when $ k = \bar{K} $, i.e.,
        %
        \begin{equation*}
            G(\bar{K}) = g(\bar{K}, \bar{K})
        \end{equation*}
        %

    \end{enumerate}


\end{frame}

% \begin{frame}{Discussion on Recursive Representation}
% \label{slide:Discussion_on_Recursive_Representation}
%     \begin{itemize}
%         \item Why?! $ \because  $ The only formulation we can put it on computer!
%         \item Date 0: how could you code the budget constraint w/ infinite sum?
%         \item Sequential: infinite number of budget constraint\ldots
%         \item Recursive: through recursion, we can keep iterate on same problem until it \textbf{converges} to a fixed point.
%         \item Difficulties: for each C.E., need to identify the \blue{\textbf{structure}} of the question such that we can represent that structure using \blue{individual} and \blue{aggregate} state variables.
%     \end{itemize}

% \end{frame}

%========================================================================================
\section[Solvable]{Why recursion is solvable/computable}
%========================================================================================

\begin{frame}{Why the recursive form is computable}
\label{slide:L2_WhySolvable}
\small
The recursive form turns the problem into a \textcolor{RoyalBlue}{fixed point}:
\[
V = T(V),
\]
where \(T\) is the \textcolor{RoyalBlue}{Bellman operator}:
\[
(TV)(k)=\max_{k'\in\Gamma(k)}
\left\{
u(f(k)+(1-\delta)k-k')+\beta V(k')
\right\}.
\]

\begin{itemize}
    \item Start with a guess \(V_0\)
    \item Update: \(V_{n+1} = T(V_n)\)
    \item Repeat until \(V_{n+1}\approx V_n\)
\end{itemize}

\vspace{0.4em}
\textbf{Interpretation:} ``Solve the same two-period problem again and again.''

\end{frame}

\begin{frame}{One model, three representations}
\label{slide:L2_ThreeWays_Overview_Formal}
\small
\begin{itemize}
    \item We keep the \blue{same primitives} and \blue{same feasibility} (neoclassical growth).
    \item What changes is the \blue{equilibrium object} we solve for:
    \begin{itemize}
        \item \blue{Date-0:} sequences of prices \(\{p_t,r_t\}_{t\ge0}\) and allocations \(\{c_t,k_{t+1}\}_{t\ge0}\)
        \item \blue{Sequential:} spot prices \(\{r_t\}_{t\ge0}\), allocations, and a \blue{TVC}
        \item \blue{Recursive:} \blue{functions} \(r(\bar K)\), policies \(g(k,\bar K)\), aggregation \(G(\bar K)\), value \(V(k,\bar K)\)
    \end{itemize}
\end{itemize}

\vspace{0.7em}
\begin{center}
\begin{tabular}{@{}p{0.30\linewidth}p{0.32\linewidth}p{0.34\linewidth}@{}}
\toprule
\blue{Date-0 CE} & \blue{Sequential CE} & \blue{Recursive CE} \\
\midrule
One PV constraint & Per-period constraints & Bellman + consistency \\
                  & Transversality condition &                       \\
Prices: \(\{p_t,r_t\}\) & Prices: \(\{r_t\}\) & Prices: \(r(\bar K)\) \\
Allocations: sequences & Allocations: sequences & Allocations: policy rules \\
\bottomrule
\end{tabular}
\end{center}

\end{frame}


\begin{frame}{Takeaway}
\label{slide:L2_Takeaway}
\small
\begin{enumerate}
    \item A model can be written as \textcolor{RoyalBlue}{plan}, \textcolor{RoyalBlue}{sequential}, or \textcolor{RoyalBlue}{recursive}.
    \item They describe the \textcolor{RoyalBlue}{same economics} but emphasize different objects.
    \item DP is the step that turns ``choose an infinite sequence'' into ``compute a rule.''
\end{enumerate}

\vspace{1em}
\begin{center}
\textcolor{RoyalBlue}{\Large Plan} \(\;\Longleftrightarrow\;\) \textcolor{RoyalBlue}{\Large Rule}
\end{center}

\end{frame}



\end{document}
