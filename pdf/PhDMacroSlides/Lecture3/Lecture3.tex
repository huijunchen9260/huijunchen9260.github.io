%----------------------------------------------------------------------------------------
%    PACKAGES AND THEMES  (match Lecture2 template style)
%----------------------------------------------------------------------------------------

\documentclass[aspectratio=169,xcolor=dvipsnames]{beamer}
\usetheme{SimpleDarkBlue}

\usepackage{hyperref}
\usepackage{graphicx}
\usepackage{booktabs}
\usepackage{amsmath,amssymb,mathtools}

\usepackage{fontspec}
\usepackage{luatexja}
\usepackage{appendixnumberbeamer}

\setmainfont{Alegreya Sans Light}[
  ItalicFont={* Italic},
  BoldFont={Alegreya Sans Medium},
  BoldItalicFont={Alegreya Sans Medium Italic}]
\setsansfont{Alegreya Sans Light}[
  ItalicFont={* Italic},
  BoldFont={Alegreya Sans Medium},
  BoldFont={Alegreya Sans Medium},
  BoldItalicFont={Alegreya Sans Medium Italic}]

% ------------------------------------ %
% Section title page with Huge bf text %
% ------------------------------------ %
\AtBeginSection[]{
  \begin{frame}[noframenumbering, plain]
    \Huge{\centerline{\textbf{\insertsection}}}
  \end{frame}
}

%%% automatically add spaces into enumerate and itemize environment
\let\tempone\itemize
\let\temptwo\enditemize
\renewenvironment{itemize}{\tempone\addtolength{\itemsep}{\fill}}{\temptwo}
\let\tempa\enumerate
\let\tempb\endenumerate
\renewenvironment{enumerate}{\tempa\addtolength{\itemsep}{\fill}}{\tempb}

%----------------------------------------------------------------------------------------
%  FOOTLINE TEMPLATE  (same as Lecture2.tex)
%----------------------------------------------------------------------------------------
\defbeamertemplate*{footline}{shadow theme}{%
    \leavevmode\hbox{%
        \hypersetup{linkcolor=white,urlcolor=white,citecolor=white}
        \begin{beamercolorbox}[wd=1.02\paperwidth, ht=2.5ex, dp=1.125ex, leftskip=.3cm plus1fil, rightskip=0.3cm]{author in head/foot}%
            \insertsectionnavigationhorizontal{.80\textwidth}{}{} \hspace{.3cm} \hfill \#\ \insertframenumber \ / \ \inserttotalframenumber%
        \end{beamercolorbox}%
    }%
}
\setbeamertemplate{footline}[shadow theme]
\setbeamertemplate{navigation symbols}{}

%----------------------------------------------------------------------------------------
%  Convenience macros (safe)
%----------------------------------------------------------------------------------------
\providecommand{\blue}[1]{\textcolor{RoyalBlue}{#1}}
\providecommand{\red}[1]{\textcolor{BrickRed}{#1}}
\newcommand{\norminf}[1]{\left\lVert #1 \right\rVert_{\infty}}
\newcommand{\E}{\mathbb{E}}
\usepackage{amsmath,amssymb,mathtools}

\providecommand{\orange}[1]{\textcolor{Orange}{#1}}
\providecommand{\norminf}[1]{\|#1\|_\infty}

%----------------------------------------------------------------------------------------
%    TITLE PAGE
%----------------------------------------------------------------------------------------
\title[Lecture 3]{Graduate Macro Sequence:\\ Dynamic Programming}
\author[Hui-Jun Chen]{Hui-Jun Chen}
\institute[]{National Tsing Hua University\\ Department of Economics}
\date{\today}

%----------------------------------------------------------------------------------------
%    PRESENTATION SLIDES
%----------------------------------------------------------------------------------------
\begin{document}

\begin{frame}[plain,noframenumbering]
    \titlepage
\end{frame}

%========================================================================================
\section[DP]{Dynamic Programming}
%========================================================================================

\begin{frame}{Lecture 3: What we need for computation}
\label{slide:L3_Intro}
\small
\begin{itemize}\setlength{\itemsep}{0.35em}
    \item We want to solve a Bellman equation: \(V = TV\).
    \item \blue{Key idea:} treat this as a \textbf{fixed point problem} for an operator \(T\).
    \item Two questions:
    \begin{enumerate}\setlength{\itemsep}{0.25em}
        \item When does a fixed point exist and is it unique?
        \item If it exists, how do we \textbf{find it on a computer}?
    \end{enumerate}
    \item The workhorse theorem: \blue{Contraction Mapping Theorem}.
\end{itemize}
\end{frame}

%========================================================================================
\section[CMT]{Contraction Mapping Theorem}
%========================================================================================

\begin{frame}{Contraction Mapping Theorem}
\label{slide:L3_CMT_1}
\small
\begin{itemize}\setlength{\itemsep}{0.45em}
    \item \textbf{Definition (Contraction Mapping).}
    Let \((S,d)\) be a metric space and \(T:S\to S\) be a mapping of \(S\) into itself.
    \(T\) is a contraction mapping with modulus \(\beta\) if for some \(\beta\in(0,1)\),
    \[
        d(Tv_1,Tv_2)\le \beta\, d(v_1,v_2)\qquad \text{for all } v_1,v_2\in S.
    \]
\end{itemize}
\end{frame}

\begin{frame}{Contraction Mapping Theorem}
\label{slide:L3_CMT_2}
\small
\begin{itemize}\setlength{\itemsep}{0.45em}
    \item \textbf{Contraction Mapping Theorem.}
    Let \((S,d)\) be a \textbf{complete} metric space and suppose that \(T:S\to S\)
    is a contraction mapping. Then, \(T\) has a \textbf{unique} fixed point \(v^\ast\in S\) such that
    \[
        Tv^\ast = v^\ast
        \;=\;
        \lim_{N\to\infty} T^{N}v_0
        \qquad \text{for all } v_0\in S.
    \]
    \item The beauty of CMT is that it is a \blue{constructive theorem}:
    it not only tells us existence/uniqueness of \(v^\ast\), but also shows us how to find it
    (\(v_{n+1}=Tv_n\)).
\end{itemize}
\end{frame}

\begin{frame}{(Optional) Why \(T^N v_0\) converges: the key inequality}
\label{slide:L3_CMT_KeyIneq}
\footnotesize
Let \(v_{n+1}=Tv_n\). Because \(T\) is a contraction,
\[
d(v_{n+1},v_n)=d(Tv_n,Tv_{n-1})\le \beta\, d(v_n,v_{n-1})
\Rightarrow d(v_{n+1},v_n)\le \beta^n d(v_1,v_0).
\]
Then for any \(m\ge 1\),
\[
d(v_{n+m},v_n)
\le \sum_{j=0}^{m-1} d(v_{n+j+1},v_{n+j})
\le \sum_{j=0}^{m-1}\beta^{n+j} d(v_1,v_0)
\le \frac{\beta^n}{1-\beta}\, d(v_1,v_0).
\]
So \(\{v_n\}\) is Cauchy; completeness gives convergence to some \(v^\ast\in S\).
(Then show \(Tv^\ast=v^\ast\), and uniqueness follows from contraction.)
\end{frame}

%========================================================================================
\section[CMT in DP]{How to use CMT in Dynamic Programming}
%========================================================================================

\begin{frame}{Step 1: choose the function space}
\label{slide:L3_FunctionSpace}
\small
In DP, \(v(\cdot)\) is a \textbf{function}. A standard choice:
\[
S = \mathcal{B}(X)\equiv \{v:X\to\mathbb{R}\;:\;\sup_{x\in X}|v(x)|<\infty\},
\quad
d(v,w)=\norminf{v-w}.
\]
\begin{itemize}\setlength{\itemsep}{0.35em}
    \item \(\big(\mathcal{B}(X),\norminf{\cdot}\big)\) is a \textbf{complete} metric space.
    \item So CMT applies if we can show \(T\) is a \textbf{contraction} on \(\mathcal{B}(X)\).
\end{itemize}
\end{frame}

\begin{frame}{Step 2: Bellman operator}
\label{slide:L3_BellmanOp}
\small
Consider a discounted DP:
\[
(TV)(x)\equiv \max_{a\in\Gamma(x)}
\left\{ u(x,a) + \beta\, \E\big[V(x')\mid x,a\big]\right\},
\qquad \beta\in(0,1).
\]
\begin{itemize}\setlength{\itemsep}{0.35em}
    \item If we can show \(T\) is a contraction on \(\mathcal{B}(X)\),
    then CMT gives \(\exists!\,V^\ast\) and \(V^\ast=\lim_{N\to\infty} T^N V_0\).
    \item Question: \textbf{How do we show \(T\) is a contraction?}
\end{itemize}
\end{frame}

%========================================================================================
\section[Blackwell]{Blackwell's sufficient conditions}
%========================================================================================

\begin{frame}{Intuition before Blackwell: discounting is geometric shrinkage}
\label{slide:L3_Blackwell_Intuition_Geometric}
\small
\begin{itemize}\setlength{\itemsep}{0.45em}
    \item Recall the geometric series:
    \[
        1+\beta+\beta^2+\cdots=
        \begin{cases}
            \dfrac{1}{1-\beta}, & 0<\beta<1,\\[0.6em]
            \infty, & \beta\ge 1.
        \end{cases}
    \]
    \item \blue{Economic meaning:} discounting makes the infinite future \emph{finite} because weights shrink geometrically.
    \item \blue{DP intuition:} suppose your continuation value is wrong by at most \(m\) everywhere:
    \[
        \norminf{V-W}\le m.
    \]
    Then the one-period-ahead value difference coming purely from the future is at most
    \[
        \beta\,\E[V(x')-W(x')] \le \beta m.
    \]
    \item Repeating the logic across iterations gives a geometric tightening:
    \[
        \norminf{T^nV-T^nW}\ \le\ \beta^n \norminf{V-W}.
    \]
\end{itemize}

\vspace{0.3em}
\textbf{Takeaway:} \(\beta<1\) is the ``geometric force'' behind contraction and convergence.
\end{frame}

\begin{frame}{Blackwell's sufficient conditions (sufficient, not necessary)}
\label{slide:L3_Blackwell_1}
\small
\begin{itemize}\setlength{\itemsep}{0.35em}
    \item CMT requires \(T\) to be a contraction.
    \item \blue{Blackwell provides sufficient conditions} to verify contraction easily.
    \item They are \textbf{not necessary}: an operator can be a contraction even if these fail.
\end{itemize}

\begin{block}{Blackwell's sufficient conditions}
Let \(T:\mathcal{B}(X)\to \mathcal{B}(X)\). Suppose:
\begin{enumerate}\setlength{\itemsep}{0.2em}
    \item \blue{Monotonicity:} \(V\le W \Rightarrow TV\le TW\).
    \item \blue{Discounting:} there exists \(\beta\in(0,1)\) s.t. for any constant \(a\ge 0\),
    \[
        T(V+a)\le TV + \beta a.
    \]
\end{enumerate}
Then \(T\) is a contraction with modulus \(\beta\) under \(\norminf{\cdot}\).
\end{block}
\end{frame}

\begin{frame}{Why Blackwell is useful here: discounting comes from \(\beta\)}
\label{slide:L3_Blackwell_2}
\small
For the Bellman operator
\[
(TV)(x)=\max_{a\in\Gamma(x)}\{u(x,a)+\beta \E[V(x')\mid x,a]\},
\]
\begin{itemize}\setlength{\itemsep}{0.35em}
    \item \textbf{Monotonicity:} if \(V\le W\), then \(\E[V]\le \E[W]\), so the RHS is smaller for every action,
    hence \(TV\le TW\).
    \item \textbf{Discounting:} add a constant \(c\ge 0\) to \(V\):
    \[
        T(V+c)(x)
        =\max_{a}\{u(x,a)+\beta \E[V(x')+c]\}
        =\max_{a}\{u(x,a)+\beta \E[V(x')]\} + \beta c
        = TV(x)+\beta c.
    \]
\end{itemize}
So Blackwell implies: \(\norminf{TV-TW}\le \beta \norminf{V-W}\).
\end{frame}

%========================================================================================
\section[VFI]{Value Function Iteration and acceleration}
%========================================================================================

\begin{frame}{Standard Value Function Iteration (VFI)}
\label{slide:L3_VFI}
\small
\begin{itemize}\setlength{\itemsep}{0.35em}
    \item Once we know \(T\) is a contraction, CMT implies:
    \[
        V^\ast = \lim_{N\to\infty} T^N V_0 \quad \text{for any initial guess } V_0.
    \]
    \item \textbf{Algorithm (Standard VFI):}
    \begin{enumerate}\setlength{\itemsep}{0.25em}
        \item Choose \(V_0\in\mathcal{B}(X)\).
        \item Iterate \(V_{n+1}=TV_n\).
        \item Stop when \(\norminf{V_{n+1}-V_n}\) is small.
    \end{enumerate}
    \item When \(\beta\approx 1\), VFI can be \blue{very slow}.
\end{itemize}
\end{frame}

\begin{frame}{A practical error bound from contraction}
\label{slide:L3_ErrorBound}
\small
For a contraction with modulus \(\beta\),
\[
\norminf{V^\ast - V_n}
\;\le\;
\frac{1}{1-\beta}\,\norminf{V_{n+1}-V_n}.
\]
\begin{itemize}\setlength{\itemsep}{0.35em}
    \item This gives a stopping rule: to be within \(\varepsilon\), it is enough to have
    \(\norminf{V_{n+1}-V_n} < (1-\beta)\varepsilon\).
\end{itemize}
\end{frame}

%------------------------------------------------
\subsection{Howard's Policy Iteration and MPI}
%------------------------------------------------

\begin{frame}{Howard's Policy Iteration}
\label{slide:L3_Howard_Intro}
\small
\begin{itemize}
    \item Howard's policy iteration follows from \orange{\textbf{two key observations}} about VFI:
    \begin{itemize}
        \item The \textbf{maximization step} is \red{\textbf{typically much more costly}} (in computation time)
              than the \textbf{evaluation step}.
        \item But the evaluation step uses the \blue{updated decision rule} \(\tilde{s}_n(k_i,z_j)\) for
              \red{\textbf{only one period}} (since decisions after tomorrow are embedded in \(V_n\) on the RHS).
    \end{itemize}

    \item \textbf{Policy Iteration:} repeat the evaluation step multiple times between each maximization step.

    \item \textbf{Definition:} for a given value function \(J\) and a decision rule \(w\),
    define \textbf{Howard's mapping} \(\tilde{T}_w\) as the operator that
    \red{``plugs in \(w\)''} into the RHS of the Bellman equation:
    \[
        \tilde{T}_w J(k_i,z_j)
        \equiv
        u\!\Big(\underbrace{e^{z_j}k_i^{\alpha}+(1-\delta)k_i-w(k_i,z_j)}_{\equiv\ c(k_i,z_j;w)}\Big)
        +\beta\,\E\!\left[J\!\big(w(k_i,z_j),z'\big)\mid z_j\right].
    \]
\end{itemize}
\end{frame}

%------------------------------------------------

\begin{frame}{Howard's Policy Iteration: Cont'd}
\label{slide:L3_Howard_Cont}
\small
\begin{itemize}
    \item We will be interested in applying the Howard mapping repeatedly, so for \(m=1,\ldots,M\), let
    \[
        J_{m+1}\ \equiv\ \tilde{T}_w\,J_m(k_i,z_j)
    \]
    denote the updated value function.

    \item Howard (1962)'s key insight is that \(\tilde{T}_w\) \textbf{is also a contraction mapping}
          with modulus \(\beta\).
          \[
              \Rightarrow\ \text{applying }\tilde{T}_w\text{ repeatedly converges to a fixed point at rate }\beta.
          \]

    \item Of course, this fixed point is \textbf{not} the solution of the original Bellman equation
          (since the policy function \(w\) is held fixed while only the value is updated).

    \item But \(\tilde{T}_w\) is an operator that is much cheaper to apply, so it is natural to apply it more than once.
\end{itemize}
\end{frame}

%------------------------------------------------

\begin{frame}{Algorithmus: VFI with Policy Iteration}
\label{slide:L3_Alg2_VFI_PI}
\small
\begin{enumerate}
    \item Set \(n=0\). Choose an initial guess \(V_0\in\mathcal{S}\).
    \item \textbf{Maximization Step:} obtain \(\tilde{s}_n\) (the updated decision rule) from \(V_n\).
    \item \textbf{Howard Step:} set \(J_0\equiv V_n\), and iterate on
    \[
        J_{m+1}\ \equiv\ \tilde{T}_{\tilde{s}_n}\,J_m(k_i,z_j),
        \qquad m=0,1,2,\ldots
    \]
    then set \(V_{n+1}=J^\ast\equiv \lim_{m\to\infty}J_m\).
    \item Stop if convergence criterion is satisfied:
    \[
        \lVert V_{n+1}-V_n\rVert_\infty < \text{toler.}
    \]
    Otherwise, increase \(n\) and return to step 2.
\end{enumerate}

\medskip
\begin{itemize}
    \item \textbf{Note:} it is often possible to obtain the fixed point \(J^\ast\) in a finite number of steps.
\end{itemize}
\end{frame}

%------------------------------------------------

\begin{frame}{VFI with Modified Policy Iteration (MPI) Algorithm}
\label{slide:L3_MPI}
\small
\begin{itemize}
    \item \orange{\textbf{Modify Step 3}} of Howard's algorithm above:

    \begin{itemize}
        \item \textbf{Modified Howard Step:}
        set \(J_0\equiv V_n\), and iterate on
        \[
            J_{m+1}\ \equiv\ \tilde{T}_{\tilde{s}_n}\,J_m(k_i,z_j),
            \qquad m=0,1,2,\ldots,M<\infty.
        \]
        Choose a \textbf{moderate} value for \(M\) (by experimentation; use smaller \(M\) for more challenging problems).
        Then set \(V_{n+1}=J_M\).
    \end{itemize}

    \item The choice of \(M\) is a \blue{\textbf{key tuning parameter}} in practice.
\end{itemize}
\end{frame}

%------------------------------------------------
\subsection[MQP]{MacQueen--Porteus (MQP) bounds}
%------------------------------------------------

\begin{frame}{MacQueen--Porteus (MQP) bounds: why we care}
\label{slide:L3_MQP_1}
\small
\begin{itemize}\setlength{\itemsep}{0.35em}
    \item Iterative algorithms need a stopping rule.
    \item MQP bounds provide \blue{upper and lower bounds} on \(V^\ast\) (especially clean in discrete state problems).
\end{itemize}

Let \(V_{n+1}=TV_n\) and define increments
\[
\Delta_n(x)\equiv V_n(x)-V_{n-1}(x),\quad
\underline{\Delta}_n\equiv \min_x \Delta_n(x),\quad
\overline{\Delta}_n\equiv \max_x \Delta_n(x).
\]
\end{frame}

\begin{frame}{MacQueen--Porteus (MQP) bounds: the bound}
\label{slide:L3_MQP_2}
\small
Define correction terms
\[
\underline{c}_n \equiv \frac{\beta}{1-\beta}\,\underline{\Delta}_n,
\qquad
\overline{c}_n \equiv \frac{\beta}{1-\beta}\,\overline{\Delta}_n.
\]
Then (under standard conditions; most transparent in finite/discrete state problems),
\[
V_n(x)+\underline{c}_n \;\le\; V^\ast(x) \;\le\; V_n(x)+\overline{c}_n
\qquad \text{for all } x.
\]

\begin{itemize}\setlength{\itemsep}{0.35em}
    \item Practical stopping: stop when \(\overline{c}_n-\underline{c}_n\) is small.
\end{itemize}
\end{frame}

%========================================================================================
\section{Wrap-up}
%========================================================================================

\begin{frame}{Wrap-up}
\label{slide:L3_WrapUp}
\small
\begin{itemize}\setlength{\itemsep}{0.4em}
    \item CMT: if \(T\) is a contraction on a complete metric space, then a unique fixed point exists
    \item In DP, the job is to show the Bellman operator \(T\) is a contraction.
    \item Blackwell conditions are a \blue{sufficient (not necessary)} way to construst contraction:
    discounting with \(\beta\).
    \item Algorithms:
    \begin{itemize}\setlength{\itemsep}{0.25em}
        \item VFI (guaranteed but slow when \(\beta\approx 1\))
        \item Howard / Modified Policy Iteration (speed)
        \item MQP bounds (error bounds / stopping rules)
    \end{itemize}
\end{itemize}
\end{frame}

\end{document}
