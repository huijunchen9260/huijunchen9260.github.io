\documentclass[14pt]{extarticle}

% \usepackage[style=authoryear,maxbibnames=9,maxcitenames=2,uniquelist=false,backend=biber,doi=false,url=false]{biblatex}
% \addbibresource{$BIB} % bibtex location
% \renewcommand*{\nameyeardelim}{\addcomma\space} % have comma in parencite
\usepackage{natbib}

\usepackage{xcolor}
\usepackage{amsmath}
\newcommand{\tuple}[1]{ \langle #1 \rangle }
%\usepackage{automata}
\usepackage{times}
\usepackage{ltablex}

%%%%%% Template
\usepackage{hyperref}
\hypersetup{colorlinks=true,allcolors=blue}

\usepackage{vmargin}
\setpapersize{USletter}
\setmarginsrb{1.0in}{1.0in}{1.0in}{0.6in}{0pt}{0pt}{0pt}{0.4in}

% HOW TO USE THE ABOVE:
%\setmarginsrb{leftmargin}{topmargin}{rightmargin}{bottommargin}{headheight}{headsep}{footheight}{footskip}
%\raggedbottom
% paragraphs indent & skip:
\parindent  0.3cm
\parskip    -0.01cm

\usepackage{tikz}
\usetikzlibrary{backgrounds}

% hyphenation:
% \hyphenpenalty=10000 % no hyphen
% \exhyphenpenalty=10000 % no hyphen
\sloppy

% notes-style paragraph spacing and indentation:
\usepackage{parskip}
\setlength{\parindent}{0cm}

% let derivations break across pages
\allowdisplaybreaks

\newcommand{\orange}[1]{\textcolor{orange}{#1}}
\newcommand{\blue}[1]{\textcolor{blue}{#1}}
\newcommand{\red}[1]{\textcolor{red}{#1}}
\newcommand{\freq}[1]{{\bf \sf F}(#1)}
\newcommand{\datafreq}[2]{{{\bf \sf F}_{#1}(#2)}}

\def\qqquad{\quad\qquad}
\def\qqqquad{\qquad\qquad}

%%%%%%%%%%%%%%%%%%%%%%%%%%%%%%%%%%%%%%%%%%%%%%%%%%%%%%%%%%%%%%%%%%%%%%%%%%%%%%%%
%%%%%%%%%%%%%%%%%%%%%%%%%%%%%%%%%%%%%%%%%%%%%%%%%%%%%%%%%%%%%%%%%%%%%%%%%%%%%%%%
\begin{document}

% \setcounter{section}{}
\centerline{\huge\bf Problem Set 2}
\smallskip
\centerline{\LARGE Hui-Jun Chen}

\medskip

\section*{Instruction}

Due at 11:59 PM (Eastern Time) on Sunday, June 14, 2022.

Please answer this problem set on Carmen quizzes ``Problem Set 2''. In the following problems, the part that is in \textbf{\red{red and bold}} are the order of questions that should be answered on Carmen quizzes.


\section*{Problem 1}

Remember the Example in Lecture 8.

    Consumer: $ \max_{C, l} \ln C + \ln l \quad \text{subject to} \quad C \le w( 1-l ) + \pi $
    %
    \begin{align}
        \text{FOC} \quad
            & \frac{C}{l} = w
            \label{eq:consumerFOC}
        \\
        \text{Binding budget constraint} \quad
            & C = w ( 1-l ) + \pi
            \label{eq:binding_budget}
        \\
        \text{Time constraint} \quad
            & N^{s} = 1 - l
            \label{eq:time_budget}
    \end{align}
    %

    Firm: $ \max_{N^{d}} ( N^{d} )^{\frac{1}{2}} - w N^{d} $
    %
    \begin{align}
        \text{FOC} \quad
            & \frac{1}{2} ( N^{d} )^{- \frac{1}{2}} = w
            \label{eq:firmFOC}
        \\
        \text{Output definition} \quad
            & Y = ( N^{d} )^{\frac{1}{2}}
            \label{eq:outputDef}
        \\
        \text{Profit definition} \quad
            & \pi = Y - w N^{d}
            \label{eq:profitDef}
    \end{align}
    %
    Market clear:
    %
    \begin{align}
        N^{s} & = N^{d}
        \label{eq:laborClear}
    \end{align}
    %

Fill the following blanks for the step-by-step guide for algebraic calculation:

\begin{enumerate}
    \item Step 1: Impose Market clear condition, so shrink all $ 7 $ equations to \textbf{\red{\underline{\quad 6 \quad}}} equations

    Consumer: $ \max_{C, l} \ln C + \ln l \quad \text{subject to} \quad C \le w( 1-l ) + \pi $
    %
    \begin{align}
        \text{FOC} \quad
            & \frac{C}{l} = w
            \label{eq:consumerFOC}
        \\
        \text{Binding budget constraint} \quad
            & C = w N + \pi
            \label{eq:binding_budget}
        \\
        \text{Time constraint} \quad
            & N = 1 - l
            \label{eq:time_budget}
    \end{align}
    %

    Firm: $ \max_{N} ( N )^{\frac{1}{2}} - w N $
    %
    \begin{align}
        \text{FOC} \quad
            & \frac{1}{2} ( N )^{- \frac{1}{2}} = w
            \label{eq:firmFOC}
        \\
        \text{Output definition} \quad
            & Y = ( N )^{\frac{1}{2}}
            \label{eq:outputDef}
        \\
        \text{Profit definition} \quad
            & \pi = Y - w N
            \label{eq:profitDef}
    \end{align}
    %
    \item Step 2: replace $ l $ in terms of $ N $ using $ l = 1-N $

    Consumer: $ \max_{C, l} \ln C + \ln l \quad \text{subject to} \quad C \le w( 1-l ) + \pi $
    %
    \begin{align}
        \text{FOC} \quad
            & \frac{C}{(\textbf{\red{\underline{\quad 1-N \quad}}})} = w
            \label{eq:consumerFOC}
        \\
        \text{Binding budget constraint} \quad
            & C = w (\textbf{\red{\underline{\quad N \quad}}}) + \pi
            \label{eq:binding_budget}
    \end{align}
    %

    Firm: $ \max_{N} ( N )^{\frac{1}{2}} - w N $
    %
    \begin{align}
        \text{FOC} \quad
            & \frac{1}{2} ( N )^{- \frac{1}{2}} = w
            \label{eq:firmFOC}
        \\
        \text{Output definition} \quad
            & Y = ( N )^{\frac{1}{2}}
            \label{eq:outputDef}
        \\
        \text{Profit definition} \quad
            & \pi = Y - w N
            \label{eq:profitDef}
    \end{align}
    %
    \item Step 3: replace $ \pi $ and $ Y $ as $ N $

    Consumer: $ \max_{C, l} \ln C + \ln l \quad \text{subject to} \quad C \le w( 1-l ) + \pi $
    %
    \begin{align}
        \text{FOC} \quad
            & \frac{C}{(\textbf{\red{\underline{\quad $1-N$ \quad}}})} = w
            \label{eq:consumerFOC}
        \\
        \text{Binding budget constraint} \quad
            & C = w (\textbf{\red{\underline{\quad N \quad}}}) + \pi
            \label{eq:binding_budget}
    \end{align}
    %

    Firm: $ \max_{N} ( N )^{\frac{1}{2}} - w N $
    %
    \begin{align}
        \text{FOC} \quad
            & \frac{1}{2} ( N )^{- \frac{1}{2}} = w
            \label{eq:firmFOC}
        \\
        \text{Profit definition} \quad
            & \pi = (\textbf{\red{\underline{\quad $N^{\frac{1}{2}}$ \quad}}}) - w N
            \label{eq:profitDef}
    \end{align}
    %
    \item Step 4: Substitute $ \pi( N ) $ into Binding budget constraint and get
    %
    \begin{equation}
    \label{eq:C_as_function_of_N}
        C = (\textbf{\red{\underline{\quad $N^{\frac{1}{2}}$ \quad}}})
    \end{equation}
    %
    \item Step 5: With consumer's FOC and firm's FOC both equate to $ w $, we can get another expression of $ C $:
    %
    \begin{equation}
    \label{eq:C_as_function_of_N_ver_2}
        C = (\textbf{\red{\underline{\quad $1-N$ \quad}}}) \times  (\textbf{\red{\underline{\quad $\frac{1}{2} N^{-\frac{1}{2}}$ \quad}}})
    \end{equation}
    %
    \item Step 6: Let \eqref{eq:C_as_function_of_N} equate \eqref{eq:C_as_function_of_N_ver_2} and we get $ N $ as
    %
    \begin{equation}
    \label{eq:Nvalue}
        N = (\textbf{\red{\underline{\quad $\frac{1}{3}$ \quad}}})
    \end{equation}
    %
    \item Step 7: Trace back to all unknowns given the value of $ N $, we get
    %
    \begin{align}
        C
            & = (\textbf{\red{\underline{\quad $\sqrt{ \frac{1}{3}} $ \quad}}}) (0.577)
        \\
        l
            & = (\textbf{\red{\underline{\quad $\frac{2}{3} $ \quad}}}) (0.666)
        \\
        Y
            & = (\textbf{\red{\underline{\quad $ \sqrt{ \frac{1}{3}}  $ \quad}}}) (0.577)
        \\
        \pi
            & = (\textbf{\red{\underline{\quad $\sqrt{\frac{1}{3}} - \frac{1}{6} \sqrt{3} $ \quad}}}) (0.288)
        \\
        w
            & = (\textbf{\red{\underline{\quad $\frac{1}{2} \sqrt{3} $ \quad}}}) (0.866)
    \end{align}
    %


\end{enumerate}


\end{document}

