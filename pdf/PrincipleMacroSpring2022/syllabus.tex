\documentclass[12pt]{article}

\usepackage[style=authoryear,maxbibnames=9,maxcitenames=2,uniquelist=false,backend=biber,doi=false,url=false]{biblatex}
\addbibresource{$BIB} % bibtex location
\renewcommand*{\nameyeardelim}{\addcomma\space} % have comma in parencite

\usepackage{xcolor}
 \usepackage{amsmath}
\newcommand{\tuple}[1]{ \langle #1 \rangle }
%\usepackage{automata}
\usepackage{times}
\usepackage{ltablex}

%%%%%% Template
\usepackage{hyperref}
\hypersetup{colorlinks=true,allcolors=blue}

\usepackage{vmargin}
\setpapersize{USletter}
\setmarginsrb{1.0in}{1.0in}{1.0in}{0.6in}{0pt}{0pt}{0pt}{0.4in}

% HOW TO USE THE ABOVE:
%\setmarginsrb{leftmargin}{topmargin}{rightmargin}{bottommargin}{headheight}{headsep}{footheight}{footskip}
%\raggedbottom
% paragraphs indent & skip:
\parindent  0.3cm
\parskip    -0.01cm

\usepackage{tikz}
\usetikzlibrary{backgrounds}

% hyphenation:
\sloppy

% notes-style paragraph spacing and indentation:
\usepackage{parskip}
\setlength{\parindent}{0cm}

% let derivations break across pages
\allowdisplaybreaks

\def\blue{\color{blue}}
\def\orange{\color{orange}}

\def\qqquad{\quad\qquad}
\def\qqqquad{\qquad\qquad}

%%%%%%%%%%%%%%%%%%%%%%%%%%%%%%%%%%%%%%%%%%%%%%%%%%%%%%%%%%%%%%%%%%%%%%%%%%%%%%%%
%%%%%%%%%%%%%%%%%%%%%%%%%%%%%%%%%%%%%%%%%%%%%%%%%%%%%%%%%%%%%%%%%%%%%%%%%%%%%%%%
\begin{document}

\centerline{\huge\bf Syllabus: ECON 2002.01-203 (19859)}
\medskip
\centerline{\LARGE \bf Principle of Macroeconomics}
\medskip
\centerline{\LARGE \bf Spring 2022}
\medskip
\centerline{\Large Instructor: Hui-Jun Chen}

\medskip

\section*{Course Overview}
\begin{itemize}

    \item Course website:
    \begin{itemize}
        \item Materials: \href{https://huijunchen9260.github.io/PrincipleMacroSpring2022.html}{Webpage}
        \item Quizzes / Exam: \href{https://osu.instructure.com/courses/114824}{Carmen}
        \item Textbook and Textbook Website: \href{https://www.core-econ.org/}{Core-econ}, \href{https://www.core-econ.org/the-economy/book/text/0-3-contents.html}{The Economy}
        \begin{itemize}
            \item You can use the student resources in the core-econ website as the tool for evaluating your understanding about the course material.
        \end{itemize}
    \end{itemize}
    \item Meeting Time: Monday, Wednesday, Friday, 8:00AM - 8:55AM
    \item Location: McPherson Lab 1015
    \item Class Dates: Jan 10, 2022-Apr 25, 2022
    \item Email address: \href{chen.9260@buckeyemail.osu.edu}{chen.9260@buckeyemail.osu.edu}.
    \item To get email reply, you \textbf{must} satisfy two conditions below, :
    \begin{enumerate}
        \item DO \textbf{NOT} SEND TO CARMEN EMAIL.
        \item Use \texttt{[E2002.01]} at the beginning of your subject title.
        \begin{itemize}
            \item example title: \texttt{[E2002.01] Question regarding Extra credit}
        \end{itemize}
    \end{enumerate}
    \item I will reply your email within \textit{2 business day}.
    \item Office hour: TBH
    \item Principles of macroeconomics will cover the following general topics: measures of national well\-being, macroeconomic models, economic growth, monetary and fiscal policy.
\end{itemize}

\newpage

\section*{Grades}

\newlength\q
\setlength\q{\dimexpr .5\textwidth -2\tabcolsep}
\begin{tabular}{|p{\q}|p{\q}|}
    \hline
    Categories  & Points \\
    \hline
    \hline
    Qizzes on course material   & 20 points \\
    \hline
    Qizzes on Calculus videos & 20 points \\
    \hline
    Midterm Exam & 30 points \\
    \hline
    Final Exam & 30 points \\
    \hline
    SEI Extra credit & 2 points \\
    \hline
    Total & 102 points \\
    \hline
\end{tabular}
\textit{See course schedule, below, for due dates}


\section*{Grading Policy}

I put a lot of weight on your performance during the semester, not just the exams.
This course wishes you not only learn Economics, but also understand some of the concept in Calculus.
To perform well in the course, in addition to doing well in the exams, you need to do two more things: Weekly quiz on Carmen, both for Economics materials and Calculus materials.

\subsection*{Extra credits}

Extra credits: At the end of the semester, I will have you do the Student Evaluation of Instruction (SEI). If I get $85\%$ response rate on SEI by the end of the semester, everyone will get 2 points of extra credits.

\subsection*{Curving}

If less than $40\%$ of the students get A- or above, I will add some points to everybody until $40\%$ of the students get A- or above, but I don’t expect this to occur.

\subsection*{Quiz Due}

All quzzis are due on the date specified on Carmen; late quizzes are not accepted, unless you have formal excuse ( require formal documentation ).

\newpage

\section*{Course Schedule}

\newlength\qq
\setlength\qq{\dimexpr .2\textwidth -2\tabcolsep}
\newlength\pp
\setlength\pp{\dimexpr .8\textwidth -2\tabcolsep}
\begin{tabular}{|p{\qq}|p{\pp}|}
    \hline
        Course Week  & Topics, Readings and Deadlines \\
    \hline
    \hline
        01/10
        \newline
        01/12
        \newline
        01/14     &
        Topic: Review of Mathematics
        \newline
        Reading: Math Supplement
        \newline
        Deadline: 01/16 11:59pm
    \\
    \hline
        01/19
        \newline
        01/21     &
        Topic: Introduction
        \newline
        Reading: Unit 1 and 2
        \newline
        Deadline: 01/23 11:59pm
    \\
    \hline
        01/24
        \newline
        01/26
        \newline
        01/28   &
        Topic: Labor markets and unemployment
        \newline
        Reading: Unit 6 and Unit 7 (sec. on wage-setting)
        \newline
        Deadline: 01/30 11:59pm
    \\
    \hline
        01/31
        \newline
        02/02
        \newline
        02/04   &
        Topic: Unemployment and Credit Market
        \newline
        Reading: Unit 9, and Unit 10 sec. 10.8
        \newline
        Deadline: 02/06 11:59pm
    \\
    \hline
        02/07
        \newline
        02/09
        \newline
        02/11   &
        Topic: Credit Market
        \newline
        Reading: Unit 3
        \newline
        Deadline: 02/13 11:59pm
    \\
    \hline
        02/14
        \newline
        02/16
        \newline
        02/18   &
        Topic: Economic accounting and fluctuations
        \newline
        Reading: Unit 13
        \newline
        Deadline: 02/20 11:59pm
    \\
    \hline
        02/21, 23, 25   &
        Week for adjustment
    \\
    \hline
        02/28   &
        Midterm Exam
    \\
    \hline
        03/07
        \newline
        03/09
        \newline
        03/11   &
        Topic: Fiscal Policy
        \newline
        Reading: Unit 14
        \newline
        Deadline: 03/13 11:59pm
    \\
    \hline
        03/21
        \newline
        03/23
        \newline
        03/25   &
        Topic: Monetary Policy
        \newline
        Reading: Unit 15
        \newline
        Deadline: 03/27 11:59pm
    \\
    \hline
        03/28
        \newline
        03/30
        \newline
        04/01   &
        Topic: Long-run economic performance
        \newline
        Reading: Unit 16
        \newline
        Deadline: 04/03 11:59pm
    \\
    \hline
        04/04
        \newline
        04/06
        \newline
        04/08   &
        Topic: Application of the models: The Great Depression
        \newline
        Reading: Unit 17
        \newline
        Deadline: 04/10 11:59pm
    \\
    \hline
        04/11
        \newline
        04/13
        \newline
        04/15   &
        Topic: Application of the models: Asset price bubbles
        \newline
        Reading: Unit 11
        \newline
        Deadline: 04/17 11:59pm
    \\
    \hline
        04/18, 20, 22   &
        Week for adjustment
    \\
    \hline
        04/27 &
        Final Exam
    \\
    \hline
\end{tabular}



\newpage

\section*{Course learning outcomes}

\textbf{This course fulfills the GE Goals and Expected Learning Outcomes for Social Science: Organizations and Polities.}

\subsection*{Social Science Goal}

Students understand the systematic study of human behavior and cognition; the structure of human societies, cultures, and institutions; and the processes by which individuals, groups, and societies interact, communicate, and use human, natural, and economic resources.

\subsection*{Organizations and Polities Expected Learning Outcomes}
\begin{enumerate}
    \item Students understand the theories and methods of social scientific inquiry as they apply to the study of organizations and polities.
    \item Students understand the formation and durability of political, economic, and social organizing principles and their differences and similarities across contexts.
    \item Students comprehend and assess the nature and values of organizations and polities and their importance in social problem solving and policy making.
\end{enumerate}

Economics 2002.01 addresses the theories and methods of social scientific inquiry through discussion of supply and demand at the national level, and the measurement of national income and other macroeconomic measures, along with applications to current events.

Students will learn about the formation and durability of political, economic, and social organizing principles through discussions of the origin and structure of central banks as well as other international organizations, and fiscal and monetary policy. These topics will include discussion of various commonly accepted points of view.

Students will comprehend and assess the nature and values of organizations and polities and their importance in social problem solving and policy making through discussion of fiscal and monetary policy, business cycles and the Federal Reserve Bank, including its values and objectives.



\newpage

\section*{Ohio State’s academic integrity policy}

Academic integrity is essential to maintaining an environment that fosters excellence in teaching, research, and other educational and scholarly activities.
Thus, The Ohio State University and the Committee on Academic Misconduct (COAM) expect that all students have read and understand the University’s Code of Student Conduct, and that all students will complete all academic and scholarly assignments with fairness and honesty.
Students must recognize that failure to follow the rules and guidelines established in the University’s Code of Student Conduct and this syllabus may constitute ``Academic Misconduct.''

The Ohio State University’s Code of Student Conduct (Section 3335-23-04) defines academic misconduct as: ``Any activity that tends to compromise the academic integrity of the University, or subvert the educational process.''
Examples of academic misconduct include (but are not limited to) plagiarism, collusion (unauthorized collaboration), copying the work of another student, and possession of unauthorized materials during an examination.
Ignorance of the University’s Code of Student Conduct is never considered an ``excuse'' for academic misconduct, so I recommend that you review the Code of Student Conduct and, specifically, the sections dealing with academic misconduct.

\textbf{If I suspect that a student has committed academic misconduct in this course, I am obligated by University Rules to report my suspicions to the Committee on Academic Misconduct.} 
If COAM determines that you have violated the University’s Code of Student Conduct (i.e., committed academic misconduct), the sanctions for the misconduct could include a failing grade in this course and suspension or dismissal from the University.

If you have any questions about the above policy or what constitutes academic misconduct in this course, please contact me.

Other sources of information on academic misconduct (integrity) to which you can refer include:
\begin{itemize}
    \item The Committee on Academic Misconduct web pages (COAM Home)
    \item Ten Suggestions for Preserving Academic Integrity (Ten Suggestions)
    \item Eight Cardinal Rules of Academic Integrity (www.northwestern.edu/uacc/8cards.htm)
\end{itemize}

Violating university or course rules as contained in the course syllabus or other information provided to the student in regard to student classroom conduct may result in your being removed from the class rolls.

\subsection*{Quizzes and Examinations Integrity Policies}

\textbf{Quizzes}: Discussions are encouraged, but each person must hand in their own quiz on Carmen.

\textbf{Examinations}: Dicussions are forbidden, either face to face or via online discussion board / Social media. You are allowed to refer to course slides, core-econ textbook, lecture videos.

\subsection*{Other Policies}

Students with disabilities that have been certified by the Office for Disability Services will be appropriately accommodated.
They should inform the instructor as soon as possible of their needs.
Students who feel that they need an accommodation based on the impact of a disability should contact the Office for Disability Service.
General information is available at http://www.ods.ohio-state.edu.

The core material contained within this syllabus will either be discussed in class or assigned as required reading.

If you decide not to complete the course, please formally withdraw from the class.
Failure to officially withdraw will result in an ``E'' on your transcript and you will have foregone the opportunity to receive a refund (partial or full).

You are expected to be on time to class.
In those events when you do arrive at class late, please find a seat as quietly and unobtrusively as possible.
Do not interrupt class to hand in assignments or request materials.
An opportunity will be provided for these activities at an appropriate time.

We will be doing in-class participation exercises that work via the internet.
Please be sure to bring a mobile device (laptop, tablet, or smartphone) with you to class each day.

\subsection*{Economics Learning Center}

Information can be found at https://economics.osu.edu/economics-learning-center.

\subsection*{Grading scale}
\begin{itemize}
    \item 93–100: A
    \item 90–92.9: A-
    \item 87–89.9: B+
    \item 83–86.9: B
    \item 80–82.9: B-
    \item 77–79.9: C+
    \item 73–76.9: C
    \item 70 –72.9: C-
    \item 67 –69.9: D+
    \item 60 –66.9: D
    \item Below 60: E
\end{itemize}







\printbibliography

\end{document}

