\documentclass[12pt]{article}

\usepackage[style=authoryear,maxbibnames=9,maxcitenames=2,uniquelist=false,backend=biber,doi=false,url=false]{biblatex}
\addbibresource{$BIB} % bibtex location
\renewcommand*{\nameyeardelim}{\addcomma\space} % have comma in parencite

\usepackage{xcolor}
 \usepackage{amsmath}
\newcommand{\tuple}[1]{ \langle #1 \rangle }
%\usepackage{automata}
\usepackage{times}
\usepackage{ltablex}

%%%%%% Template
\usepackage{hyperref}
\hypersetup{colorlinks=true,allcolors=blue}

\usepackage{vmargin}
\setpapersize{USletter}
\setmarginsrb{1.0in}{1.0in}{1.0in}{0.6in}{0pt}{0pt}{0pt}{0.4in}

% HOW TO USE THE ABOVE:
%\setmarginsrb{leftmargin}{topmargin}{rightmargin}{bottommargin}{headheight}{headsep}{footheight}{footskip}
%\raggedbottom
% paragraphs indent & skip:
\parindent  0.3cm
\parskip    -0.01cm

\usepackage{tikz}
\usetikzlibrary{backgrounds}

% hyphenation:
\sloppy

% notes-style paragraph spacing and indentation:
\usepackage{parskip}
\setlength{\parindent}{0cm}

% let derivations break across pages
\allowdisplaybreaks

\def\blue{\color{blue}}
\def\orange{\color{orange}}

\def\qqquad{\quad\qquad}
\def\qqqquad{\qquad\qquad}

%%%%%%%%%%%%%%%%%%%%%%%%%%%%%%%%%%%%%%%%%%%%%%%%%%%%%%%%%%%%%%%%%%%%%%%%%%%%%%%%
%%%%%%%%%%%%%%%%%%%%%%%%%%%%%%%%%%%%%%%%%%%%%%%%%%%%%%%%%%%%%%%%%%%%%%%%%%%%%%%%
\begin{document}

\centerline{\huge\bf Syllabus: ECON 2002.01-203 (19859)}
\medskip
\centerline{\LARGE \bf Principle of Macroeconomics}
\medskip
\centerline{\LARGE \bf Spring 2022}
\medskip
\centerline{\Large Instructor: Hui-Jun Chen}

\medskip

\section*{Course Overview}
\begin{itemize}
    \item Meeting Time: Monday, Wednesday, Friday, 8:00AM - 8:55AM
    \item Location: McPherson Lab 1015
    \item Class Dates: Jan 10, 2022-Apr 25, 2022
    \item Email address: \href{chen.9260@buckeyemail.osu.edu}{chen.9260@buckeyemail.osu.edu}
    \begin{itemize}
        \item DO \textbf{NOT} SEND TO CARMEN EMAIL.
        \item I will reply your email within \textit{2 business day}.
        \item Use \texttt{E2002.01} in your subject title, or I cannot find your email.
    \end{itemize}
    \item Office hour:
    \item Principles of macroeconomics will cover the following general topics: measures of national well-being, macroeconomic models, economic growth, monetary and fiscal policy.
\end{itemize}

\section*{Course learning outcomes}

\textbf{This course fulfills the GE Goals and Expected Learning Outcomes for Social Science: Organizations and Polities.}

\subsection*{Social Science Goal}

Students understand the systematic study of human behavior and cognition; the structure of human societies, cultures, and institutions; and the processes by which individuals, groups, and societies interact, communicate, and use human, natural, and economic resources.

\subsection*{Organizations and Polities Expected Learning Outcomes}
\begin{enumerate}
    \item Students understand the theories and methods of social scientific inquiry as they apply to the study of organizations and polities.
    \item Students understand the formation and durability of political, economic, and social organizing principles and their differences and similarities across contexts.
    \item Students comprehend and assess the nature and values of organizations and polities and their importance in social problem solving and policy making.
\end{enumerate}

Economics 2002.01 addresses the theories and methods of social scientific inquiry through discussion of supply and demand at the national level, and the measurement of national income and other macroeconomic measures, along with applications to current events.

Students will learn about the formation and durability of political, economic, and social organizing principles through discussions of the origin and structure of central banks as well as other international organizations, and fiscal and monetary policy. These topics will include discussion of various commonly accepted points of view.

Students will comprehend and assess the nature and values of organizations and polities and their importance in social problem solving and policy making through discussion of fiscal and monetary policy, business cycles and the Federal Reserve Bank, including its values and objectives.

\section*{Course materials}

Please register as student on \href{https://www.core-econ.org/}{https://www.core-econ.org/}, and use the open source textbook \href{https://www.core-econ.org/the-economy/book/text/0-3-contents.html}{The Economy} as the major textbook for this course.
You can use the student resources in the core-econ website as the tool for evaluating your understanding about the course material.






\printbibliography

\end{document}

