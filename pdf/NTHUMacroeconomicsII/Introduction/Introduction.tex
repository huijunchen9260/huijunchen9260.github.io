%----------------------------------------------------------------------------------------
%    PACKAGES AND THEMES (template-compatible)
%----------------------------------------------------------------------------------------

\documentclass[aspectratio=169,xcolor=dvipsnames]{beamer}
\usetheme{SimpleDarkBlue}

\usepackage{hyperref}
\usepackage{graphicx}
\usepackage{booktabs}
\usepackage{appendixnumberbeamer}
\usepackage{amsmath}
\usepackage{iftex}

% --- Font handling: compile with pdfLaTeX here; switch to LuaLaTeX locally if desired ---
\ifPDFTeX
  \usepackage[T1]{fontenc}
  \usepackage{lmodern}
  \usepackage{microtype}
\else
  \usepackage{fontspec}
  \usepackage{luatexja}
  \setmainfont{Alegreya Sans Light}[
    ItalicFont={* Italic},
    BoldFont={Alegreya Sans Medium},
    BoldItalicFont={Alegreya Sans Medium Italic}]
  \setsansfont{Alegreya Sans Light}[
    ItalicFont={* Italic},
    BoldFont={Alegreya Sans Medium},
    BoldItalicFont={Alegreya Sans Medium Italic}]
\fi

% ------------ %
% beamerbutton %
% ------------ %
\newcommand{\goto}[2]{\hyperlink{#2}{\beamergotobutton{#1}}}
\newcommand{\return}[2]{\hyperlink{#2}{\beamerreturnbutton{#1}}}
\newcommand{\extgoto}[2]{\href{#2}{\beamergotobutton{#1}}}

% ------------------------------------ %
% Section title page with Huge bf text %
% ------------------------------------ %
\AtBeginSection[]{
  \begin{frame}[noframenumbering, plain]
    \Huge{\centerline{\textbf{\insertsection}}}
  \end{frame}
}

%%% automatically add spaces into enumerate and itemize environment
\let\tempone\itemize
\let\temptwo\enditemize
\renewenvironment{itemize}{\tempone\addtolength{\itemsep}{\fill}}{\temptwo}
\let\tempa\enumerate
\let\tempb\endenumerate
\renewenvironment{enumerate}{\tempa\addtolength{\itemsep}{\fill}}{\tempb}

%%=============================================================
%%  FOOTLINE TEMPLATE
%%=============================================================
\defbeamertemplate*{footline}{shadow theme}{%
    \leavevmode\hbox{%
        \hypersetup{linkcolor=white,urlcolor=white,citecolor=white}
        \begin{beamercolorbox}[wd=1.02\paperwidth, ht=2.5ex, dp=1.125ex, leftskip=.3cm plus1fil, rightskip=0.3cm]{author in head/foot}%
            \insertsectionnavigationhorizontal{.80\textwidth}{}{} \hspace{.3cm} \hfill \#\ \insertframenumber \ / \ \inserttotalframenumber%
        \end{beamercolorbox}%
    }%
}
\setbeamertemplate{footline}[shadow theme]

%----------------------------------------------------------------------------------------
%    TITLE PAGE
%----------------------------------------------------------------------------------------

\title[Inflation \& the Price Level]{Intermediate Macroeconomics II\\\vspace{4pt}Inflation \& the Price Level: Motivation}
\author[Hui-Jun Chen]{Hui-Jun Chen}
\institute[]{National Tsing Hua University}
\date{Spring 2026}

\graphicspath{{figures/}}

%----------------------------------------------------------------------------------------
%    PRESENTATION SLIDES
%----------------------------------------------------------------------------------------

\begin{document}

%------------------------------------------------
\begin{frame}[plain,noframenumbering]
    \titlepage
\end{frame}

%------------------------------------------------
\begin{frame}{Acknowledgement and source}
\begin{block}{Where today\'s motivation comes from}
This introduction is adapted from \textbf{John H. Cochrane}, \emph{Inflation} (Karl Brunner Distinguished Lecture; February 10, 2026), especially \textbf{Chapter 1 (Introduction)}, \textbf{Chapter 2 (The 2021--2022 inflation)}, and \textbf{Chapter 3 (Monetary Policy)}.
\end{block}

\begin{itemize}
\item Figures shown are reproduced from Cochrane\'s lecture PDF (data series are primarily from FRED).
\item Goal here is \textbf{pedagogical}: motivate what theories of inflation must explain, before we build models.
\item Any remaining mistakes/interpretations are mine.
\end{itemize}
\end{frame}

%------------------------------------------------
\section{Motivation}

%------------------------------------------------
\begin{frame}{What this course is about}
\begin{itemize}
\item \textbf{Price level} $P_t$: the value of money (how many dollars buy one unit of goods).
\item \textbf{Inflation} $\pi_t$: the rate of change in the price level.
\item These are \textbf{macroeconomic prices}: they shape wages, contracts, taxes, debt burdens, and policy.
\item A central challenge: \textbf{real macro} can explain quantities, but we also need a theory for \textbf{nominal units}.
\end{itemize}

\begin{block}{Core aim}
Build models that explain \textbf{(i) why inflation happens} and \textbf{(ii) what anchors the level of prices}.
\end{block}
\end{frame}

%------------------------------------------------
\begin{frame}{Three questions to keep asking all semester}
\begin{enumerate}
\item \textbf{Nominal anchor:} What pins down the price level $P_t$?
\begin{itemize}
\item Examples: money-growth rules, a gold standard/backing, a Taylor rule, or fiscal backing of debt.
\end{itemize}
\item \textbf{Dynamics:} What makes inflation persist or fade? \\
\hspace{1em} (expectations, frictions, Phillips curve, credibility)
\item \textbf{Policy regime:} What do monetary and fiscal authorities \emph{commit} to do when shocks hit?
\begin{itemize}
\item ``Monetary dominance'' vs ``fiscal dominance'' is a regime statement.
\end{itemize}
\end{enumerate}

\begin{block}{A theme}
Many disagreements about inflation are really disagreements about the \textbf{policy regime} and the \textbf{anchor}.
\end{block}
\end{frame}

%------------------------------------------------
\begin{frame}{The standard policy doctrine (in words)}
\small
\begin{itemize}
\item Central banks often describe a \textbf{monetary transmission mechanism}:
\begin{itemize}
\item Raise interest rates $\Rightarrow$ spending slows $\Rightarrow$ unemployment rises $\Rightarrow$ inflation falls.
\item Lower interest rates $\Rightarrow$ demand expands $\Rightarrow$ inflation rises.
\end{itemize}
\item Cochrane\'s complaint: this doctrine is \textbf{useful operationally}, but it is not yet a complete \textbf{economic theory}.
\item Two missing pieces show up immediately:
\begin{itemize}
\item \textbf{Expectations:} what determines expected inflation and expected future policy?
\item \textbf{Price-level determination:} why doesn\'t $P_t$ drift under an interest-rate target?
\end{itemize}
\item Even the Fisher equation warns us: higher nominal rates can also reflect \textbf{higher expected inflation}.
\end{itemize}

\normalsize
\begin{block}{Takeaway}
We want a framework that makes the doctrine \textbf{precise}, \textbf{internally consistent}, and \textbf{testable}.
\end{block}
\end{frame}

%------------------------------------------------
\begin{frame}{Why the standard story is under strain: institutions}
\small
\begin{itemize}
\item Modern monetary systems are not the textbook world of money-supply control:
\begin{itemize}
\item \textbf{Interest-rate targets} are the main instrument.
\item \textbf{Ample reserves} pay market interest (``floor'' systems); reserves are not scarce.
\item Broad money and deposits are largely \textbf{endogenous} (created by private credit).
\end{itemize}
\item ``QE'' changes the \textbf{composition} of government liabilities (reserves vs bonds), not necessarily ``money'' in the old sense.
\item Cochrane\'s bar: a theory must fit current institutions: \textbf{fiat money, interest-rate targets, ample reserves, no money-supply control}.
\end{itemize}

\normalsize
\begin{block}{Takeaway}
A credible theory must match \textbf{current institutions}, not only historical ones.
\end{block}
\end{frame}

%------------------------------------------------
\begin{frame}{Why the standard story is under strain: theory}
\small
\begin{itemize}
\item Cochrane argues we are in a moment of \textbf{uncertainty} in monetary economics:
\begin{itemize}
\item The verbal doctrine has changed little since the 1970s.
\item Yet standard models struggle when pushed hard (especially around expectations and regime).
\end{itemize}
\item New Keynesian models make the doctrine formal, but rely on:
\begin{itemize}
\item a \textbf{specific policy rule} (e.g., Taylor principle) for determinacy,
\item a \textbf{mechanism for expectation formation and credibility},
\item a clean treatment of fiscal policy and government debt.
\end{itemize}
\item The 2021--2022 episode is a stress test:
\begin{itemize}
\item Inflation rose quickly while policy rates stayed near zero.
\item Inflation later fell substantially without a catastrophic recession.
\end{itemize}
\end{itemize}

\normalsize
\begin{block}{Takeaway}
We will study inflation like economists: \textbf{start from episodes}, then discipline theories.
\end{block}
\end{frame}

%------------------------------------------------
\section[2021-2022]{A recent episode: 2021--2022}

%------------------------------------------------
\begin{frame}{The timeline (US): recession, zero rates, then inflation}
\begin{columns}
  \begin{column}{0.62\textwidth}
    \centering
    \includegraphics[width=\linewidth]{fig2_1.png}
  \end{column}
  \begin{column}{0.38\textwidth}
    \small
    \begin{itemize}
    \item 2020: COVID recession; inflation eases; Fed funds rate near zero.
    \item Late Jan 2021: new administration; economy rebounds; inflation takes off.
    \item 2022: sharp rate hikes come \emph{after} inflation is already high.
    \item Disinflation later arrives with a slowdown, but not a deep 1980s-style recession.
    \end{itemize}

    \vspace{0.5em}
    \begin{block}{First puzzle}
    Inflation surged \textbf{before} interest rates rose.
    \end{block}
  \end{column}
\end{columns}
\end{frame}

%------------------------------------------------
\begin{frame}{Facts and puzzles to explain}
\small
\begin{itemize}
\item \textbf{What caused inflation to surge?}
\begin{itemize}
\item Demand boom, supply disruptions, energy shocks, markups, expectations,\ldots?
\end{itemize}
\item \textbf{Timing:} why early 2021 and not 2020, when policy was already extremely expansionary?
\item \textbf{Persistence:} why did inflation remain above target for so long?
\item \textbf{Disinflation:} why did inflation later decline without a deep recession?
\item \textbf{Policy interpretation:} if rates were near zero at the start, what was the \emph{nominal anchor}?
\end{itemize}

\normalsize
\begin{block}{Takeaway}
Good theories must match the \textbf{timing} and the \textbf{comovement} of inflation, rates, output, and policy.
\end{block}
\end{frame}

%------------------------------------------------
\begin{frame}{The ``elephant in the room'': fiscal expansion}
\begin{columns}
  \begin{column}{0.62\textwidth}
    \centering
    \includegraphics[width=\linewidth]{fig2_2.png}
  \end{column}
  \begin{column}{0.38\textwidth}
    \small
    \begin{itemize}
    \item 2020--2021: very large deficits and transfers.
    \item The key question is not only \emph{how big} deficits are, but \textbf{how they are expected to be paid for}.
    \item Fiscal expansions can work through:
    \begin{itemize}
      \item \textbf{Demand} (Keynesian spending/transfer channel), and
      \item \textbf{Valuation} (news about backing of nominal liabilities).
    \end{itemize}
    \end{itemize}
  \end{column}
\end{columns}
\begin{center}
    Are deficits a \textbf{demand boom}, a \textbf{debt-valuation event}, or both?
\end{center}
    % \begin{block}{Competing interpretations}
    % \small
    % \end{block}
\end{frame}

%------------------------------------------------
\begin{frame}{Why ``deficits cause inflation'' is not enough}
\small
\begin{itemize}
\item History shows: some large deficits are followed by inflation; others are not.
\item Cochrane\'s emphasis: the difference is often \textbf{expectations about the future fiscal path}.
\begin{itemize}
\item Do people expect future taxes/spending adjustments (future primary surpluses)?
\item Or do they expect persistent deficits and debt rollover without credible backing?
\end{itemize}
\item This shifts attention from current deficits to the \textbf{entire expected present value of surpluses}.
\item Put differently: inflation is about \textbf{news} and \textbf{regime}, not only today\'s deficit number.
\end{itemize}

\normalsize
\begin{block}{Takeaway}
To explain inflation we must explain how \textbf{nominal government liabilities} are valued.
\end{block}
\end{frame}

%------------------------------------------------
\section[Fiscal Theory]{Fiscal theory: the price level as debt valuation}

%------------------------------------------------
\begin{frame}{The key equation (government debt valuation)}
\small
\begin{block}{Debt valuation / fiscal theory relationship}
\[
\frac{B_{t-1}}{P_t}
= \mathbb{E}_t \sum_{j=0}^{\infty} \beta^{j} \, s_{t+j}
\]
\end{block}

\small
\begin{itemize}
\item $B_{t-1}$: outstanding \textbf{nominal} government liabilities (money + bonds).
\item $P_t$: price level, so $B_{t-1}/P_t$ is their \textbf{real value}.
\item $s_{t+j}$: future \textbf{primary surpluses} (taxes minus non-interest spending) that \emph{back} liabilities.
\item Interpretation: an \textbf{asset-pricing equation} (real value = present value of backing).
\item A warning from Chapter 3: a present value equation is \emph{not} a full model by itself---we still need a regime (what policy does when shocks hit).
\end{itemize}

\normalsize
\begin{block}{Interpretation}
Like an asset price: \textbf{real value = present value of backing}.
\end{block}
\end{frame}

%------------------------------------------------
\begin{frame}{Mechanism in words: how fiscal news moves prices}
\small
\begin{enumerate}
\item Government issues nominal liabilities (money/bonds) today.
\item Households value them based on expected future \textbf{real backing} (surpluses).
\item If expected backing \textbf{falls}, the real value must fall.
\end{enumerate}

\vspace{0.3em}
\begin{block}{How can the real value fall?}
With nominal $B$ given in the short run, adjustment can occur via a \textbf{higher $P_t$}.
\end{block}

\small
\begin{itemize}
\item In a frictionless model, $P_t$ can \textbf{jump} instantly (a one-time price-level revaluation).
\item With sticky prices, the same revaluation can show up as \textbf{persistent inflation} over time.
\item This helps explain why inflation may rise even if policy rates do not move immediately.
\end{itemize}
\end{frame}

%------------------------------------------------
\begin{frame}{Three fiscal paths: why expectations matter}
\centering
\includegraphics[width=0.95\linewidth]{fig2_3.png}

\vspace{0.4em}
\small
\begin{itemize}
\item \textbf{Temporary deficit} + credible future surpluses: little inflation pressure.
\item \textbf{Deficit without backing}: inflation/price-level jump reduces the real value of liabilities.
\item \textbf{Persistent expected deficits}: inflation can remain elevated until fiscal news improves.
\item Cochrane\'s framing: the key object is \textbf{news about the present value of surpluses}.
\end{itemize}

\begin{block}{Takeaway}
Inflation is about the \textbf{expected fiscal/monetary regime}, not only current spending.
\end{block}
\end{frame}

%------------------------------------------------
\begin{frame}{What this approach adds (relative to slogans)}
\small
\begin{itemize}
\item It separates two ideas often mixed together:
\begin{itemize}
\item \textbf{Aggregate demand} effects of fiscal stimulus (Keynesian channel).
\item \textbf{Valuation/anchor} effects through backing of nominal liabilities (fiscal theory channel).
\end{itemize}
\item It makes a clean prediction: inflation responds to \textbf{news} about future surpluses/deficits.
\item It highlights an empirical task: measuring \textbf{expectations} and detecting \textbf{regime changes}.
\item It also reframes monetary policy (Chapter 3): rate changes work differently depending on\
\hspace{1em} debt maturity, price stickiness, and whether fiscal policy adjusts.
\end{itemize}

\normalsize
\begin{block}{Takeaway}
We need models that talk about \textbf{expectations, policy rules, and debt} in one framework.
\end{block}
\end{frame}

%------------------------------------------------
\section[Monetary Policy]{Monetary policy}

%------------------------------------------------
\begin{frame}{A monetary-policy puzzle: interest rates and inflation}
\small
\begin{itemize}
\item The doctrine says: \textbf{raise rates to lower inflation}.
\item But basic accounting points the other way:
\begin{itemize}
\item Higher interest rates raise interest costs on government debt.
\item With one-period debt and no fiscal adjustment, higher nominal rates often imply \textbf{higher expected inflation} (a Fisher effect).
\end{itemize}
\item Cochrane\'s Chapter 3 asks a clean conceptual question:
\begin{itemize}
\item What can a central bank do \textbf{on its own}, holding fiscal surpluses fixed?
\item What requires \textbf{secondary fiscal channels} (recession-induced deficits/tightening)?
\end{itemize}
\item This matters because modern central banks cannot directly compel taxes or spending.
\end{itemize}

\normalsize
\begin{block}{Takeaway}
To understand ``rate hikes reduce inflation,'' we must specify \textbf{the regime and the debt structure}.
\end{block}
\end{frame}

%------------------------------------------------
\begin{frame}{Sticky prices and long-term debt: a channel for disinflation}
\small
\begin{itemize}
\item Chapter 3.1 adds two ingredients to the fiscal-valuation view:
\begin{itemize}
\item \textbf{Sticky prices} to avoid unrealistic instant price-level jumps.
\item \textbf{Long-term government debt} (not only one-period debt).
\end{itemize}
\item Key intuition (valuation):
\begin{itemize}
\item Higher expected future rates lower \textbf{today's long-term bond prices}.
\item If the present value of primary surpluses is unchanged, the real value of nominal liabilities must match.
\item Therefore, a fall in bond prices can require a \textbf{lower price level today} (disinflation on impact).
\end{itemize}
\item In these models, rate hikes can generate a \textbf{temporary} fall in inflation and output, while long-run Fisher forces may still differ.
\end{itemize}

\normalsize
\begin{block}{Takeaway}
A coherent story of monetary policy requires: \textbf{sticky prices + debt maturity + fiscal expectations}.
\end{block}
\end{frame}

%------------------------------------------------
\section[Roadmap]{Roadmap for the semester}

%------------------------------------------------
\begin{frame}{How we will study inflation and the price level}
\small
\begin{enumerate}
\item \textbf{Measurement and facts}
\begin{itemize}
\item CPI vs PCE; expectations; Fisher equation; stylized inflation episodes.
\end{itemize}
\item \textbf{IS--LM and AD--AS (benchmark)}
\begin{itemize}
\item A first short-run framework; what it gets right/wrong given modern institutions.
\end{itemize}
\item \textbf{New Keynesian model (workhorse)}
\begin{itemize}
\item Euler/IS curve, Phillips curve, Taylor rule; determinacy and credibility.
\end{itemize}
\item \textbf{Fiscal theory and regimes}
\begin{itemize}
\item Government budget constraint, debt valuation, monetary vs fiscal dominance.
\end{itemize}
\item \textbf{Interest rates and inflation (the hard question)}
\begin{itemize}
\item When do rate hikes disinflate? How do debt maturity and fiscal responses matter?
\end{itemize}
\end{enumerate}

\normalsize
\begin{block}{Promise}
By the end, you should be able to tell a coherent story about \textbf{episodes} like 2021--2022.
\end{block}
\end{frame}

%------------------------------------------------
\begin{frame}{How this connects to what you learned last semester}
\small
\begin{itemize}
\item RBC emphasized: real allocations, productivity shocks, and intertemporal prices.
\item This semester adds two ingredients (and keeps the RBC core):
\begin{itemize}
\item \textbf{Nominal frictions} (sticky prices/wages) so monetary policy matters in the short run.
\item \textbf{Nominal liabilities and fiscal backing} so the price level is anchored.
\end{itemize}
\item A useful mental map:
\begin{itemize}
\item \textbf{Flexible prices} determine the \emph{real} equilibrium (output, natural rate).
\item \textbf{Nominal anchor/regime} determines $P_t$ and inflation dynamics.
\end{itemize}
\end{itemize}

\normalsize
\begin{block}{Takeaway}
Think: \textbf{RBC core} + \textbf{nominal side} + \textbf{policy regime} $\Rightarrow$ inflation and $P_t$.
\end{block}
\end{frame}

%------------------------------------------------
\begin{frame}{Next time}
\large
\begin{itemize}
\item Define inflation vs. the price level; real vs. nominal variables.
\item Introduce basic identities and the Fisher equation.
\item A first benchmark model: IS--LM (and why we will later move beyond it).
\end{itemize}
\end{frame}

\end{document}
