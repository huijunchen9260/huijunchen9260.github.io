\documentclass[11pt,aspectratio=43,usenames,dvipsnames]{beamer}
\usepackage[utf8]{inputenc}
\usepackage{amsmath, amsfonts, amssymb, amsthm}
\usepackage[T1]{fontenc}
% mint: code chuck and syntax highlighting
%% outputdir should change according to pdf build directory
\usepackage[outputdir=build,cache=false]{minted}
\usepackage{lmodern}
\usepackage{xcolor}
\usepackage{setspace}
\usepackage{booktabs}
\usepackage{multirow}
\usepackage{graphicx}
\usepackage[mode=build]{standalone}
\usepackage{tikz}
% % \usetikzlibrary{decorations}
% \usetikzlibrary{decorations.pathreplacing, intersections}

% \usepackage{pgfplots}
% \usetikzlibrary{calc,positioning}
% \pgfplotsset{compat=newest, scale only axis, width = 10cm}
% \pgfplotsset{sciclean/.style={axis lines=left,
%         axis x line shift=0.5em,
%         axis y line shift=0.5em,
%         axis line style={-,very thin},
%         axis background/.style={draw,ultra thin,gray},
%         tick align=outside,
%         xtick distance=1,
%         ytick distance=1,
%         major tick length=2pt}}
\usepackage{ulem}
\usepackage{hyperref}
\usepackage{booktabs}
\usepackage{babel}
\usepackage{makecell}
\usepackage[para,online,flushleft]{threeparttable}
\usepackage{pdfpages}
\usepackage{tcolorbox}
\usepackage{bm}
\usepackage{appendixnumberbeamer}
\usepackage{natbib}
\usepackage{caption}
\captionsetup[figure]{labelformat=empty}% redefines the caption setup of the figures environment in the beamer class.
\usetheme[compress]{Boadilla}
\usecolortheme{default}
\useoutertheme{miniframes}
\usefonttheme[onlymath]{serif}
\usepackage{fontawesome}

\newcommand{\jump}[2]{\hyperlink{#1}{\beamerbutton{#2}}}
\newcommand{\extjump}[2]{\href{#1}{\beamerbutton{#2}}}
\newcommand{\orange}[1]{\textcolor{orange}{#1}}
\newcommand{\red}[1]{\textcolor{red}{#1}}
\newcommand{\blue}[1]{\textcolor{blue}{#1}}
\newcommand{\green}[1]{\textcolor{OliveGreen}{#1}}

\renewcommand{\square}{\scalebox{0.7}{$\blacksquare$ \hspace{0.5em}}}
\setbeamertemplate{itemize item}{\raisebox{0.1em}{\scalebox{0.7}{$\blacksquare$}}}
\setbeamertemplate{itemize subitem}[circle]
\setbeamertemplate{itemize subsubitem}{--}
\setbeamercolor{itemize item}{fg=black}
\setbeamercolor{itemize subitem}{fg=black}
\setbeamercolor{itemize subsubitem}{fg=black}
\setbeamercolor{item projected}{bg=darkgray,fg=white}
\definecolor{blue}{rgb}{0.2, 0.2, 0.7}
\setbeamercolor{alerted text}{fg=blue}
\setbeamertemplate{enumerate items}[circle]


\setbeamertemplate{headline}{}

%==========================================
\let\olditemize=\itemize
\let\endolditemize=\enditemize
\renewenvironment{itemize}{\olditemize \itemsep1em}{\endolditemize}
\let\oldenumerate=\enumerate
\let\endoldenumerate=\endenumerate
\renewenvironment{enumerate}{\oldenumerate \itemsep1em}{ \endoldenumerate}

\DeclareMathOperator*{\argmax}{\arg\!\max}
\DeclareMathOperator*{\E}{\mathbb{E}}
\DeclareMathOperator*{\var}{\rm Var}
\DeclareMathOperator*{\cov}{\rm Cov}

\theoremstyle{definition}
\newtheorem{assume}{Assumption}
\newtheorem{lem}{Lemma}
\newtheorem{proposition}{Proposition}
\newtheorem{thm}{Theorem}
\newtheorem{corol}{Corollary}

\AtBeginSection[]{
  \begin{frame}[noframenumbering]
  \vfill
  \centering
  \begin{beamercolorbox}[sep=8pt,center,shadow=true,rounded=true]{title}
    \usebeamerfont{title}\insertsection\par%
  \end{beamercolorbox}
  \vfill
  \end{frame}
}

\begin{document}
    \title[Unit 6]{Unit 6 \\ The Firm: \\ Owners, Managers, and Employees}
    \author[Hui-Jun Chen]{Hui-Jun Chen}
    \institute[OSU]{The Ohio State University}
    % \date{\today}
    \date{\today}
    \setbeamertemplate{navigation symbols}{}
    \setstretch{1.2}

%-------------------------------------------------------
{
%	\usebackgroundtemplate{\includegraphics[width=1\paperwidth]{../EveningSky_cropped_edit43_bright.jpg}}
    \begin{frame}
% \vspace{3em}
        \centering
%		{\footnotesize 	ECON 4002 Intermediate Macroeconomic Theory}
        \maketitle
% \vspace{-1.5em}
% \centering
% \includegraphics[width=0.55\linewidth]{Pictures/houses.jpeg}


    \end{frame}
}

% -------------------------------------------
\setbeamertemplate{headline}
{
\setbeamercolor{section in head/foot}{fg=black, bg=white}
\vskip1em \tiny \insertsectionnavigationhorizontal{1\paperwidth}{\hspace{0.50\paperwidth}}{}
}
%------------------------------------------

\section[Intro]{Introduction}
\label{sec:Introduction}


\begin{frame}{Introduction}
\label{slide:Introduction}
    \begin{center}
        How does the firm interacts internally and externally?
    \end{center}
    \begin{itemize}
        \item Firms are legal entity, yet still are \alert{composed by human}
        \begin{itemize}
            \item internal: Owner(s) v.s. Managers
            \item external: Employees (labor market), consumer (goods market)
        \end{itemize}
        \item Internal conflict: asymmetric information (e.g. \cite{Akerlof_1970_TQJE})
        \item External conflict: hidden action (Principal-Agent Problem)
        \item As before, wage is determined by $ MRS = MRT $
        \item Further reading: \blue{\extjump{https://tinyurl.com/ysdzy8k9}{Unit 6}}
    \end{itemize}

\end{frame}

\section[Internal]{Internal Structure of the Firm}
\label{sec:Internal_Structure_of_the_Firm}

\begin{frame}{Firm's Internal Structure}
\label{slide:Firm_s_Internal_Structure}
    \begin{definition}
        Firm is a business organization which (1) hires ppl, (2) buy inputs to produce good/services, and (3) set prices $ \ge  $ cost.
    \end{definition}
    \begin{columns}
        \begin{column}{0.5\textwidth}
        \only<1> {Black arrow downward: }
        \only<2> {Green arrow upward: }
        \begin{itemize}
            \only<1> { \item Owners: set long-term goal }
            \only<1> { \item Managers: implement owners' goal by assigning tasks }
            \only<1> { \item Workers: doing tasks }
            \only<2> { \item Owners: Receive profit as a result of management}
            \only<2> { \item Managers: payment not directly related to effort $ \Rightarrow  $ other's \$, riskier investment / lowering effort }
            \only<2> { \item Workers: salary not increasing with effort}
        \end{itemize}
        \end{column}
        \begin{column}{0.5\textwidth}
            \begin{figure}
                \centering
                \includegraphics[width=\textwidth]{./figures/firmStructure.png}
            \end{figure}

        \end{column}
    \end{columns}

\end{frame}

\begin{frame}{Align the Interests}
\label{slide:Align_the_Interests}

    \begin{itemize}
        \item Contracts are \alert{incomplete}: outcome depends on \alert{future/unknown} events, and hard to \alert{measure} effort
        \item Incomplete contracts are inevitable, since modern job are mostly \textbf{not able to measure output} and \textbf{works as a team}
        \item Ways to alleviate incomplete contract:
        \begin{enumerate}
            \item pay with company shares: company profit $ \uparrow  $, share price $ \uparrow  $
            \item piece rate pay: \$5 to assembly one toy (low-end job)
            \item monitoring
        \end{enumerate}
    \end{itemize}
\end{frame}

\section[Labor]{Labor Discipline Model}
\label{sec:Labor_Discipline_Model}

\begin{frame}{Why do workers work hard?}
\label{slide:Why_do_workers_work_hard_}
    Workers work hard while firms' cannot directly measure effort because
    \begin{enumerate}
        \item work ethic
        \item feelings of responsibility
        \item reciprocate a feeling of gratitude for good working conditions
        \item benefits for measurable output
        \item promotions
        \item \alert{fear of being fired}
    \end{enumerate}
    \ldots Rational thinking sometimes means negative thinking \faMehO
\end{frame}

\begin{frame}{Fear of being Fired}
\label{slide:Fear_of_Fired}
    \begin{itemize}
        \item \alert{Rent} in Economics: payment to the owner greater than the costs
        \item If workers being unemployed, they get unemployment benefits $ \Rightarrow  $ \textbf{reservation wage}
        \item \alert{Employment rent}: benefit from employment $ - $ disutility from work $ - $ reservation wage, includes
        \begin{itemize}
            \item lost income when searching
            \item cost to start a new job, e.g. relocation
            \item Loss of non-wage benefits
            \item Social costs (scarring effects, lost of company connections/skill)
        \end{itemize}
        \item \alert{Larger employment rent (higher wage)} $ \Rightarrow  $ larger cost of job loss $ \Rightarrow  $ workers work hard to reduce chance of getting fired
    \end{itemize}
\end{frame}

\begin{frame}{Employment Game}
\label{slide:Employment_Game}
    \begin{enumerate}
        \item Employer: choose maximum wage to keep worker work hard enough
        \begin{itemize}
            \item payoff: output $ - $ wage
        \end{itemize}
        \item Worker: choose minimum effort to keep him/herself from firing
        \begin{itemize}
            \item payoff: employment rent
        \end{itemize}
        \item Workers are the \textbf{supply side} in labor market: trade off are \textbf{MRT}
        \item Employers are the \textbf{demand side} in labor market: trade off are \textbf{MRS}
        \item \alert{Best response curve}:
        \begin{itemize}
            \item for workers: optimal amount of effort workers will exert for each wage offered
            \item for employers: optimal level of wage employers will offer for each targeted level of effort.
        \end{itemize}
    \end{enumerate}
\end{frame}

\begin{frame}{Best response curves}
\label{slide:Best_response_curves}
    \begin{columns}
        \begin{column}{0.5\textwidth}
            Employers: \alert{assume revenue doesn't change}, firms minimize cost to max profit

            $ \Rightarrow  $ find a \textbf{isocost} line that minimize wage spending
            \begin{figure}
                \centering
                \includegraphics[width=\textwidth]{./figures/MRSEmployer.png}
            \end{figure}
        \end{column}
        \begin{column}{0.5\textwidth}
            Workers: Feasible frontier for wage \& effort
            \begin{figure}
                \centering
                \includegraphics[width=\textwidth]{./figures/MRTWorkers.png}
            \end{figure}

        \end{column}
    \end{columns}

\end{frame}

\begin{frame}{Determining Wages}
\label{slide:Determining_Wages}
    Equilibrium is at MRS $ = $ MRT, efficiency wage $ = 12 >$ reservation wage
    \begin{figure}
        \centering
        \includegraphics[width=0.8\textwidth]{./figures/EquilibriumLaborMkt.png}
    \end{figure}

\end{frame}

\begin{frame}{Involuntary Unemployment}
\label{slide:Involuntary_Unemployment}
    \begin{definition}
        \textbf{Involuntary unemployment} is being out of work, but preferring to have a job at the wages/working conditions as other workers.
    \end{definition}
    \begin{columns}
        \begin{column}{0.5\textwidth}
            \begin{itemize}
                \item Must have involuntary unemployment in the labor discipline model!
                \begin{itemize}
                    \item $ \because $ ensure employment rent is high enough for workers to put in effort.
                \end{itemize}
                \item Foreshadowing: is higher unemployment benefit leads to higher/lower unemployment rate?
            \end{itemize}
        \end{column}
        \begin{column}{0.5\textwidth}
            \begin{figure}
                \centering
                \includestandalone[width=\linewidth]{./figures/BestResponse}
            \end{figure}
        \end{column}
    \end{columns}
\end{frame}


\section{Appendix}
\label{sec:Appendix}

\appendix
% -------------------------------------------
\setbeamertemplate{headline}
{
\setbeamercolor{section in head/foot}{fg=black, bg=white}
\vskip1em \tiny \insertsectionnavigationhorizontal{1\paperwidth}{\hspace{0.50\paperwidth}}{}
}
%------------------------------------------
% \begin{frame}\frametitle{}
% \begin{columns}
% \label{Appendix}
% \column{1\linewidth}
% \centering
% {\Large \alert{Appendix}}
% \end{columns}
% \end{frame}
%------------------------------------------
\begin{frame}[allowframebreaks]{References}
\footnotesize
\bibliographystyle{$BIB_STYLE}
\bibliography{$BIBFILE}
\end{frame}

\end{document}
