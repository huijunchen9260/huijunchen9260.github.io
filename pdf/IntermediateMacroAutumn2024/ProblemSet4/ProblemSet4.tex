\documentclass[14pt]{extarticle}
% \documentclass[14pt]{article}

% \usepackage[style=authoryear,maxbibnames=9,maxcitenames=2,uniquelist=false,backend=biber,doi=false,url=false]{biblatex}
% \addbibresource{$BIB} % bibtex location
% \renewcommand*{\nameyeardelim}{\addcomma\space} % have comma in parencite
\usepackage{natbib}

\usepackage{xcolor}
\usepackage{amsmath}
\newcommand{\tuple}[1]{ \langle #1 \rangle }
%\usepackage{automata}
\usepackage{times}
\usepackage{ltablex}
\usepackage{tasks}

%%%%%% Template
\usepackage{hyperref}
\hypersetup{colorlinks=true,allcolors=blue}

\usepackage{vmargin}
\setpapersize{USletter}
\setmarginsrb{1.0in}{1.0in}{1.0in}{0.6in}{0pt}{0pt}{0pt}{0.4in}

% HOW TO USE THE ABOVE:
%\setmarginsrb{leftmargin}{topmargin}{rightmargin}{bottommargin}{headheight}{headsep}{footheight}{footskip}
%\raggedbottom
% paragraphs indent & skip:
\parindent  0.3cm
\parskip    -0.01cm

\usepackage{tikz}
\usetikzlibrary{backgrounds}

% hyphenation:
% \hyphenpenalty=10000 % no hyphen
% \exhyphenpenalty=10000 % no hyphen
\sloppy

% notes-style paragraph spacing and indentation:
\usepackage{parskip}
\setlength{\parindent}{0cm}

% let derivations break across pages
\allowdisplaybreaks

\newcommand{\orange}[1]{\textcolor{orange}{#1}}
\newcommand{\blue}[1]{\textcolor{blue}{#1}}
\newcommand{\red}[1]{\textcolor{red}{#1}}
\newcommand{\freq}[1]{{\bf \sf F}(#1)}
\newcommand{\datafreq}[2]{{{\bf \sf F}_{#1}(#2)}}

\def\qqquad{\quad\qquad}
\def\qqqquad{\qquad\qquad}

%%%%%%%%%%%%%%%%%%%%%%%%%%%%%%%%%%%%%%%%%%%%%%%%%%%%%%%%%%%%%%%%%%%%%%%%%%%%%%%%
%%%%%%%%%%%%%%%%%%%%%%%%%%%%%%%%%%%%%%%%%%%%%%%%%%%%%%%%%%%%%%%%%%%%%%%%%%%%%%%%

% fill-in-blank question style, found in https://tex.stackexchange.com/a/505089

\usepackage{ifthen}
\usepackage{tocloft}
\usepackage{exercise}
% \usepackage{xcolor}

% Set the Show Answers Boolean
\newboolean{showAns}
\setboolean{showAns}{false}
\newcommand{\showAns}{\setboolean{showAns}{true}}

% The length of the Answer line
\newlength{\answerlength}
\newcommand{\anslen}[1]{\settowidth{\answerlength}{#1}}

% ans command that indicates space for an answer or shows the answer in red
\newcommand{\ans}[1]{\settowidth{\answerlength}{\hspace{2ex}#1\hspace{2ex}}%
    \ifthenelse{\boolean{showAns}}%
        {\textcolor{red}{\underline{\hspace{2ex}#1\hspace{2ex}}}}%
        {\underline{\hspace{\answerlength}}}}%

% Formatting how multiple choices Questions are formated.
\settasks{label=(\Alph*), label-width=30pt}


% Some commands for the Exercise Question package
\renewcommand{\QuestionNB}{\Large\protect\textcircled{\small\bfseries\arabic{Question}}\ }
\renewcommand{\ExerciseHeader}{} %no header
\renewcommand{\QuestionBefore}{3ex} %Space above each Q
\setlength{\QuestionIndent}{8pt} % Indent after Q number


% To create the list of answers with tocloft...
\newcommand{\listanswername}{Answers}
\newlistof[Question]{answer}{Answers}{\listanswername}

% Creates a TOC for Answers
\newcounter{prevQ}
\newcommand{\answer}[1]{\refstepcounter{answer}%
\ans{#1}%
\ifnum\theQuestion=\theprevQ%
        \addcontentsline{Answers}{answer}{\protect\numberline{}#1}% don't include the Q number
        \else%
        \addcontentsline{Answers}{answer}{\protect\numberline{\theQuestion}#1}%
        \setcounter{prevQ}{\value{Question}}%
        \fi%
        }%


%tocloft formatting listofanswers
\renewcommand{\cftAnswerstitlefont}{\bfseries\large}
\renewcommand{\cftanswerdotsep}{\cftnodots}
\cftpagenumbersoff{answer}
\addtolength{\cftanswernumwidth}{10pt}


%%%%%%%%%%%%%%%%%%%%%%%%%%%%%%%%%%%%%%%%%%%%%%%%%%%%%%%%%%%%%%%%%%%%%%%%%%%%%%%%
%%%%%%%%%%%%%%%%%%%%%%%%%%%%%%%%%%%%%%%%%%%%%%%%%%%%%%%%%%%%%%%%%%%%%%%%%%%%%%%%
\begin{document}

% \setcounter{section}{}
\centerline{\huge\bf ECON 4002.01 Problem Set 4}
\smallskip
\centerline{\LARGE Hui-Jun Chen}

\medskip

\showAns
% \listofanswer

\begin{Exercise}

\section*{Question 1}
\label{sec:Question_1}
\addcontentsline{toc}{section}{Question 1}

Consider a model that is \textbf{similar to} (not exactly the same!) Lecture 17 RBC model but with several differences:
\begin{enumerate}
    \item Now consumer values leisure in date $ 1 $. The lifetime utility function is given by
        %
        \begin{equation*}
            U(C, N, C', N') = \ln C + \ln (1-N) + \ln C' + \ln (1-N')
        .\end{equation*}
        %
\end{enumerate}

First, we start by defining the competitive equilibrium:

\Question
Given the exogenous quantities
\answer{A}
\begin{tasks}(2)
    \task $ \{G, G', z, z', K\} $
    \task $ \{G, G', z, z'\} $
    \task $ \{G, G'\} $
    \task $ \{z, z', K\} $
\end{tasks}

a competitive equilibrium is a set of

\Question consumer choices \answer{C}
\begin{tasks}(2)
    \task $ \{ C, C', N_{S}, S \} $
    \task $ \{ N_{S}, N'_{S}, l, l', S \} $
    \task $ \{ C, C', N_{S}, N'_{S}, l, l', S \} $
    \task $ \{ C, C', S \} $
\end{tasks}

\Question
firm choices \answer{B}
\begin{tasks}(2)
    \task $ \{ Y, Y', N_{D}, N'_{D}, I, K' \} $
    \task $ \{ Y, Y', \pi, \pi', N_{D}, N'_{D}, I, K' \} $
    \task $ \{ Y, Y', \pi, \pi', I, K' \} $
    \task $ \{ \pi, \pi', N_{D}, N'_{D}, I, K' \} $
\end{tasks}

\Question
government choices \answer{D}
\begin{tasks}(2)
    \task $ \{ G, G', T, T', B \} $
    \task $ \{ G, G', B \} $
    \task $ \{ G, G', T, T' \} $
    \task $ \{ T, T', B \} $
\end{tasks}

\Question and prices \answer{B}
\begin{tasks}(2)
    \task $ \{ w, w', q \} $
    \task $ \{ w, w', r \} $
    \task $ \{ q, q', r \} $
    \task $ \{ r, r', q \} $
\end{tasks}


such that

\begin{enumerate}
    \item
    \Question
        Taken \answer{A}
        \begin{tasks}(2)
            \task $ \{ w, w', r, \pi, \pi' \} $
            \task $ \{ w, w', r\} $
            \task $ \{ w, w', \pi, \pi' \} $
            \task $ \{ r, \pi, \pi' \} $
        \end{tasks}
    as given,
    \Question
    consumer chooses \answer{D}
        \begin{tasks}(2)
            \task $ \{ r', N_{S}, N'_{S} \} $
            \task $ \{ C', K, K' \} $
            \task $ \{ r', K, K' \} $
            \task $ \{ C', N_{S}, N'_{S} \} $
        \end{tasks}
     to solve
        %
        %
        \begin{equation*}
            \begin{split}
                \max_{C', N_{S}, N'_{S}}
                    & \ln \left(
                    w N_{S} + \pi - T +
                    \frac{w' N'_{S} + \pi' - T' - C'}{1+r}
                \right)
                \\
                    &  + \ln C' + \ln ( 1-N_{S} ) + \ln (1-N_{S}')
                \\
            \end{split}
        .\end{equation*}
        %
        where we can back out $ \{ C, S, l, l' \} $.
    \item
    \Question
    Taken \answer{B} as given,
    \begin{tasks}(2)
        \task $ \{ w, w', q \} $
        \task $ \{ w, w', r \} $
        \task $ \{ q, q', r \} $
        \task $ \{ r, r', q \} $
    \end{tasks}
    \Question
    firm chooses \answer{C}
    \begin{tasks}(2)
        \task $ \{ H_{D}, H'_{D}, K' \} $
        \task $ \{ N_{D}, N'_{D}, C' \} $
        \task $ \{ N_{D}, N'_{D}, K' \} $
        \task $ \{ \pi, \pi', K' \} $
    \end{tasks}
     to solve
     %
     \begin{equation*}
         \begin{split}
            \max_{N_{D}, N'_{D}, K'}
                 & z K^{\alpha} N_{D}^{1-\alpha} - w N_{D} - [ K' - ( 1-\delta ) K ] +
             \\
                & \frac{z' ( K' )^{\alpha}( N'_{D} )^{1-\alpha} - w' N'_{D} + ( 1-\delta ) K'}{1+r}
            \\
         \end{split}
     .\end{equation*}
     %
        where we can back out $ \{ Y, Y', \pi, \pi', I \} $.
    \item
    \Question Taxes and deficit satisfy \answer{B}
    \begin{tasks}(2)
        \task $T + \frac{T'}{1+q} = G + \frac{G'}{1+q}$
        \task $T + \frac{T'}{1+r} = G + \frac{G'}{1+r}$
        \task $T + \frac{T'}{1+w} = G + \frac{G'}{1+w}$
        \task $\pi + \frac{\pi'}{1+r} = G + \frac{G'}{1+r}$
    \end{tasks}
     and $ G - T = B $.
    \item All markets clear: (i) labor, $ N_{S} = N_{D} $ \& $ N'_{S} = N'_{D} $; (ii) goods, $ Y = C + G $ \& $Y' = C' + G' $; (iii) bonds at date 0, $ S = B $.
\end{enumerate}

After defining the competitive equilibrium, now we are going to solve this model.

Step 1: Labor market

\Question From the lecture, we know that the current marginal product of labor ($MPN$) will equal to current wage. $ MPN =  $ \answer{D}
\begin{tasks}(2)
    \task $ z' ( 1-\alpha ) \left( \frac{K}{ N_{D}} \right)^{\alpha} $
    \task $ z ( 1-\alpha ) \left( \frac{K'}{ N_{D}} \right)^{\alpha} $
    \task $ z' ( 1-\alpha ) \left( \frac{K'}{ N'_{D}} \right)^{\alpha} $
    \task $ z ( 1-\alpha ) \left( \frac{K}{ N_{D}} \right)^{\alpha} $
\end{tasks}

\Question \label{ND} and thus the current labor demand $ N_{D} $ given the wage $ w $ is \answer{C}
\begin{tasks}(2)
    \task $N_{D} = \left( \frac{z' ( 1-\alpha )}{w} \right)^{\frac{1}{\alpha}} K$
    \task $N_{D} = \left( \frac{z ( 1-\alpha )}{w'} \right)^{\frac{1}{\alpha}} K$
    \task $N_{D} = \left( \frac{z ( 1-\alpha )}{w} \right)^{\frac{1}{\alpha}} K$
    \task $N_{D} = \left( \frac{z' ( 1-\alpha )}{w'} \right)^{\frac{1}{\alpha}} K'$
\end{tasks}

\Question From the lecture, we know that the future marginal product of labor ($MPN'$) will equal to future wage. $ MPN' =  $ \answer{C}
\begin{tasks}(2)
    \task $ z' ( 1-\alpha ) \left( \frac{K}{ N_{D}} \right)^{\alpha} $
    \task $ z ( 1-\alpha ) \left( \frac{K'}{ N_{D}} \right)^{\alpha} $
    \task $ z' ( 1-\alpha ) \left( \frac{K'}{ N'_{D}} \right)^{\alpha} $
    \task $ z ( 1-\alpha ) \left( \frac{K}{ N_{D}} \right)^{\alpha} $
\end{tasks}

\Question and thus the future labor demand $ N'_{D} $ given the future wage $ w' $ is \answer{D}
\begin{tasks}(2)
    \task $N'_{D} = \left( \frac{z' ( 1-\alpha )}{w} \right)^{\frac{1}{\alpha}} K$
    \task $N'_{D} = \left( \frac{z ( 1-\alpha )}{w'} \right)^{\frac{1}{\alpha}} K$
    \task $N'_{D} = \left( \frac{z ( 1-\alpha )}{w} \right)^{\frac{1}{\alpha}} K$
    \task $N'_{D} = \left( \frac{z' ( 1-\alpha )}{w'} \right)^{\frac{1}{\alpha}} K'$
\end{tasks}

In the labor supply part, we know that the marginal rate of substitution between leisure and consumption $ MRS_{l, C} $ equals to the wage.
\Question  \label{MRSlC} $ MRS_{l, C} =  $ \answer{A}
\begin{tasks}(2)
    \task $\frac{C}{1-N_{S}}$
    \task $\frac{1-N_{S}}{C}$
    \task $\frac{N_{S}}{1-C}$
    \task $\frac{N_{S}'}{1-N_{S}}$
\end{tasks}


In the saving part, we know that the marginal rate of substitution between current and future consumption $ MRS_{C, C'} $ equals to the real interest rate $ (1+r) $
\Question $ MRS_{C, C'} =  $ \answer{C}
\begin{tasks}(2)
    \task $\frac{N_{S}'}{N_{S}}$
    \task $\frac{C}{C'}$
    \task $\frac{C'}{C}$
    \task $\frac{N_{S}}{N_{S}'}$
\end{tasks}

\Question \label{C'} Solve for $ C' $, we get \answer{B}
\begin{tasks}(2)
    \task $ C' = (1+r)N_{S} $
    \task $ C' = (1+r)C $
    \task $ C' = (1+r)C' $
    \task $ C' = (1+r)N_{S}' $
\end{tasks}

Start from now we denote the income that is not directly affected by consumer choice as $ x $ and $ x' $, similar to Lecture 17.

\Question \label{budgetConstraint} Substitute $ C' $ using your answer in \ref{C'} into the budget constraint and solve for $ C $, we get \answer{A}
\begin{tasks}(2)
    \task $ C = \frac{1}{2} \left(
        w N_{S} + x + \frac{x'}{1+r}
    \right)$
    \task $ C = \frac{1}{1+\beta} \left(
        w N_{S} + x + \frac{x'}{1+r}
    \right)$
    \task $ C = \frac{1}{1+\beta} \left(
        w N_{S} + C' + \frac{C'}{1+r}
    \right)$
    \task $ C = \frac{1}{2} \left(
        w N_{S} + N_{S}' + \frac{N_{S}'}{1+r}
    \right)$
\end{tasks}

\Question \label{NS} Substitute your answer of \ref{budgetConstraint} into your answer in \ref{MRSlC}, we can solve the labor supply $ N_{S} = $ \answer{D}

\begin{tasks}(2)
    \task $ \frac{1}{3} - \frac{2}{3w} \left(
        x + \frac{x'}{1+r}
    \right)$
    \task $ \frac{2}{3} - \frac{w}{3} \left(
        x + \frac{x'}{1+r}
    \right)$
    \task $ \frac{2}{5} - \frac{5}{3w} \left(
        x + \frac{x'}{1+r}
    \right)$
    \task $ \frac{2}{3} - \frac{1}{3w} \left(
        x + \frac{x'}{1+r}
    \right)$
\end{tasks}


\Question From \ref{ND} we solve for labor demand $ N_{D} $. From \ref{NS} we solve for labor supply $ N_{S} $. If for this question we let $ \alpha = 1 $, then we can solve the wage $ w $ as a function of real interest rate $ r $ as \answer{C}
\begin{tasks}(2)
    \task $w^{*}(r) = x + \frac{x'}{1+r} $
    \task $w^{*}(r) = \frac{1}{3} \left(
        x + \frac{x'}{1+r}
    \right)$
    \task $w^{*}(r) = \frac{1}{2} \left(
        x + \frac{x'}{1+r}
    \right)$
    \task $w^{*}(r) = zK\left(
        x + \frac{x'}{1+r}
    \right)$
\end{tasks}

For the output demand curve, we know that the optimal investment schedule is given by $ MPK' - \delta = r $.

\Question We know that the $ MPK' $ is \answer{B}
\begin{tasks}(2)
    \task $ \alpha z K^{\alpha-1} N^{1-\alpha} $
    \task $ \alpha z' K'^{\alpha-1} N'^{1-\alpha} $
    \task $ (1-\alpha) z' K'^{\alpha} N'^{-\alpha} $
    \task $ \alpha z K^{\alpha} N^{-\alpha} $
\end{tasks}

\Question
We can solve the optimal investment schedule and get $ K' = $ \answer{C}
\begin{tasks}(2)
    \task $ \left( \frac{z' \alpha}{q + \delta} \right)^{\frac{1}{1-\alpha}} N'$
    \task $ \left( \frac{z' \alpha}{r + \delta} \right)^{\frac{1}{1-\alpha}} N$
    \task $ \left( \frac{z' \alpha}{r + \delta} \right)^{\frac{1}{1-\alpha}} N'$
    \task $ \left( \frac{z' \alpha}{q + \delta} \right)^{\frac{1}{1-\alpha}} N$
\end{tasks}

\Question \label{ID} and the investment $ I_{D} $ is determined by capital accumulation process $ K' - (1-\delta)K $ and is \answer{D}
\begin{tasks}(2)
    \task $ \left( \frac{z' \alpha}{q + \delta} \right)^{\frac{1}{1-\alpha}} N' - (1-\delta)K$
    \task $ \left( \frac{z' \alpha}{r + \delta} \right)^{\frac{1}{1-\alpha}} N - (1-\delta)K$
    \task $ \left( \frac{z' \alpha}{q + \delta} \right)^{\frac{1}{1-\alpha}} N - (1-\delta)K$
    \task $ \left( \frac{z' \alpha}{r + \delta} \right)^{\frac{1}{1-\alpha}} N' - (1-\delta)K$
\end{tasks}

\Question Based on your answer in \ref{ID}, the investment demand $ I_{D} $ is \answer{A} in future labor $ N' $.
\begin{tasks}(3)
    \task increasing
    \task no related
    \task decreasing
\end{tasks}


\end{Exercise}




\end{document}
