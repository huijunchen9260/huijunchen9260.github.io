\documentclass[14pt]{extarticle}
% \documentclass[14pt]{article}

% \usepackage[style=authoryear,maxbibnames=9,maxcitenames=2,uniquelist=false,backend=biber,doi=false,url=false]{biblatex}
% \addbibresource{$BIB} % bibtex location
% \renewcommand*{\nameyeardelim}{\addcomma\space} % have comma in parencite
\usepackage{natbib}

\usepackage{xcolor}
\usepackage{amsmath}
\newcommand{\tuple}[1]{ \langle #1 \rangle }
%\usepackage{automata}
\usepackage{times}
\usepackage{ltablex}
\usepackage{tasks}

%%%%%% Template
\usepackage{hyperref}
\hypersetup{colorlinks=true,allcolors=blue}

\usepackage{vmargin}
\setpapersize{USletter}
\setmarginsrb{1.0in}{1.0in}{1.0in}{0.6in}{0pt}{0pt}{0pt}{0.4in}

% HOW TO USE THE ABOVE:
%\setmarginsrb{leftmargin}{topmargin}{rightmargin}{bottommargin}{headheight}{headsep}{footheight}{footskip}
%\raggedbottom
% paragraphs indent & skip:
\parindent  0.3cm
\parskip    -0.01cm

\usepackage{tikz}
\usetikzlibrary{backgrounds}

% hyphenation:
% \hyphenpenalty=10000 % no hyphen
% \exhyphenpenalty=10000 % no hyphen
\sloppy

% notes-style paragraph spacing and indentation:
\usepackage{parskip}
\setlength{\parindent}{0cm}

% let derivations break across pages
\allowdisplaybreaks

\newcommand{\orange}[1]{\textcolor{orange}{#1}}
\newcommand{\blue}[1]{\textcolor{blue}{#1}}
\newcommand{\red}[1]{\textcolor{red}{#1}}
\newcommand{\freq}[1]{{\bf \sf F}(#1)}
\newcommand{\datafreq}[2]{{{\bf \sf F}_{#1}(#2)}}

\def\qqquad{\quad\qquad}
\def\qqqquad{\qquad\qquad}

%%%%%%%%%%%%%%%%%%%%%%%%%%%%%%%%%%%%%%%%%%%%%%%%%%%%%%%%%%%%%%%%%%%%%%%%%%%%%%%%
%%%%%%%%%%%%%%%%%%%%%%%%%%%%%%%%%%%%%%%%%%%%%%%%%%%%%%%%%%%%%%%%%%%%%%%%%%%%%%%%

% fill-in-blank question style, found in https://tex.stackexchange.com/a/505089

\usepackage{ifthen}
\usepackage{tocloft}
\usepackage{exercise}
% \usepackage{xcolor}

% Set the Show Answers Boolean
\newboolean{showAns}
\setboolean{showAns}{false}
\newcommand{\showAns}{\setboolean{showAns}{true}}

% The length of the Answer line
\newlength{\answerlength}
\newcommand{\anslen}[1]{\settowidth{\answerlength}{#1}}

% ans command that indicates space for an answer or shows the answer in red
\newcommand{\ans}[1]{\settowidth{\answerlength}{\hspace{2ex}#1\hspace{2ex}}%
    \ifthenelse{\boolean{showAns}}%
        {\textcolor{red}{\underline{\hspace{2ex}#1\hspace{2ex}}}}%
        {\underline{\hspace{\answerlength}}}}%

% Formatting how multiple choices Questions are formated.
\settasks{label=(\Alph*), label-width=30pt}


% Some commands for the Exercise Question package
\renewcommand{\QuestionNB}{\Large\protect\textcircled{\small\bfseries\arabic{Question}}\ }
\renewcommand{\ExerciseHeader}{} %no header
\renewcommand{\QuestionBefore}{3ex} %Space above each Q
\setlength{\QuestionIndent}{8pt} % Indent after Q number


% To create the list of answers with tocloft...
\newcommand{\listanswername}{Answers}
\newlistof[Question]{answer}{Answers}{\listanswername}

% Creates a TOC for Answers
\newcounter{prevQ}
\newcommand{\answer}[1]{\refstepcounter{answer}%
\ans{#1}%
\ifnum\theQuestion=\theprevQ%
        \addcontentsline{Answers}{answer}{\protect\numberline{}#1}% don't include the Q number
        \else%
        \addcontentsline{Answers}{answer}{\protect\numberline{\theQuestion}#1}%
        \setcounter{prevQ}{\value{Question}}%
        \fi%
        }%


%tocloft formatting listofanswers
\renewcommand{\cftAnswerstitlefont}{\bfseries\large}
\renewcommand{\cftanswerdotsep}{\cftnodots}
\cftpagenumbersoff{answer}
\addtolength{\cftanswernumwidth}{10pt}


%%%%%%%%%%%%%%%%%%%%%%%%%%%%%%%%%%%%%%%%%%%%%%%%%%%%%%%%%%%%%%%%%%%%%%%%%%%%%%%%
%%%%%%%%%%%%%%%%%%%%%%%%%%%%%%%%%%%%%%%%%%%%%%%%%%%%%%%%%%%%%%%%%%%%%%%%%%%%%%%%
\begin{document}

% \setcounter{section}{}
\centerline{\huge\bf ECON 4002.01 Problem Set 3}
\smallskip
\centerline{\LARGE Hui-Jun Chen}

\medskip

\showAns
% \listofanswer

\section*{Question 1}
\label{sec:Question_1}
\addcontentsline{toc}{section}{Question 1}

Consider a model that is \textbf{similar to} (not exactly!) the Lecture 14 Consumer Problem, but there are three differences:

\begin{enumerate}
    \item Consumers' utility function is given by $ U(C, C', N_{S}, N_{S}') = \log C - b N_{S} + \log C' - b N_{S}' $
    \item Consumers do \textbf{not} own the whole firm; instead, they buy shares of the firm \blue{$ s $} in date $ 0 $ to achieve intertemporal saving at per-unit price \blue{$ q $}. At date $ 1 $, consumers redeem their share to the firm and get $ s $ of reward.
    \item Consumers are \textbf{not} subject to the lump-sum tax.
\end{enumerate}


\begin{Exercise}

Firstly, let's follow the slide and think about the consumer's budget constraint, you can refer to Lecture 14, slide 4.

\Question there are $ \answer{A} $ choice variables,
    \begin{tasks}(4)
        \task $ 5 $
        \task $ 3 $
        \task $ 2 $
        \task $ 4 $
    \end{tasks}
\Question and they are $ \{C, C', N_{S}, N_{S}', \answer{C} \} $
    \begin{tasks}(4)
        \task $ S $
        \task $ S' $
        \task $ s $
        \task $ s' $
    \end{tasks}

\Question consumers own \textbf{part} of the firm and get $ \answer{B} $  of reward
    \begin{tasks}(4)
        \task $ \pi $
        \task $ s $
        \task $ \pi' $
        \task $ S $
    \end{tasks}

\Question and they are taken the equilibrium price $ \{w, w', \answer{D}\} $ as given.
    \begin{tasks}(4)
        \task $ r $
        \task $ r' $
        \task $ q' $
        \task $ q $
    \end{tasks}

After defining all of the variables, consumer's budget constraints in each period are

\Question date $ 0 $ budget constraints is \answer{$A$}
    \begin{tasks}(2)
        \task $ C + q s  = w N_{S}$
        \task $ C + S = w N_{S} + \pi - T $
        \task $ C = w N_{S} + qs $
        \task $ C = w N_{S} + \frac{s}{q} + \pi - T $
    \end{tasks}

\Question date $ 1 $ budget constraints is \answer{$C$}
    \begin{tasks}(2)
        \task $ C' = w N_{S} + \pi' - T' + (1+r)S$
        \task $ C' = w' N_{S}' + q s $
        \task $ C' = w' N_{S}' + s$
        \task $ C' = w' N_{S}' + \frac{s'}{q'} + \pi' - T' $
    \end{tasks}


\Question The lifetime budget constraint by combining date $ 0 $ and date $ 1 $ budget constraints is \answer{$D$}
    \begin{tasks}(1)
        \task $ C + \frac{C'}{1+r} = w N_{S} + \frac{w' N_{S}'}{1+r} $
        \task $ C + \frac{C'}{1+r} = w N_{S} + \pi - T + \frac{w' N_{S}' + \pi' - T'}{1+r} $
        \task $ C - qC' = w N_{S} - q w' N_{S}' $
        \task $ C + qC' = w N_{S} + q w' N_{S}' $
    \end{tasks}

    Some calculation details:

    ~\answer{ $ s = C' - w' N_{S}' \Rightarrow C + q (C' - w' N_{S}') = w N_{S}$  }

    ~\answer{ $ \Rightarrow C + qC' = w N_{S} + q w' N_{S}' $ }

After finishing consumer's budget constraint, let's turn to the analysis preference:

\Question Consumer


\end{Exercise}




\end{document}
